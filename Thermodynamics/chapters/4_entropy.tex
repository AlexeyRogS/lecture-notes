\documentclass[../thermodynamics.tex]{subfiles}

\begin{document}

    \section{Multiplicity, Macrostates, \& Microstates}
        \paragraph{}
        Why does heat transfer only take place from hot to cold?
        Technically, the first law does not disallow heat flowing from cold to hot.
        If we want to examine how systems exchange energy energy, it makes sense to first look at how energy is spread out in a system.
        Consider a system which is comprised of many particles but which can be described by a few state parameters which describe the system as a whole.
        These bulk properties define the \textbf{macrostate} of the system (for example, having a certain temperature, pressure, volume).
        A different approach would be to specify the properties of every single particle e.g. the position and momentum.
        This defines the \textbf{microstate} of the system.
        For any given macrostate, there will be many possible microstates.
        The number of microstates consistent with the macrostate of the system is called the \textbf{multiplicity}.

        \paragraph{}
        For a isolated system in thermal equilibrium, all the accessible microstates are equally probable.
        For example, consider a system of $N$ coins.
        We can define the macrostate of the system as how many heads we have after tossing all of them.
        % TODO: include diagrams and tables for this

        \paragraph{}
        Imagine a 2-egg egg box and the macrostate of 2 eggs in the box.
        There is only 1 way to arrange these eggs so the multiplicity is 1.
        Consider increasing the volume of the box to 4 eggs.
        Now there are 6 different ways to arrange the two eggs in the box, so the multiplicity has increase.
        This is a general feature, multiplicity of a given macrostate will increase when the volume or ``size'' of the system is increased.
        % TODO: include diagram for this

        \paragraph{}
        Now consider an ideal gas at pressure $P$ and volume $V$ in thermal equilibrium with a reservoir at temperature $T$.
        If there is a movable frictionless insulating piston on top and the force on the pistonis reduced, the gas expands to volume $V+\mathrm{d}V$.
        The gas goes work, but its temperature stays constant, so $\Delta U=0$.
        Thus, we have
        \begin{equation}
            \dbar Q=-\dbar Q_R,
        \end{equation}
        where $Q_R$ stands for heat added in a reversible process.
        % TODO: include diagram
        The larger the volume, the greater the multiplicity.
        Let $x$ be the number of positions available for the particles.
        Note that $x>>N$ since we have an ideal gas which must be at low density.
        There are approximately $x^N$ ways of arranging the particles, so the multiplicity is proportional to $x^N$
        \begin{equation}
            \Omega\propto x^N.
        \end{equation}
        $x$ is proportional to $V$, so $\Omega\propto V^N$.
        Hence we can write
        \begin{equation}
            \frac{\Omega_{V+\mathrm{d}V}}{\Omega_V}=\left(\frac{V+\mathrm{d}V}{V}\right).
        \end{equation}
        Since we know from the last chapter that the work done by the gas is $\dbar W=P\mathrm{d}V$, we have that
        \begin{align}
            \dbar Q_R&=P\mathrm{d}V\\
            &=Nk_BT\frac{\mathrm{d}V}{V}\\
            \implies\frac{\mathrm{d}{V}}{V}&=\frac{\dbar Q_R}{Nk_BT}.
        \end{align}
        Combining this with the relation above we get
        \begin{equation}
            \ln\left(\frac{\Omega_{V+\mathrm{d}V}}{\Omega_V}\right)=N\ln\left(1+\frac{\dbar Q_R}{Nk_BT}\right).
        \end{equation}
        Note that $\frac{\dbar Q_R}{Nk_BT}<<1$, so we can use the Taylor expansion for $\ln(1+x)\approx x$, and we get
        \begin{equation}
            k_B\ln\Omega_{V+\mathrm{d}V}-k_B\ln\Omega_V=\frac{\dbar Q_R}{T}.
        \end{equation}

        \paragraph{}
        What about free expansion?
        As we saw before, no work is done by the gas and there is no heat transfer to the gas, so $\Delta U=0$.
        The start and end macrostates under an expansion from $V$ to $V+\mathrm{d}V$ are the same for the isothermal reversible expansion considered above.
        Thus the quantity $k_B\ln\Omega_{V+\mathrm{d}V}-k_B\ln\Omega_V$ must be the same for both processes.

        \paragraph{}
        Consider adding a small amount of heat to an ideal gas but keeping the volume constant.
        The average momentum of the particles changes, so the number of ways that momenta can be assigned to individual particles changes, meaning the multiplicity changes.
        Assuming that $U=\frac{3}{2}k_BT$, we have
        \begin{align}
            \frac{3}{2}k_BT&=\frac{\bar{p}^2}{2m}\\
            \implies\bar{p}&\propto\sqrt{T}.
        \end{align}
        The number of allowed momentum values in any given direction is proportional to $\bar{p}$ (the width of the probability curve is proportional to $\bar{p}$).
        Thus the total number of allowed momentum values, which is proportional to the muliplicity, is given by
        \begin{equation}
            \Omega\propto\bar{p}^3\propto T^\frac{3}{2}.
        \end{equation}
        For $N$ particles, this means
        \begin{equation}
            \Omega\propto T^\frac{3N}{2}.
        \end{equation}
        From everything we have seen, we can say that the general expression for multiplicity is
        \begin{align}
            \Omega&=F(N)V^NT^\frac{3N}{2}
            \ln\Omega&=\ln F(N)+N\ln V+\frac{3N}{2}\ln T.
        \end{align}
        Note that
        \begin{equation}
            \ln(T+\mathrm{d}T)-\ln T=\frac{\mathrm{d}T}{T},
        \end{equation}
        so
        \begin{align}
            k_B\ln\Omega_{T+\mathrm{d}T}-k_B\ln\Omega_T&=\frac{3N}{2}k_B\frac{\mathrm{d}T}{T}\\
            k_B\ln\Omega_{T+\mathrm{d}T}-k_B\ln\Omega_T&=\frac{\dbar Q_R}{T},
        \end{align}
        where in the last line we have used $\mathrm{d}U=\dbar Q_R$.

        \paragraph{}
        What we have shown is that this quantity $k_B\ln\Omega$, which we will call $S$ is a state variable (path independent).
        We call $S$ the \textbf{entropy}.
        We have found that the change in entropy is the same for slow heating, where heat is added reversibly, and in free expansion, where no heat is added.
        In general, for any isothermal process, we have
        \begin{equation}
            \Delta S\geq\frac{Q}{T}.
        \end{equation}
        \begin{definition}[Second Law of Thermodynamics]
            The entropy of a thermally isolated system increases in any irreversible process and stays the same in a reversible one.
            \begin{equation}
                \Delta S\geq 0.
            \end{equation}      
        \end{definition}
        An isolated system will always evolve to the state of highest entropy and stay there.
        The energy in the system ``spreads out''.
        We now have a new definition for a reversible process, one where the entropy of the system and the rest of the universe stays the same.
        \begin{example}
            Consider two blocks in thermal contact between two thermal reservoirs of temperature $T_1$ and $T_2$ respectively.
            If $T_2>T_1$, then heat will flow through the blocks.
            We have that
            \begin{align}
                \Delta S_1&=\frac{Q}{T_1}\\
                \Delta S_2&=-\frac{Q}{T_2}.
            \end{align}
            But since we know $\abs{\Delta S_1}>\abs{\Delta S_2}$, we have
            \begin{equation}
                \Delta S_1+\Delta S_2>0.
            \end{equation}
        \end{example}

    \section{Entropy \& Temperature}
        \paragraph{}
        Recall that when a gas is heated slowly at constant volume,
        \begin{equation}
            \mathrm{d}S=\frac{\dbar Q_R}{T}.
        \end{equation}
        In this case, the work done is 0, so by the first law we have
        \begin{equation}
            \mathrm{d}S=\frac{\mathrm{d}U}{T},
        \end{equation}
        or alternatively,
        \begin{equation}
            \left(\pder{S}{U}\right)_{V,N}=\frac{1}{T}.
        \end{equation}
        From this we get an alternative definition of the first law:
        \begin{equation}
            \mathrm{d}{U}=T\mathrm{d}S-P\mathrm{d}V.
        \end{equation}
        This is the first law defined completely in terms of state variables.
        To calculate entropy changes for a reversible process, we have
        \begin{equation}
            \Delta S=\int\mathrm{d}S=\int\frac{\dbar Q_R}{T}.
        \end{equation}
        If our process is irreversible, free expansion for example, then we need to identify an equivalent reversible process with the same start and end conditions.
        This works because we know the entropy change will always be the same.

        \paragraph{}
        At constant volume with no work done, we have
        \begin{align}
            \mathrm{d}S&=\der{U}{T}\frac{\mathrm{d}T}{T}\\
            &=C_V\frac{\mathrm{d}T}{T}\\
            \implies C_V&=T\left(\pder{S}{T}\right)_V.
        \end{align}

\end{document}
