\documentclass[../thermodynamics.tex]{subfiles}

\begin{document}

    \section{Thermal Equilibrium}
        \paragraph{}
        When two systems are in \textbf{thermal contact}, they are able to exchange energy with each other.
        The one that loses energy is said to be hotter/higher temperature.
        If two systems are in \textbf{thermal equilibrium}, it means that there is no \textit{net} energy transfer.

    \section{The 0\textsuperscript{th} Law}
        \begin{definition}[0\textsuperscript{th} Law of Thermodynamics]
            If two systems are each in thermal equilibrium with a third system, then they must be in thermal equilibrium with each other.
        \end{definition}
        % TODO: include diagram of 0th law

    \section{Measuring Temperature}
        \paragraph{}
        To measure temperature, we can make use of the fact that properties of materials are often a function of temperature.
        We pick a property of a system, measure it, and compare to measurments made at other known temperatures.
        For example, the Kelvin temperature scale uses the triple point of water (defined as 273.16K) and absolute zero to define a scale.
        A good thermometer for calibrating is a fixed-volume gas thermometer, because there is a linear relationship between pressure and temperature (for an ideal gas).

    \section{The 1\textsuperscript{st} Law}
        \paragraph{}
        We denote the \textbf{internal energy} of a system $U$.
        Energy can be transfered into or out of the system by \textbf{heat} ($Q$) or \textbf{work} ($W$).
        Heat is the energy transfered due to a difference in temperature, and work is the usable mechanical energy.
        \begin{definition}[1\textsuperscript{st} Law of Thermodynamics]
            The change in internal energy of a system is equal to the sum of the change in heat and the work done on the system.
            \begin{equation}
                \Delta U=Q+W
            \end{equation}
        \end{definition}
        % TODO: note about sign convention

        \paragraph{}
        If $Q$ is positive, the system is being heated up, if $Q$ is negative, the system is heating something else.
        Likewise, if $W$ is positive, then work is being done on the system and if $W$ is negative, then work is being done by the system.

    \section{Crystal Structure \& Interatomic Forces}
        \paragraph{}
        Without interatomic forces, everything would be a gas.
        In a solid, energy of the atomc is minimised by the bonding.
        For example, in ionic bonding, an electron is transfered from one atom to another, and the pair are held together by Coulomb's law.
        In covalent bonding, the energy of bonding electrons minimised by sharing between atoms.
        There is also the van der Whaals interaction which is due to dipole-dipole interactions.

        \paragraph{}
        Consider two atoms in a solid. When they get very close, there must be a repulsive force (otherwise matter would be infinitely dense).
        However, the force must become attractive further away (but still very close on a macroscopic scale).
        A simple model of an intermolecular force like this is the \textbf{Lennard-Jones potential}.
        It is given by
        \begin{equation}
            V(x)=4\alpha\left[\left(\frac{\beta}{x}\right)^{12}-\left(\frac{\beta}{x}\right)^6\right]
        \end{equation}
        % TODO: include diagram of potential
        This potential only really describes bonding between electrically neutral atoms or molecules, however the same is very similar for all types of bonding.

        \paragraph{}
        Close to the equilibrium separation, we can describe the Lennard-Jones potential can be approximated with a parabola.
        \begin{align}
            V&\propto(x-x_0)^2\\
            \implies F&\propto-(x-x_0).
        \end{align}
        When looking at the stretching of materials, we can define two quantities.
        The \textbf{stress} is defined as the force per unit area.
        \begin{equation}
            \sigma=\frac{F}{A}.
        \end{equation}
        The \textbf{strain} is defined as the proportional deformation of the material.
        \begin{equation}
            \varepsilon=\frac{\Delta L}{L}.
        \end{equation}
        We can then define Young's modulus, which is a property that measures the stiffness of a solid material.
        \begin{equation}
            E=\frac{\sigma}{\varepsilon}.
        \end{equation}

    \section{Thermal Expansion}
        \paragraph{}
        Near absolute zero, the average spacing of atomcs is near $x_0$, but as energy is increased, more atoms will be sitting at higher energies so the average atomic separation will get larger (because of the asymmetry of the potential).
        % TODO: include diagram of thermal expansion
        Let $L_0$ be the length of a solid bar at a temperature $T$.
        Suppose that the length of the bar increases linearly with temperature (first approximation).
        If the temperature increases by $\Delta T$, the length increases by $\Delta L$ and we write
        \begin{equation}
            \frac{\Delta L}{L_0}=\alpha\Delta T.
        \end{equation}
        $\alpha$ is called the \textbf{coefficient of linear expansion} and may be different at different temperatures.
        Analogously, if a volume $V_0$ increases by $\Delta V$, then we can write
        \begin{equation}
            \frac{\Delta V}{V_0}=\beta\Delta T,
        \end{equation}
        where $\beta$ is the \textbf{coefficient of volume expansion}.
        How is $\beta$ related to $\alpha$?
        \begin{align}
            V_1&=V_0+\Delta V\\
            &=(1+\beta\Delta T)V_0.
        \end{align}
        Similarly, $L_1=(1+\alpha\Delta T)L_0$, and using $V_1=L_1^3$, we get
        \begin{align}
            V_1&=L_0^3(1+\alpha\Delta T)^3\\
            1+\beta\Delta T&=(1+\alpha\Delta T)^3\\
            1+\beta\Delta T&\approx 1+3\alpha\Delta T\\
            \beta&\approx 3\alpha.
        \end{align}
        Where in the third line we have used the approximation $(1+x)^n\approx 1+nx$ for small $x$.
        \begin{example}
            Consider a doughnut-shaped block of material which expands linearly with temperature.
            It has an initial inner and outer diameter $r_0$ and $R_0$ respectively.
            Find the new area of the block in terms of $r_0$ and $R_0$ after a temperature increase of $\Delta T$, and hence show that they increase linear with the change in temperature.
            % TODO: include diagram
            % TODO: complete this example
        \end{example}

    \section{Heat Transfer}
        \paragraph{}
        The transfer of energy by heating is always due to a temperature difference.
        The specific mechanism depends on the the material properties.
        For example, in a fluid the heating is caused by molecular collisions which transfer energy.
        In solids, energy is transfered by electrons or lattice vibrations known as phonons, depending on whether the material is electrically insulating or not.
        Fourier's heat conduction law states that
        \begin{equation}
            \frac{Q}{\Delta t}=P=-\kappa A\der{T}{x}.
        \end{equation}
        $P$ is the power transfered and $\kappa$ is the thermal conductivity (measured in W/m/K).
        Under \textbf{steady state} conditions, meaning the temperature of the system and the surroundings at every point does not change with time (there may still be energy transfer), the law becomes
        \begin{equation}
            P=-\kappa A\frac{\Delta T}{\Delta x}.
        \end{equation}
        An object that is so big that its temperature can be assumed to remain constant, no matter how much we heat or cool it, is known as a \textbf{thermal reservoir}.
        For example, a cup of tea will exchange heat with the air in the room that it sits in. It will cool down, but the room will not notically heat up.
        \begin{example}
            Consider three solid blocks between two thermal reservoirs which have temperatures $T_1$ and $T_2$ respectively.
            % TODO: complete this example
        \end{example}
        \begin{example}
            Consider two concentric cylinders.
            % TODO: complete this example
        \end{example}

    \section{Radiation}
        \paragraph{}
        When it comes to heat transfer by radiation, the most important equation is the \textbf{Stefan-Boltzmann Law},
        \begin{equation}
            P=\sigma\varepsilon AT^4.
        \end{equation}
        $A$ is the surface area of the object, $T$ is the surface temperature, $\varepsilon$ is the emissivity, a dimensionless constant between 0 and 1 that determines how efficiently the object radiates energy, and $\sigma$ is the Stefan-Boltzmann constant.
        A perfectly reflecting object has $\varepsilon=0$ and a perfectly absorbing and emitting object has $\varepsilon=1$.

\end{document}
