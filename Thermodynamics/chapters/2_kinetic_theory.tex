\documentclass[../thermodynamics.tex]{subfiles}

\begin{document}

    \section{The Ideal Gas Law}
        \paragraph{}
        An ideal gas it one where the particles can be considered point-like, meaning the average distance between particles is very large compared to particle size.
        We also assume that the particles interact only in perfectly elastic collisions (there are no intermolecular forces).
        This might sound like a very simplistic set of assumptions, but it turns out to be a very useful model.
        From these assumptions, we get the ideal gas law:
        \begin{equation}
            PV=nRT,
        \end{equation}
        where $R$ is the gas constant, equal to $N_AK_B$. Using this we can write the physicists version of the ideal gas law:
        \begin{equation}
            PV=NK_BT.
        \end{equation}
        The ideal gas law is an example of an equation of state, one that relates state variables of a system to each other.
        Due to the assumptions that we have made, this equation only holds in the real world for gases that are in equilibrium (or very close to it).
        Note that a ``state variable'' is a property of the system that depends only on the current state, and not the past states.

        \paragraph{}
        Let's focus on a single particle in an ideal gas in a container.
        If the particle hits the wall in the $y-z$ plane, only the $x$ component of the velocity will change since it is an elastic collision.
        Thus the pressure on the wall is due to the $x$ component of the momentum.
        The particle's momentum changes by $\Delta p_{x,i}=-2mv_{x,i}$ so due to conservation of momentum there is $\Delta p_{x,i}=2mv_{x,i}$ imparted to the wall.
        The change in momentum is equal to the force multiplied by the time interval $\Delta p=F\Delta t$, so since pressure is defined as force per unit area, we get
        \begin{equation}
            \Delta p_{x,i}=P_iA\Delta t_i.
        \end{equation}
        $P_i$ is the pressure due to particle $i$ on the wall and $\Delta t_i$ is the time interval between collisions for particle $i$.
        Now if we imagine the particle bouncing back and forth in the box of width $L$, then he time between collisions is given by
        \begin{equation}
            \Delta t_i=\frac{2L}{v_{x,i}}.
        \end{equation}
        Substituting this into the equation above we get
        \begin{equation}
            P_iV=mv_{x,i}^2,
        \end{equation}
        where $V=AL$.
        The total pressure on the wall is simply equal to the sum of this quantity for all particles:
        \begin{equation}
            PV=\sum_{i=1}^{N}mv_{x,i}^2=m\underbrace{\sum_{i=1}^{N}v_{x,i}^2}_{N\bar{v_x^2}}
        \end{equation}
        The bar above the velocity denotes the mean.
        Now, note that
        \begin{equation}
            \bar{v^2}=\bar{v_x^2}+\bar{v_y^2}+\bar{v_z^2},
        \end{equation}
        so assuming the velocity of the particles is \textbf{isotropic} (meaning there is not preferred direction),
        \begin{equation}
            \bar{v_x^2}=\frac{1}{3}\bar{v^2},
        \end{equation}
        and hence by using the ideal gas law,
        \begin{align}
            PV=\frac{1}{3}Nm\bar{v^2}&=Nk_BT\\
            \implies\bar{v^2}&=\frac{3k_BT}{m}.
        \end{align}
        If we take the square root of both sides, we get the \textbf{root mean square velocity}:
        \begin{equation}
            v_\text{rms}=\sqrt{\frac{3k_BT}{m}}.
        \end{equation}
        % TODO: add diagrams for this section

    \section{Equipartition Theorem}
        \paragraph{}
        The internal energy of a gas is a state variable.
        It is the sum of all of the types of energy contained in the gas; kinetic, potential, mass, and chemical.
        For a monatomic ideal gas, we only have kinetic energy, and moreover the kinetic energy is purely translational:
        \begin{equation}
            U=\frac{1}{2}Nm\bar{v^2}=\frac{3}{2}Nk_BT.
        \end{equation}
        Note that this is a constant times $N$, meaning that on average each atom has an energy of $\frac{3}{2}k_BT$, or $\frac{1}{2}k_BT$ for each translational degree of freedom.
        This is known as the \textbf{equipartition theorem}.
        In general, an ideal gas with $N$ particles and $f$ quadratic degrees of freedom will have an internal energy of
        \begin{equation}
            U=\frac{f}{2}Nk_BT.
        \end{equation}
        Note that a quadratic degree of freedom is one where the associated energy is proportional to the square of a position of momentum variable.
        For example, translation motion $E=\frac{1}{2}mv^2$, elastic potential energy $E=\frac{1}{2}kx^2$, or rotational kinetic energy $E=\frac{1}{2}I\omega^2$.
        A diatomic gas has 3 translational degrees of freedom, along with 2 rotational 2 vibrational (one for the kinetic energy and one for the potential energy).
        This makes 7 in total, however at lower temperatures some of these modes will not be excited and the degrees of freedom will be ``frozen out''.
        For example, at room temperature, diatomic oxygen and nitrogen have 5 degrees of freedom.

        \paragraph{}
        Let's look at the average distance between collisions, also known as the \textbf{mean free path}.
        A collision between two particles occurs if a particle comes within a diameter $d$ of another.
        Identically, we can consider a particle with cross-sectional area $2d$ and ask about the probability that a particle enters the volume traced out:
        \begin{align}
            \pi d^2\lambda\frac{N}{v}&=1\\
            \implies\lambda&=\frac{V}{N\pi d^2}.
        \end{align}
        If we remember that all particles are moving, this gets reduced by a factor of $\frac{1}{\sqrt{2}}$:
        \begin{equation}
            \lambda=\frac{V}{\sqrt{2}N\pi d^2}=\frac{1}{\sqrt{2}n\pi d^2}.
        \end{equation}
        Note that we have defined the number density $n=\frac{N}{V}$.

        \paragraph{}
        It can be shown that in an ideal gas, the velocities are distributed according to the \textbf{Maxwell-Boltzmann Distribution}.
        This takes the form
        \begin{equation}
            p(v)=\sqrt{\frac{2}{\pi}}\left(\frac{m}{k_BT}\right)^\frac{3}{2}v^2\exp\left(-\frac{mv^2}{2k_BT}\right),
        \end{equation}
        where $p(v)$ is the probability density of a single particle having speed $v$.
        Note that this probability distribution is properly normalised so that $\int_{0}^{\infty}p(v)\mathrm{d}v=1$.
        At higher temperatures, the tail drops off much less i.e. there are a much wider range of velocities.
        % TODO: include diagram for this
        From this distribution, we can measure some characteristic speeds.
        The most probable speed is found at the peak of the distribution, which is found by differentiating:
        \begin{align}
            \der{p}{v}&=0=2Ave^{-av^2}(1-av^2)\\
            \implies av_\text{mp}^2&=1\\
            v_\text{mp}&=\sqrt{\frac{1}{a}}=\sqrt{\frac{2k_BT}{m}}.
        \end{align}
        The mean speed will be to the right of the most probable speed since the distribution is asymmetric.
        It is found by integrating the distribution multiplied by $v$:
        \begin{align}
            \bar{v}=\int_{0}^{\infty}vp(v)\mathrm{d}v&=A\int_{0}^{\infty}v^3e^{-av^2}\mathrm{d}v\\
            &=\frac{1}{2}Aa^{-2}\\
            &=\sqrt{\frac{8k_BT}{\pi m}}.
        \end{align}
        Finally, we can calculate the root mean square speed just to check our answer from before:
        \begin{align}
            v_\text{rms}=\sqrt{\int_{0}^{\infty}v^2p(v)\mathrm{d}v}&=\sqrt{A\int_{0}^{\infty}v^4e^{-av^2}\mathrm{d}v}\\
            &=\sqrt{A\frac{3}{8}}\sqrt{\pi}a^{-\frac{5}{2}}\\
            &=\sqrt{\frac{3k_BT}{m}}.
        \end{align}

\end{document}
