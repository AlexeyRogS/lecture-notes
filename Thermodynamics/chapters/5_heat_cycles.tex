\documentclass[../thermodynamics.tex]{subfiles}

\begin{document}

    \section{Heat Engines}
        \paragraph{}
        A heat engine is a generic name for a system that absorbs heat and converts some of it into useful work.
        Heat engines contain a working substance which we do processes on.
        The process in the engine must be cyclic to be practical.
        Heat is added to the working substance, the entropy of the working substance increases, then we dump the extra entropy into a cold reservoir to return the substance to its original state.
        % TODO: include digram

        \paragraph{}
        An alternative way of stating the second law is that no series of processes which transfer energy to a body as heat and the completely converts the energy to work is possible.
        One of the most historically significant and well-known heat engines is the \textbf{Carnot cycle}.
        \begin{enumerate}[(i)]
            \item Isothermal expansion, gas is going work on outside world.
            \item Adiabatic expansion, gas is doing work again.
            \item Isothermal compression, outside world is doing work on the gas.
            \item Adiabatic compression, outside world is doing work on the gas again.
        \end{enumerate}
        The work extracted from this cycle is given by the area enclosed by the process in a $P$-$V$ diagram.
        % TODO include diagram
        The isothermal expansion is propelled by absorption of heat.
        In the second phase, the gas is allowed to expand, doing work and losing energy.
        The the gas is cooled by the reservoir and compressed, which takes less energy to the expansion phase due to the gas being at lower pressure.
        Finally, the gas is adiabatically compressed to bring its internal energy back to where it started.
        Let's look at the change in entropy during the Carnot cycle.
        We already know what the entropy change is from processes we have studied before.
        In the isothermal expansion phase, the entropy change is
        \begin{equation}
            \Delta S_\text{gas}=\frac{Q_\text{hot}}{T_\text{hot}},\quad\Delta S_\text{hot}=-\frac{Q_\text{hot}}{T_\text{hot}},
        \end{equation}
        where $Q_\text{hot}$, is the heat transferred from the hot reservoir and $T_\text{hot}$ is the temperature of the hot reservoir.
        In the next phase, the expansion is adiabatic so the entropy change is zero.
        \begin{equation}
            \Delta S_\text{gas}=0.
        \end{equation}
        Now in the isothermal compression phase, the gas is exchanging heat with the cold reservoir:
        \begin{equation}
            \Delta S_\text{gas}=-\frac{Q_\text{cold}}{T_\text{cold}},\quad\Delta S_\text{cold}=\frac{Q_\text{cold}}{T_\text{cold}}.
        \end{equation}
        Finally, in the last phase, the entropy change is once again zero since the compression is adiabatic.
        \begin{equation}
            \Delta S_\text{gas}=0.
        \end{equation}
        In total, since the gas starts and finishes in the exact same configuration, the entropy change over the whole cycle must be zero.
        This implies that
        \begin{equation}
            \frac{Q_\text{hot}}{T_\text{hot}}=\frac{Q_\text{cold}}{T_\text{cold}}.
        \end{equation}

        \paragraph{}
        We can define the efficiency of the Carnot cycle as the work done by the gas during phase 2 compared to the heat transferred from the hot reservoir to the gas.
        \begin{equation}
            \varepsilon=-\frac{W_\text{cycle}}{Q_\text{hot}}.
        \end{equation}
        By the first law of thermodynamics we have
        \begin{equation}
            Q_\text{hot}-Q_\text{cold}+W_\text{cycle}=0,
        \end{equation}
        since the total change in internal energy is zero.
        By dividing this equation by $Q_\text{hot}$ and rearranging, we get
        \begin{align}
            \varepsilon&=1-\frac{Q_\text{cold}}{Q_\text{hot}}\\
            &=1-\frac{T_\text{cold}}{T_\text{hot}}.
        \end{align}
        It can be proven that this is actually an upper limit on the efficiency of \textit{any} reversible process!

    \section{Refrigerators}
        \paragraph{}
        We can also model thermodyamic cycles which \textit{remove} energy from a substance (i.e. a refrigerator) by looking at a heat engine in reverse.
        For example, a simple model of a refrigerator is as a Carnot cycle in reverse.
        The refrigerator remove energy from a cold reservoir and dumps it in a hot reservoir by using work.
        A hypothetical perfect fridge would not need energy to run i.e. $W=0$.
        In this case, the entropy changes of the hot and cold reservoirs are given by
        \begin{equation}
            \Delta S_\text{cold}=-\frac{Q}{T_\text{cold}},\quad\Delta S_\text{hot}=\frac{Q}{T_\text{hot}}.
        \end{equation}
        Note that $Q_\text{cold}=Q_\text{hot}=Q$ since $W=0$.
        Then the total entropy change of the entire system is
        \begin{align}
            \Delta S_\text{total}&=\Delta S_\text{cold}+\Delta S_\text{hot}\\
            &=\frac{Q}{T_\text{hot}}-\frac{Q}{T_\text{cold}}.
        \end{align}
        Since $T_\text{hot}>T_\text{cold}$, $\Delta S_\text{total}<0$ which is disallowed by the second law, so a perfect fridge \textit{cannot} exist.
        In other words, the second law says that no series of processes that transfers energy from a cold body to a hot body purely by heat transfer is possible.

        \paragraph{}
        We can define a coefficient of performance analogously to the efficiency of a heat engine as the heat extracted from the cold reservoir divided by the work input to the cycle.
        \begin{equation}
            \kappa=\frac{Q_\text{cold}}{W_\text{cycle}}.
        \end{equation}
        Using the first law again just like before, we get
        \begin{align}
            W_\text{cycle}&=Q_\text{hot}-Q_\text{cold}\\
            \kappa&=\frac{Q_\text{cold}}{Q_\text{hot}-Q_\text{cold}}.
        \end{align}

    \section{Heat Pumps}
        \paragraph{}
        For a heat pump, where the objective of the cycle is to take energy from a cold reservoir and ``pump'' it into a hot reservoir, the cycle is equivalent to that of a refrigerator.
        We define the coefficient of performance differently to reflect the different purpose of the cycle.
        \begin{align}
            \text{C.O.P.}&=\frac{Q_\text{hot}}{W}\\
            &=\frac{Q_\text{cold}+W}{W}\\
            &=\frac{T_\text{hot}}{T_\text{hot}-T_\text{cold}}.
        \end{align}
        % TODO: include diagram for this

\end{document}