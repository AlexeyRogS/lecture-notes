\documentclass[../abstract_algebra.tex]{subfiles}

\begin{document}

    \section{Introduction}
        \paragraph{}
        The objective of the field of abstract algebra is to create a general theory of algebraic structures so we can study algebraic properties of mathematical objects we are familiar with.
        Some examples of these are integers, real numbers, matrices, and polynomials.
        But what is an algebraic structure?
        An algebraic structure in its most basic form is a \textbf{set} of mathematical objects with some \textbf{operations} that act on the members of the set with some given constraints.
        In abstract algebra there are three main types of algebraic structure: \textbf{groups}, \textbf{rings}, and \textbf{fields}.
        These abstract structures generalise structures that we may be more familiar with such as the set of integers or matrices with addition and multiplication.
        Abstraction helps us investigate similarities and relationships between more familiar structures.

    \section{Binary relations}
        \paragraph{}
        Binary relations are how we formalise relationships between elements of sets.
        \begin{definition}
            Let $A$ and $B$ be sets. Then a \textbf{binary relation} $R$ over $A$ and $B$ is a subset of the Cartesian product of $A$ with $B$. Let $(a, b) \in A \times B$. Then we say $a$ is \textit{R-related} to $b$ and write $aRb \iff (a, b) \in R$. 
        \end{definition}
        \begin{example}
            Let $D \subseteq \mathbb{N} \times \mathbb{P}$ be a binary relation on $\mathbb{N}$ and $\mathbb{P}$ - the set of prime numbers - given by $(n, p) \in D \iff n|p$. Clearly then, by definition the prime numbers, for any given prime $p$ we have $(p, p) \in D$ and $(1, p) \in D$ and nothing else. So $D = \{(p, p): p \in \mathbb{P}\} \cup \{(1, p): p \in \mathbb{P}\}.$
            % TODO: explain this example more
        \end{example}
        Binary relations over a set and itself are called \textbf{homogeneous}. 
        Homogeneous relations can be characterised by several different properties, of which we will now define some.
        \begin{definition}\label{homogeneous-relation-types}
            Let $X$ be a set, $R \subseteq X^2$ a binary relation.
            \begin{itemize}
                \item $R$ is \textbf{reflexive} $\iff \forall x \in X,\  xRx$.
                \item $R$ is \textbf{symmetric} $\iff \forall x, y \in X,\ xRy \implies yRx$.
                \item $R$ is \textbf{antisymmetric} $\iff \forall x, y \in X,\ xRy\ \text{and}\ yRx \implies x=y$.
                \item $R$ is \textbf{transitive} $\iff \forall x, y, z \in X, xRy\ \text{and}\ yRz \implies xRz$.
            \end{itemize}
        \end{definition}
        
        \paragraph{}
        A \textbf{function} (or \textbf{mapping}) is a type of binary relation and a \textbf{binary operation} is a type of function.
        % TODO: discuss closure of binary operations

    \section{Rings}
        \paragraph{}
        We now give the definition of a ring which is like a generalisation of the integers.
        \begin{definition}
            A \textbf{ring} is a set $R$ equipped with two binary operations denoted $+$ and $\cdot$ (which we usually call ``addition'' and ``multiplication'') which have the following properties:
            \begin{enumerate}[(i)]
                \item $(a+b)+c=a+(b+c)$\quad$\forall a,b,c\in R$\quad($+$ is associative)
                \item There exists an element in $R$ denoted $0$ such that $0+a=a+0=a\quad\forall a\in R$\quad($+$ has an identity element)
                \item $\forall a\in R$, there exists an element $-a\in R$ such that $a+(-a)=0$\quad(each element in $R$ has an additive inverse)
                \item $a+b=b+a$\quad$\forall a,b\in R$\quad($+$ is commutative)
                \item $(a\cdot b)\cdot c=a\cdot(b\cdot c)$\quad$\forall a,b,c\in R$\quad($\cdot$ is associative)
                \item There exists an element in $R$ denoted $1$ such that $1\cdot a=a\cdot 1=a\quad\forall a\in R$\quad($\cdot$ has an identity element)
                \item $a\cdot(b+c)=a\cdot b+a\cdot c$ and $(a+b)\cdot c=a\cdot c +b\cdot c\quad\forall a,b,c\in R$\quad($\cdot$ distributes over $+$)
            \end{enumerate}
        \end{definition}
        % TODO: split these axioms into groups

        \paragraph{}
        A note on closure: binary operations are closed by default, meaning that a ring is automatically closed under both $+$ and $\cdot$.
        This is the first thing we should check if we want to know if something is a ring.
        An observation we can make from the definition is that rings are non-empty by definition (they contain 0 and 1).
        % TODO: can 0=1?

        \paragraph{}
        Let's now look at some consequences of the axioms. Let $R$ be a ring and let $a_1$, $a_2$, $\dots,a_n\in R$.
        Then $a_1+a_2+\dots+a_n$ is a well-defined element in $R$ through repeated use of the associativity axiom for addition (proof by induction).
        Similarly, repeated use of the commutativity axiom allows us to rearrange the terms.
        Brackets can also be removed in a product $a_1\cdot a_2\cdot\dots\cdot a_n$, although here we cannot rearrange the terms since multiplication is not necessarily commutative.
        Often, we don't write $\cdot$ to denote a multiplication and simply write it as a juxtaposition i.e. $ab\in R$.

        \begin{example}
            Let $R$ be a ring and let $M_{n\times n}(R)$ be the set of $n\times n$ matrices with entries in $R$.
            Let $A\in M_{n\times n}(R)$ be denoted $[a_{ij}]$, $B\in M_{n\times n}(R)=[b_{ij}]$.
            Then we define two binary operations $+$ and $\cdot$ by
            \begin{align}
                A+B&=[a_{ij}+b_{ij}],\\
                AB&=[c_{ij}]\text{, where } c_{ij}=\sum_{k=1}^{n}a_{ik}b_{kj}.
            \end{align}
            These are simply the familiar matrix addition and multiplication.
            We can now show that $M_{n\times n}(R)$ forms a ring with these operations.
            
            \paragraph{}
            First note that if $A$ and $B$ are $n\times n$ matrices over $R$, then $A+B$ and $AB$ are also $n\times n$ matrices over $R$, so $M_{n\times n}(R)$ is closed under these operations.
            We can now verify each of the ring axioms:
            % TODO: prove this
        \end{example}

        \paragraph{}
        A commutative ring is a ring $R$ which satisfies
        \begin{equation}
            a\cdot b=b\cdot a\quad\forall a,b\in R\quad\text{($\cdot$ is commutative)}.
        \end{equation}
        % TODO: replace number with axiom label
        So a matrix ring like $M_{n\times n}(R)$ is not a commutative ring, but $\ZZ$ is.

        \begin{example}
            Let $R$ be a ring. A \textbf{polynomial} over $R$ is an expression
            \begin{equation}
                \sum_{k=0}^{n}a_kx^k=a_0+a_1x+a_2x^2+\dots+a_nx^n,
            \end{equation}
            where $n\geq0$ and $a_1,a_2,\dots,a_n\in R$.
            If $i$ is the largest number such that $a_i\neq0$, then we say that the polynomial has \textbf{degree} i.
            Two polynomials $\sum_{k=0}^{n}a_kx^2,\sum_{k=0}^{m}b_kx^2$ are defined to be equal if $a_i=b_i$ $\forall i=0,\dots,\max\{n,m\}$.

            \paragraph{}
            We now define $R[x]$ to be the set of all polynomials over $R$.
            We can define the sum and product of two polynomials in the normal way we would expect; if $f(x)=\sum_{k=0}^{n}a_kx^k$ and $g(x)=\sum_{k=0}^{m}b_kx^k$, then
            \begin{align}
                f(x)+g(x)&=\sum_{k=0}^{\max\{n,m\}}(a_k+b_k)x^k\\
                f(x)\cdot g(x)&=\sum_{k=0}^{n+m}c_kx^k\text{, where } c_k=a_0b_k+a_1b_{k-1}+\dots+a_kb_0=\sum_{i=0}^{k}a_ib_{k-i}.
            \end{align}
            The last part comes from the distributive law.
            % TODO: note about coefficients higher than n being defined to be 0
            Just like matrices with entries in a ring, polynomials with coefficients in a ring also form a ring.
            % TODO: show this
        \end{example}

    \section{Subrings}
        \paragraph{}
        Let $R$ be a ring and let $S\in R$ such that $S$ is a ring under the same operations on $R$.
        Then $S$ is a \textbf{subring} of $R$.

        \begin{example}
            $\ZZ$ is a subring of $\QQ$, which is in turn a subring of $\RR$.
            Similarly, $M_{n\times n}(\ZZ)$ is a subring of $M_{n\times n}(\QQ)$ which is a subring of $M_{n\times n}(\RR)$ and $\ZZ[x]$ is a subring of $\QQ[x]$ which is a subring of $\RR[x]$.
        \end{example}
        % TODO: subring test?

    \section{Fields}
        \paragraph{}
        A field is simply an abstraction of the rational, real or complex numbers that we are used to, where we can add, substract, multiply and divide any numbers (except 0).
        A field can therefore be considered as a commutative ring with two additional properties.
        \begin{definition}
            A \textbf{field} is a commutative ring $F$, with operations $+$ and $\cdot$, which also satisfies two additional properties:
            \begin{enumerate}[(i)]
                \item $1\neq0$
                \item $\forall a\in F$ with $a\neq0$, there exists an element $a^{-1}\in F$ such that $a\cdot a^{-1}=a^{-1}\cdot a=1$ (each nonzero element in $F$ has a multiplicative inverse)
            \end{enumerate}
            % TODO: replace with labels
        \end{definition}
        % TODO: consequences of the axioms -a=(−1)a, if ab=0 then a or b must be 0

\end{document}
