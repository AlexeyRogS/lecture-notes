\documentclass[../abstract_algebra.tex]{subfiles}

\begin{document}

    % TODO: possibly move the section on homogeneous binary relations to here?
    \section{Binary relation recap}
        \paragraph{}
        We want some way of generalising the concept of equality for numbers as we know it.
        Looking at the properties for homogeneous binary relations (definition \ref{homogeneous-relation-types}) that we defined before, we can see that equality satisfies the properties of \textbf{reflexivity}, \textbf{symmetry}, and \textbf{transitivity}.
        \begin{definition}
            An \textbf{equivalence relation} is a relation $\sim $ on a set $A$ which satisfies the following properties:
            \begin{enumerate}[(i)]
                \item Reflexivity: $a\sim a$ $\forall a\in A$
                \item Symmetry: $a\sim b\implies b\sim a$ $\forall a,b\in A$
                \item Transitivity: $a\sim b$ and $b\sim c\implies a\sim c$ $\forall a,b,c\in A$
            \end{enumerate}
        \end{definition}
        
        \paragraph{}
        Let's look at some relations and see if they satisfy the definition of equivalence.
        \begin{example}
            Consider the ``divides'' relation on $\ZZ$ that we studied in the last chapter.
            \begin{itemize}
                \item If $a\in\ZZ$, then $a=1\cdot a$, so $a|a$. Hence $|$ is reflexive.
                \item If $a|b$ and $b|c$ for $a,b,c\in\ZZ$, then for some $x$ and $y$ in $\ZZ$, $b=ax$ and $c=by=axy$, so $a|c$ and hence $|$ is transitive.
                \item However, it is easy to see that $|$ is not symmetric, for example, $1|2$ but $2\slashed{|}1$.
            \end{itemize}
            So $|$ is not an equivalence relation.
        \end{example}
        \begin{example}
            Define a binary relation $\sim $ on $\ZZ$ by $a\sim b$ when $b-a$ is even for $a,b\in\ZZ$.
            \begin{itemize}
                \item If $a\in\ZZ$, then $a-a=0$, which is even, so $a\sim a$ and $\sim $ is reflexive.
                \item Let $a,b\in\ZZ$ and suppose $a\sim b$. Then $b-a=2x$ for some integer $x$, but then $a-b=-2x$ which is also even, so $b\sim a$ and $\sim $ is symmetric.
                \item Finally, let $a,b,c\in\ZZ$ and suppose $a\sim b$ ($b-a=2x$) and $b\sim c$ ($c-b=2y)$ for two integers $x$ and $y$. Then $c-a=(c-b)+(b-a)=2y+2x=2(x+y)$, so $a\sim c$ and $\sim $ is transitive.
            \end{itemize}
            Therefore, $\sim $ is an equivalence relation.
        \end{example}
        \begin{example}
            Let $A=\{(a,b)|a,b\in\ZZ,b\neq0\}$. Then define a binary relation $\sim$ on $\ZZ$ by $(a,b)\sim(c,d)$ when $ad=bc$.
            \begin{itemize}
                \item Let $(a,b)\in A$, then $ab=ba$ since $\ZZ$ is a commutative ring, so $(a,b)\sim(a,b)$.
                \item Let $(a,b),(c,d)\in A$ with $(a,b)\sim(c,d)$. Then $ad=bc$. Hence $cb=da$ and $(c,d)\sim(a,b)$.
                \item Let $(a,b),(c,d),(e,f)\in A$ with $(a,b)\sim(c,d)$ and $(c,d)\sim(e,f)$. So $ad=bc$ and $cf=de$. Then $afd=(ad)f=(bc)f=b(cf)=b(de)$ and since $d\neq0$, $af=bc$ so $(a,b)\sim(e,f)$.
            \end{itemize}
            Note that this example is equivalent to $\frac{a}{b}=\frac{c}{d}$, which is equality for rational numbers.
        \end{example}

    \section{Equivalence classes}
        \begin{definition}
            Let $\sim$ be an equivalence relation on a set $A$.
            If $a\in A$, then we define the \textbf{equivalence class} of $a$ as
            \begin{equation}
                [a]=\{b\in A|a\sim b\}.
            \end{equation}
            So $[a]$ contains all the elements of $A$ that are related to $a$.
        \end{definition}
        \begin{theorem}[Properties of equivalence classes]
            Let $\sim$ be an equivalence relation on a set $A$ and let $a\in A$.
            \begin{enumerate}[(i)]
                \item $a\in[a]$.
                \item $A$ is the union of all the equivalence classes of $\sim$.
                \item Let $b\in A$, then either $[a]=[b]$ or $[a]\cap[b]=\emptyset$.
            \end{enumerate}
        \end{theorem}
        \begin{proof}
            Let $a\in A$.
            \begin{enumerate}[(i)]
                \item Since $\sim$ is an equivalence relation, it is reflexive, so $a\sim a$ and thus $a\in[a]$.
                \item By (i), every element of $A$ belongs to at least one equivalence class, hence (ii) is true.
                \item Let $b\in A$. Suppose $[a]\cap[b]\neq\emptyset$. Then there exists $c\in[a]\cap[b]$.
                So $a\sim c$ and $b\sim c$ since $c$ is in both $[a]$ and $[b]$.
                Since $\sim$ is symmetric, $c\sim b$. Since $\sim$ is transitive, $a\sim c$ and $c\sim b$ implies $a\sim b$ (and $b\sim a$).
                Now, let $x\in[a]$, so $a\sim x$. Then by transitivity, $b\sim a$ and $a\sim x$ impliy $b\sim x$, so $x\in[b]$.
                Therefore, $[a]\subseteq[b]$.
                Similarly, it can be shown that $[b]\subseteq[a]$, so $[a]=[b]$.
            \end{enumerate}
        \end{proof}

        \paragraph{}
        There are two things which we learn from this proof that actually strengthen the statements in the theorem.
        \begin{corollary}
            Let $\sim$ be an equivalence relation on a set $A$ and let $a,b\in A$.
            \begin{enumerate}[(i)]
                \item $A$ is the \textbf{disjoint} union of the equivalence classes of $\sim$.
                \item $[a]=[b]$ if and only if $a\sim b$ (or equivalently, $[a]\cap[b]=\emptyset\iff a\slashed{\sim}b$).
            \end{enumerate}
        \end{corollary}

    \section{Partitions}
        \paragraph{}
        Now we will look at ways of partitioning a set into disjoint subsets.
        \begin{definition}
            Let $A$ and $I$ be sets. Let $P=\{B_i|i\in I\}$ be a collection of non-empty subsets of $A$ indexed by $I$.
            Then $P$ is called a \textbf{partition} of $A$ if the following properties hold:
            \begin{enumerate}[(i)]
                \item $A=\bigcup_{i\in I}B_i$ ($A$ is the union of all the sets in $P$).
                \item $B_i\cap B_j=\emptyset$ for $i\neq j$ (any two distinct members of $P$ are disjoint).
            \end{enumerate}
        \end{definition}
        
        \paragraph{}
         % TODO: include partition diagram
        It is clear to see that for \textit{any} given equivalence relation $\sim$, the equivalence classes divide the set $A$ into a partition.
        In fact, it works in reverse as well.
        \begin{theorem}
            Let $A,I$ be sets and let $P=\{B_i|i\in I\}$ be a partiton of $A$.
            Define a binary relation $\sim$ by $a\sim b$ if $a$ and $b$ lie in the same part $B_i$ of the partition.
            Then $\sim$ is an equivalence relation and $B_i$ are exactly the equivalence classes of $\sim$.
        \end{theorem}

    \section{Modular Arithmetic}
        \paragraph{}
        Let $n$ be an integer greater than 1.
        Then we say that two integers $a$ and $b$ are \textbf{congruent modulo $n$} if $n|(b-a)$.
        We write $a\equiv b\mod n$.
        For example, when $n=2$, $a\equiv b\mod 2$ when $b-a$ is even.
        Note $a\equiv 0\mod n$ if and only if $n|a$.
        If two positive integers are congruent mod 10, then they have the same last digit (in base 10).
        \begin{theorem}
            Congruency modulo $n$ is an equivalence relation.
        \end{theorem}
        \begin{proof}
            Let $a,b,c\in\ZZ$.
            \begin{enumerate}[(i)]
                \item $a-a=0=n\cdot 0$, so $n|(a-a)$.
                \item Suppose $a\equiv b\mod n$, so $b-a=nq$ for some integer $q$. Then $a-b=-nq=n(-q)$, so $n|(a-b)$ and $b\equiv a\mod n$.
                \item Finally, suppose $a\equiv b\mod n$ and $b\equiv c\mod n$, so $b-a=nq$ and $c-b=np$ for two integers $q$ and $p$. Then $c-a=c-b+b-a=np+nq=n(p+q)$, so $n|(c-a)$ and $a\equiv c\mod n$.
            \end{enumerate}
        \end{proof}

        \paragraph{}
        There are a couple of nice algebraic properties of congruence modulo $n$.
        These are properties which other equivalence relations might not respect.
        % TODO: elaborate more on this
        \begin{theorem}\label{congruence-properties}
            Let $a,b,c,d\in\ZZ$ with $a\equiv b\mod n$ and $c\equiv d\mod n$.
            \begin{enumerate}[(i)]
                \item $a+c\equiv b+d\mod n$.
                \item $ac\equiv bd\mod n$.
                \item $a+c\equiv b+c\mod n$.
                \item $ac\equiv bc\mod n$.
            \end{enumerate}
        \end{theorem}
        \begin{proof}
            First note that $b-a=nq$ and $d-c=np$ for two integers $q$ and $p$.
            \begin{enumerate}[(i)]
                \item $(b+d)-(a+c)=b-a+d-c=nq+np=n(q+p)$, so $n|((b+d)-(a+c))$.
                \item $bd-ac=bd-bc+bc-ac=b(d-c)+c(b-a)=bnp+cnq=n(bp+cq)$, so $n|(bd-ac)$.
            \end{enumerate}
            (iii) and (iv) follow from the (i) and (ii) by setting $d=c$.
        \end{proof}

    \section{Quotient Structures}
        \paragraph{}
        A powerful technique in abstract algebra is to use what we have been studying in this chapter to create smaller algebraic structures from larger ones while preserving some of the original structure.
        We start with some set $X$ which is endowed with some mathematical structure (for example, $X$ could be a ring, vector space, group, etc.).
        We then consider an equivalence relation on $X$ which preserves this structure in the same way as in the above theorem (it respects the operations).
        Then we construct the same structure on the set of equivalence classes. This is called the \textbf{quotient structure}.

        \paragraph{}
        For example, we have the integers $\ZZ$ which form a commutative ring under normal addition and multiplication.
        Congruence modulo $n$ is an equivalence relation which we have shown respects the operations.
        Can we define a ring structure on the set of equivalence classes of congruence modulo $n$?
        First we need to know what the equivalence classes actually are.
        \begin{theorem}
            Congruence modulo $n$ partitions $\ZZ$ into precisely $n$ equivalence classes which are given by
            \begin{equation}
                [r]=\{kn+r|k\in\ZZ\}\quad\text{for }r=0,1,\dots,n-1.
            \end{equation}
            We call these sets \textbf{congruence classes mod $n$}.
        \end{theorem}
        \begin{proof}
            Let $0\leq r\leq n-1$, $a\in\ZZ$. Then
            \begin{align}
                a\in[r]&\iff r\equiv a\mod n\\
                &\iff n|(a-r)\\
                &\iff a-r=nk\quad\text{for some integer }k\\
                &\iff a=nk+r,
            \end{align}
            hence $[r]=\{nk+r|k\in\ZZ\}$.
            Now we need to show that \textit{all} integers belong to one of these equivalence classes.

            \paragraph{}
            Let $a\in\ZZ$. Divide $a$ by $n$ to get a quotient and remainder $a=nq+r$ for some $q,r\in\ZZ$ (remember that $0\leq r\leq n-1$).
            Hence $a\in[r]$.
            So every integer belongs to one of the equivalence classes $[0],[1],\dots,[n-1]$, so there are \textit{at most} $n$ equivalence classes.
            In order to show that there are \textit{exactly} $n$, we need to show that they are distinct.

            \paragraph{}
            Let $0\leq r,s\leq n-1$ and suppose $[r]=[s]$.
            Then $r\equiv s\mod n$, so $n|(s-r)$.
            But $-n<s-r<n$, so $s-r$ must be 0 and $s=r$.
            This shows that the equivalence classes are distinct and there are therefore exactly $n$ of them.
        \end{proof}

        \paragraph{}
        Now that we have our equivalence classes we can work on defining addition and multiplication for the congruence classes.
        The easiest way to do this is to define them in terms of the representative element of the class, for example
        \begin{align}
            [a]+[b]&=[a+b]\\
            [a]\cdot[b]&=[ab].
        \end{align}
        However, these may not be well defined because the result of the operations depend on what the representative element of each class is.
        For example, in the case of $n=3$, we want to show that $[1]+[2]$ and $[7]+[5]$ result in the same equivalence class.
        \begin{lemma}
            The operations defined above are well defined.
        \end{lemma}
        \begin{proof}
            Let $a,c$ be two representatives of the same congruence class (so $[a]=[c]$ and $a\equiv c\mod n$).
            Let $b,d$ be two representatives of another congruence class.
            Then by theorem \ref{congruence-properties}:
            \begin{align}
                a+b&\equiv c+d\mod n\\
                ab&\equiv cd\mod n.
            \end{align}
            So $[a+b]=[c+d]$ and $[ab]=[cd]$ and thus the operations are well defined.
        \end{proof}

        \paragraph{}
        We can now show that the set of congruence classes with these operations also forms a ring structure which is the same as the original set.
        \begin{theorem}
            Let $n>1\in\ZZ$ The set $\ZZ/n\ZZ=\{[0],[1],\dots,[n-1]\}$ of congruence classes is a commutative ring under the operations of addition and multiplication defined above.
        \end{theorem}
        \begin{proof}
            We go step-by-step through the ring axioms.
            They are very simple to verify as they all follow from the fact that $\ZZ$ is itself a commutative ring.
            \begin{enumerate}[(i)]
                \item $([a]+[b])+[c]=[a+b]+[c]=[(a+b)+c]=[a+(b+c)]=[a]+[b+c]=[a]+([b]+[c])\quad\forall a,b,c\in\ZZ$.
                \item $[a]+[0]=[a+0]=[a]=[0+a]=[0]+[a]\quad\forall a\in\ZZ$.
                \item $[a]+[-a]=[a+(-a)]=[0]\quad\forall a\in\ZZ$.
                \item $[a]+[b]=[a+b]=[b+a]=[b]+[a]\quad\forall a,b\in\ZZ$.
                \item $([a]\cdot[b])\cdot[c]=[ab]\cdot[c]=[(ab)c]=[a(bc)]=[a]\cdot[bc]=[a]\cdot([b]\cdot[c])\quad\forall a,b,c\in\ZZ$.
                \item $[a]\cdot[1]=[a\cdot 1]=[a]=[1\cdot a]=[1]\cdot[a]\quad\forall a\in\ZZ$.
                \item $[a]\cdot[b]=[ab]=[ba]=[b]\cdot[a]\quad\forall a,b\in\ZZ$.
                \item $[a]\cdot([b]+[c])=[a]\cdot[b+c]=[a(b+c)]=[ab+ac]=[ab]+[ac]=[a]\cdot[b]+[a]\cdot[c]\quad\forall a,b,c\in\ZZ$ (and similarly for the right-distributivity).
            \end{enumerate}
            % TODO: include proper labels
        \end{proof}

        \paragraph{}
        Notice that what we have done is quite amazing.
        We have taken a commutative ring $\ZZ$ with infinitely many elements and ``reduced'' it down to a commutative ring $\ZZ/n\ZZ$ with only $n$ elements!
        When we are working with this ring, we sometimes omit the brackets for the congruence classes and simply write them as numbers.
        For example, we write the elements of $\ZZ/3\ZZ$ as $\{0,1,2\}$ and $\ZZ/6\ZZ$ and $\{0,1,2,3,4,5\}$.
        % TODO: include clock diagrams
         
    \section{Multiplication Tables}
        \paragraph{}
        One way to examine the structure of these finite rings is by looking at the addition and multiplication tables.
        % TODO: insert multiplication tables of Z/4Z and Z/5Z
        Note that in $\ZZ/4\ZZ$, some elements don't have multiplicative inverses.
        For example, the equation $2a\equiv 1\mod4$ has no solutions in $\ZZ/4\ZZ$, so 2 has no multiplicative inverse.
        However, in $\ZZ/5\ZZ$, all elements have multiplicative inverses.
        \begin{equation}
            1^{-1}=1,\quad 2^{-1}=3,\quad 3^{-1}=2,\quad 4^{-1}=4.
        \end{equation}
        Hence $\ZZ/5\ZZ$ is a field.
        It turns out that whenever $n$ is a prime number, $\ZZ/n\ZZ$ is a field.
        \begin{theorem}
            Let $p$ be a prime number. Then $\ZZ/p\ZZ$ is a field.
        \end{theorem}
        \begin{proof}
            % TODO: reference theorem
            We know that $\ZZ/p\ZZ$ is a commutative ring, so all that remains it to prove the final two field axioms.
            \begin{enumerate}[(i)]
                \item $1\neq0\mod p$, so 1 is a valid multiplicative identity.
                \item Let $a\in\ZZ$ such that $a\slashed\equiv0\mod p$. Since $p$ is prime, $\gcd(a,p)=1$.
                By Bezout's identity, there exists two integers $u,v$ such that $ua+vp=1$.
                Reducing mod $p$, we get $ua\equiv 1\mod p$, hence $[u]$ is a multiplicative inverse of $[a]$ in $\ZZ/p\ZZ$. 
            \end{enumerate}
            % TODO: include proper labels 
        \end{proof}

        \paragraph{}
        Let's do an example of solving some modular arithmetic equations.
        \begin{example}
            Determine all solutions (if any) in the ring $\ZZ/49\ZZ$ of the following equations.
            \begin{enumerate}[(i)]
                \item $4x\equiv 9\mod 49$.
                Observe that $\gcd(4,49)=1$. Using the extended Euclidean algorithm, we can see that $37(4)\equiv 1\mod49$.
                % TODO: add calculation of this
                Hence 37 is a multiplicative inverse for 4 in $\ZZ/49\ZZ$, so if we multiply the equation on the left by 37 we will cancel out the 4.
                \begin{align}
                    37\cdot 4x&\equiv 37\cdot 9\mod 49\\
                    x&\equiv 39\mod 49.
                \end{align}
                \item $7x\equiv 0\mod 49$.
                Observe that $7x\equiv 0\mod 49\iff 49|7x\iff 7x=49q$ for some integer $q$.
                Hence the solutions in $\ZZ/49\ZZ$ are 0, 7, 14, 21, 28, 35, and 42.
                % TODO: how can we multiply on the left by the inverse of 7?
                \item $7x\equiv 9\mod 49$.
                Suppose $x$ is a solution of the equation.
                Then multiplying on the left by 7 we get
                \begin{align}
                    7\cdot 7x&\equiv 49x\equiv 9\mod 49\\
                    &\implies 0\equiv 14\mod 49.
                \end{align}
                This last implication is impossible, so the equation has no solutions in $\ZZ/49\ZZ$.
            \end{enumerate}
        \end{example}

\end{document}
