\documentclass[../newtonian_mechanics.tex]{subfiles}

\begin{document}

    \section{Simple Harmonic Motion}
        \paragraph{}
        Simple harmonic motion is defined as the motion determined by a force which is directly proportional to the displacement from some equilibrium position.
        Consider a mass on a horizontal spring. This is a system which we have seen before, but we have not studied the full motion of the mass in detail.
        The only force acting on the mass is the spring restoring force
        % TODO: include diagram
        \begin{equation}
            F = -kx = ma,
        \end{equation}
        so by Newton II, we can write
        \begin{equation}\label{eq-SHM-spring}
            a = \frac{\mathrm{d}^2x}{\mathrm{d}t^2}=-\frac{k}{m}x.
        \end{equation}
        The general solution to this differential equation is
        \begin{equation}
            x(t)=A\cos(\omega t) + B\sin(\omega t),
        \end{equation}
        where
        \begin{equation}
            \omega=\sqrt{\frac{k}{m}}.
        \end{equation}
        This can be rewritten as
        \begin{equation}
            x(t) = A\cos(\omega t + \phi),
        \end{equation}
        where $\phi$ is the initial phase at $t=0$.
        Notice that $A$ is the maximum amplitude of the mass. 
        Working out the derivatives of $x(t)$, We see that
        \begin{align}
            &\frac{\mathrm{d}x(t)}{\mathrm{d}t} = v(t) = -A\omega\sin(\omega t + \phi)\\
            &\frac{\mathrm{d}^2x(t)}{\mathrm{d}t^2} = a(t) = -A\omega^2\cos(\omega t + \phi) = -\omega^2x(t),
        \end{align}
        and so we have recovered the equation of motion.
        % TODO: include graphs of position, velocity, acceleration and discussion of phase difference between them

        \paragraph{}
        % TODO: link to what has been covered earlier, should have been able to work out the period of oscillation by energy conservation previously
        Let's work out the period of oscillation.
        It should be equal to the time taken for the phase to change by $2\pi$.
        Hence
        \begin{align}
            \omega t_0 + \phi &= \omega(t_0 + T) + \phi\\
            \implies \omega T &= 2\pi\\
            T &= \frac{2\pi}{\omega}.
        \end{align}
        Since $f=1/T$, the frequency of the oscillation is related to $\omega$, the \textbf{angular frequency}, by
        \begin{equation}
            \omega = 2\pi f.
        \end{equation}
        % TODO: note that this is the same definition as for angular velocity.

        \paragraph{}
        % TODO: move this to a new section? (energy in SHM?)
        % TODO: make references for previous equations
        Now consider the energy of the mass on the spring. From before, we know this is given by
        \begin{align}
            E &= K + U_s\\
            &=\frac{1}{2}mv^2=\frac{1}{2}kx^2.
        \end{align}
        Since we now know that $x(t)=A\cos(\omega t+\phi)$ and $v(t)=-A\omega\sin(\omega t+\phi)$, we can show that
        \begin{align}
            E &=\frac{1}{2}m(-A\omega\sin(\omega t+\phi))^2+\frac{1}{2}k(A\cos(\omega t+\phi))^2\\
            &=\frac{1}{2}mA^2\omega^2\sin^2(\omega t+\phi)+\frac{1}{2}kA^2\cos^2(\omega t+\phi)\\
            &=\frac{1}{2}mA^2\left(\frac{k}{m}\right)\sin^2(\omega t+\phi)+\frac{1}{2}kA^2\cos^2(\omega t+\phi)\\
            &=\frac{1}{2}kA^2\left(\sin^2(\omega t+\phi)+\cos^2(\omega t+\phi)\right)\\
            &=\frac{1}{2}kA^2.
        \end{align}
        Where in the last line we have used the identity $\sin^2(\theta)+\cos^2(\theta)=1$.
        Note that this is independent of time, so the total energy is conserved as we found before.

        \paragraph{}
        Let's look at a pendulum on a string of length $L$ now.
        We are assuming that the string is massless and ignoring the effects of air resistance.
        % TODO: include diagram and free=body diagram
        % TODO: redo this with torques
        Given that the arc length of the pendulum is related to the angular displacement by $s=L\theta$ and that the restoring force is $F=-mg\sin(\theta)$, we have the equation of motion:
        \begin{equation}
            F=m\frac{\mathrm{d}^2s}{\mathrm{d}t^2}=-mg\sin(\theta).
        \end{equation}
        This is a nonlinear differential equation. We will simplify this by assuming that the angle $\theta$ is small so $\sin(\theta)\approx\theta$ (small-angle approximation).
        Then the equation of motion becomes
        \begin{equation}
            \frac{\mathrm{d}^2s}{\mathrm{d}t^2}=-g\theta=-\frac{g}{L}s.
        \end{equation}
        Note that this has the same form as equation \ref{eq-SHM-spring}, so the trajectory will have the same form!
        \begin{equation}
            s(t)=A\cos(\omega t+\phi),
        \end{equation}
        where $\omega$ in this case is given by
        \begin{equation}
            \omega = \sqrt{\frac{g}{L}}.
        \end{equation}

        \paragraph{}
        As an aside, think back to uniform circular motion.
        % TODO: include a diagram
        An object moving with UCM has a trajectory of the form
        \begin{equation}
            \theta(t)=\omega t+\phi,
        \end{equation}
        where $\omega$ is the angular velocity and $\phi$ is the angular displacement at $t=0$.
        If we describe this motion in cartesian coordinates, the horizontal motion takes the form
        \begin{equation}
            x(t) = r\cos(\omega t+\phi),
        \end{equation}
        which is exactly the same form as an object moving under SHM!

\end{document}
