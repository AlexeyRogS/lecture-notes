\documentclass[../newtonian_mechanics.tex]{subfiles}

\begin{document}

    \section{The Wave Equation}
        \paragraph{}
        A wave is a periodic variation or disturbance which travels at a well defined speed through space.
        How do we describe waves mathematically?
        Suppose $f(\chi)$ is some periodic function which takes a phase $\chi$ measured in radians (fractions of $2\pi$).
        Then we can describe the variation in space at a specific point in time as a snapshot:
        % TODO: rewrite this using cos instead of general function?
        \begin{equation}
            y(x) = A f\left(\frac{2\pi}{\lambda}x+\delta\right),
        \end{equation}
        where $\lambda$ is the \textbf{wavelength} of the wave (the spatial period).
        % TODO: include figures
        We can also describe the variation in amplitude at a single point in space over time:
        \begin{equation}
            y(t)=A f\left(\frac{2\pi}{T}t+\theta\right),
        \end{equation}
        where $T$ is the temporal period.

        \paragraph{}
        To put these pictures together, we can consider the snapshot picture with a shift $x-vt$ where $v$ is the speed of the wave.
        Then we have
        \begin{align}
            y(x, t) &= A f\left(\frac{2\pi}{\lambda}(x-vt) + \phi\right)\\
            &= A f\left(\frac{2\pi}{\lambda}x - \frac{2\pi v}{\lambda}t + \phi\right)\\
            &= A f\left(\frac{2\pi}{\lambda}x - \frac{2\pi}{T}t + \phi\right)\\
            &= A f(kx - \omega t + \phi).
        \end{align} 
        Where we have defined the \textbf{wavenumber} (spatial frequency measured in radians/m) $k=2\pi/\lambda$ and recalled $v=f\lambda$, $f=1/T$ and $\omega=2\pi f$.
        \begin{example}
            Suppose we have a sinusoidal wave given by $y(x, t)=A\cos(kx-\omega t+\phi)$.
            What is the particle velocity at fixed position $x$?

            To solve this, we take the \textit{partial derivative} with respect to time (to keep $x$ constant)
            \begin{equation}
                \frac{\partial y(x, t)}{\partial t}=A\omega\sin(kx-\omega t+\phi).
            \end{equation}
            Note that this is different to the propagation speed of the wave itself, which is constant.
        \end{example}

        \paragraph{}
        For simplicity, let's consider a general right-travelling wave $f(x-vt)$. We can make a substitution $u=x-vt$.
        Then we get
        \begin{align}
            \pder{f}{x}&=\pder{f}{u}\pder{u}{x}\\
            &=\pder{f}{u}\\
            \implies\pdertwo{f}{x}&=\pdertwo{f}{u},\\
            \pder{f}{t}&=\pder{f}{u}\pder{u}{t}\\
            &=-v\pder{f}{u}\\
            \implies\pdertwo{f}{t}&=v^2\pdertwo{f}{u}.
        \end{align}
        If we do this same calculation with a left-travelling wave $f(x+vt)$, we get the same relation.
        Thus, by construction, the general solution to the differential equation
        \begin{equation}
            \pdertwo{f}{t}=v^2\pdertwo{f}{x},
        \end{equation}
        is $f(x,t)=f_l(x+vt)+f_r(x-vt)$.
        This linear differential equation is known as the \textbf{wave equation}.
        All functions which satisfy our definition of a wave solve this equation.
        % TODO: there are also non-periodic solutions

    \section{Superposition of Waves}
        \paragraph{}
        Since the wave equation is linear, the solutions follow the principle of \textbf{linear superposition}.
        This means that when multiple waves come together, the amplitude at every point in space and time is determined by the sum of all the waves at that point.
        \begin{example}
            Consider two sinusoidal waves with the travelling with the same frequency and direction.
            Then the superposition is given by
            \begin{align}
                y(x, t) &= A\cos(kx-\omega t) + A\cos(kx-\omega t)\\
                &= 2A\cos(kx-\omega t).
            \end{align}
            So, the resultant wave has the same frequency and direction but double the amplitude.
            % TODO: include diagrams for these examples

            \paragraph{}
            Now consider what happens if one of the waves has a phase shift of $\pi$ radians.
            The resultant wave is
            \begin{align}
                y(x, t) &= A\cos(kx-\omega t) + A\cos(kx-\omega t+\pi)\\
                &= A\cos(kx-\omega t) - A\cos(kx-\omega t)\\
                &= 0.
            \end{align}
            The two waves cancel each other out completely.

            \paragraph{}
            In the general case with a phase shift $\Omega$, we get
            \begin{align}
                y(x, t) &=A\cos(kx-\omega t) + A\cos(kx-\omega t+\Omega)\\
                &= 2A\cos\left(kx-\omega t+\frac{\Omega}{2}\right)\cos\left(-\frac{\Omega}{2}\right)\\
                &= \underbrace{2A\cos\left(\frac{\Omega}{2}\right)}_{\text{Amplitude }\leq 2A}\underbrace{\cos\left(kx-\omega t+\frac{\Omega}{2}\right)}_{\text{Time-dependent part}}.
            \end{align}
            Note that we have used the identity $\cos(\alpha)+\cos(\beta)=2\cos\left(\frac{\alpha+\beta}{2}\right)\cos\left(\frac{\alpha-\beta}{2}\right)$.
        \end{example}

        \paragraph{}
        Now consider what happens if the waves still have the same frequency but are moving in opposite directions.
        In this case the superposition is
        % TODO: put this in an example?
        % TOOD: include diagrams for standing waves
        \begin{align}
            y(x, t) &= A\cos(kx -\omega t) + A\cos(kx + \omega t)\\
            &= 2A\cos\left(\frac{2kx}{2}\right)\cos\left(-\frac{2\omega t}{2}\right)\\
            &= \underbrace{2A\cos(kx)}_{A(x)}\cos(\omega t).
        \end{align}
        So we have a spatially varying amplitude $A(x)$ multiplied by a time-dependent variation.
        This is known as a \textbf{standing wave}.
        \begin{example}
            In the case where one of the waves has a phase shift $\Omega$. The relation above becomes
            \begin{align}
                y(x, t) &=A\cos(kx-\omega t) + A\cos(kx+\omega t+\Omega)\\
                &= 2A\cos\left(\frac{2kx+\Omega}{2}\right)\cos\left(-\frac{w\omega t+\Omega}{2}\right)\\
                &= 2A\cos\left(kx+\frac{\Omega}{2}\right)\cos\left(\omega t+\frac{\Omega}{2}\right).
            \end{align}
        \end{example}
        % TODO: include discussion of phasors for modelling superposition and interference

        \paragraph{}
        The standing waves allowed in a one-dimensional region of length $L$ are given by
        \begin{equation}
            \lambda_\frac{2L}{p}, \quad f_p=p\frac{v}{2L}=pf_1,
        \end{equation}
        where $p\in\mathbb{Z}$.

        \paragraph{}
        Note that sometimes it is more convenient to express waves in terms of complex exponential functions according to Euler's formula:
        \begin{equation}
            e^{i\theta}=\cos\theta+i\sin\theta,
        \end{equation}
        so a general sinusoidal wave like above would be written as
        \begin{equation}
            y(x,t)=Ae^{i(kx-\omega t+\phi)}.
        \end{equation}
        We recover the trigonometric form (cosine in this case) by taking the real part of this function.
        \begin{example}
            Given a periodic wave, what is the phase difference between two points on the wave separated by a distance $\Delta x$?
            \begin{equation}
                \Delta\phi=2\pi\frac{\Delta x}{\lambda}=k\Delta k.
            \end{equation}
            What is the phase difference between a single point over a interval of time $\Delta t$?
            \begin{equation}
                \Delta\phi=2\pi\frac{\Delta t}{T}=\omega\Delta t.
            \end{equation}
        \end{example}

    \section{Phase Velocity \& Group Velocity}
        \paragraph{}
        As we have seen, the speed you need to keep up with a point of constant phase along the wave is given by
        \begin{equation}
            v_\phi=f\lambda=\frac{\omega}{k}.
        \end{equation}
        This is known as the \textbf{phase velocity}.
        The dependence of $\omega$ on $k$ (or vice-versa) is called the \textbf{dispersion relation}.
        If the relationship is linear, i.e. if $v_\phi$ is constant, the wave is said to be dispersionless.
        Otherwise, the wave will undergo dispersion as different frequencies will travel at difference speeds.

        \paragraph{}
        The \textbf{group velocity} is defined as
        \begin{equation}
            v_g=\der{\omega}{k}.
        \end{equation}
        So if a wave is dispersionless, the phase velocity and group velocity will be the same.
        In the case where $v_g\neq v_\phi$, the group velocity is the speed that the wave envelope propagates.

    \section{Transverse Waves on a String}
        \paragraph{}
        Consider an infinite string under constant tension $T$.
        We will now show that the equation of motion of the string is the wave equation and derive the wave speed.
        Consider a short section of the string of length $\Delta x$.
        % TODO: include diagram of this
        We are assuming that the string has linear density $\mu$, zero stiffness, and we are ignoring the effects of gravity.
        Then assuming that there are only small displacements on the string, then $\pder{y}{X}$ is small, so the angles $\theta_1$ and $\theta_2$ are also small.
        Hence we use the small angle approximation and say that $\cos\theta_1\approx\cos\theta_2\approx 1$.
        \begin{align}
            \sum F_x&=-\abs{\vect{T}_1}\cos\theta_1+\abs{\vect{T}_2}\cos\theta_2=0\\
            \implies\abs{T_{1,x}}&\approx\abs{T_{2,x}}\approx T.
        \end{align}
        From these we get that $T_{1,x}\approx -T$ and $T_{2,x}\approx T$.
        
        \paragraph{}
        Now, using some trigonometry, notice that
        \begin{align}
            \left.\pder{y}{x}\right|_x&=\frac{T_{1,y}}{T_{1,x}}\approx -\frac{T_{1,y}}{T}\\
            \left.\pder{y}{x}\right|_{x+\Delta x}&=\frac{T_{2,y}}{T_{2,y}}\approx\frac{T_{2,y}}{T}.
        \end{align}
        Thus the net force in the $y$-direction is given by
        \begin{align}
            F_y&=T_{1,y}+T_{2,y}\\
            &=T\left(-\left.\pder{y}{x}\right|_x+\left.\pder{y}{x}\right|_{x+\Delta x}\right).
        \end{align}
        % TODO: replace this part with the better (in my opinion) derivation from Griffiths
        Using Newton's second law, we get
        \begin{align}
            F_y&=ma_y\\
            &=\mu\Delta xa_y\\
            &=\mu\Delta x\left.\pdertwo{y}{t}\right|_{x+\frac{\Delta x}{2}}=\left(\left.\pder{y}{x}\right|_{x+\Delta x}-\left.\pder{y}{x}\right|_x\right)T.
        \end{align}
        Finally we divide by $\Delta x$ on both sides and take the limit as $\Delta x\to0$ to get
        \begin{align}
            \lim_{\Delta x\to0}\left(\mu\left.\pdertwo{y}{t}\right|_{x+\frac{\Delta x}{2}}\right)&=\lim_{\Delta x\to0}\left[\frac{\left(\left.\pder{y}{x}\right|_{x+\Delta x}-\left.\pder{y}{x}\right|_x\right)T}{\Delta x}\right]\\
            \mu\pdertwo{y}{t}&=T\pdertwo{y}{x}\\
            \implies\pdertwo{y}{t}&=\frac{T}{\mu}\pdertwo{y}{x}.
        \end{align}
        This is the wave equations and we can see that for waves on a string, $v=\sqrt{\frac{T}{\mu}}$.

        \paragraph{}
        What is the mechanical energy stored in a wave on a string?
        It will have two contributions, potential energy which depends on the displacement of every point and kinetic energy which depends on the velocity of every point.
        Consider a segment of the string of length $\mathrm{d}x$, mass $\mathrm{d}m=\mu\mathrm{d}x$.
        The infinitesimal contribution to the kinetic energy of the wave is given by
        \begin{equation}
            \mathrm{d}K=\frac{1}{2}\mathrm{d}mv_y^2=\frac{1}{2}\mu\mathrm{d}x\left(\pder{y}{t}\right)^2.
        \end{equation}
        To get a value for this, we integrate it over some length $L$.
        \begin{equation}
            K=\frac{1}{2}\mu\int_{0}^{L}\left(\pder{y}{t}\right)^2\mathrm{d}x.
        \end{equation}
        The potential energy is due to the stretching of the string.
        A segment of length $\mathrm{d}x$ stretches to a length $\mathrm{d}s$, and we can calculate the relationship between the two as follows:
        \begin{align}
            \mathrm{d}s&=\sqrt{\mathrm{d}x^2+\mathrm{d}y^2}\\
            &=\sqrt{\mathrm{d}x^2+\mathrm{d}x^2\left(\pder{y}{x}\right)^2}\\
            &=\mathrm{d}x\sqrt{1+\left(\pder{y}{x}\right)^2}\\
            &\approx\mathrm{d}x\left(1+\frac{1}{2}\left(\pder{y}{x}\right)^2\right),
        \end{align}
        where in the last line we have used the Taylor expansion $\sqrt(1+u^2)\approx 1+\frac{1}{2}u^2$ when $u$ is small.
        This means we can calculate the potential energy as
        \begin{align}
            \mathrm{d}U&=T(\mathrm{d}s-\mathrm{d}x)\\
            &\approx\frac{1}{2}T\left(\pder{y}{x}\right)^2\mathrm{d}x\\
            \implies U&=\frac{1}{2}T\int_{0}^{L}\left(\pder{y}{x}\right)^2\mathrm{d}x.
        \end{align}
        \begin{example}
            Consider a sinusoidal wave $y=A\cos(kx-\omega t)$.
            What is the energy per unit wavelength?
            The partial derivatives are given by
            \begin{align}
                \pder{y}{t}&=A\omega\sin(kx-\omega t)\\
                \pder{y}{x}&=-Ak\sin(kx-\omega t),
            \end{align}
            so the infinitesimal contribution to the total energy is
            \begin{align}
                \mathrm{d}E&=\mathrm{d}K+\mathrm{d}U\\
                &=\frac{1}{2}\left[\mu\left(\pder{y}{t}\right)^2+T\left(\pder{y}{x}\right)^2\right]\mathrm{d}x\\
                &=\frac{1}{2}A^2\sin^2(kx-\omega t)(\mu\omega^2+Tk^2)\mathrm{d}x.
            \end{align}
            Note that $v=\frac{\omega}{k}=\sqrt{T}{\mu}$, so $Tk^2=\mu\omega^2$.
            Hence for a sinudoidal wave, the kinetic and potential energies are the same.
            The energy per unit wavelength is then
            \begin{align}
                E_\lambda&=\mu A^2\omega^2\int_{0}^{\lambda}\sin^2(kx)\mathrm{d}x\\
                &=\frac{1}{2}\lambda\mu A^2\omega^2.
            \end{align}
            Note that we choose to write the energy in terms of $\mu$ rather than $T$ because linear density is an easily measurable property whereas the tension is not.
            One important thing to mention is that the dependence on $A^2$ is actually general to all forms of waves, not just sinusoidal.
            We can calculate the power transmitted through a single point by the wave as
            \begin{align}
                P=E_\lambda f&=\frac{1}{2}\lambda f\mu A^2\omega^2\\
                &=\frac{1}{2}v\mu A^2\omega^2\\
                &=\frac{1}{2}\sqrt{\mu T}A^2\omega^2.
            \end{align}
        \end{example}
        
\end{document}
