\documentclass[../newtonian_mechanics.tex]{subfiles}

\begin{document}

    \section{The Wave Equation}
        \paragraph{}
        A wave is a periodic variation in space and time.
        How do we describe waves mathematically?
        Suppose $f(\chi)$ is some periodic function which takes a phase $\chi$ measured in radians (fractions of $2\pi$).
        Then we can describe the variation in space at a specific point in time as a snapshot:
        % TODO: rewrite this using cos instead of general function?
        \begin{equation}
            y(x) = A f\left(\frac{2\pi}{\lambda}x+\delta\right),
        \end{equation}
        where $\lambda$ is the \textbf{wavelength} of the wave (the spatial period).
        % TODO: include figures
        We can also describe the variation in amplitude at a single point in space over time:
        \begin{equation}
            y(t)=A f\left(\frac{2\pi}{T}t+\theta\right),
        \end{equation}
        where $T$ is the temporal period.

        \paragraph{}
        To put these pictures together, we can consider the snapshot picture with a shift $x-vt$ where $v$ is the speed of the wave.
        Then we have
        \begin{align}
            y(x, t) &= A f\left(\frac{2\pi}{\lambda}(x-vt) + \phi\right)\\
            &= A f\left(\frac{2\pi}{\lambda}x - \frac{2\pi v}{\lambda}t + \phi\right)\\
            &= A f\left(\frac{2\pi}{\lambda}x - \frac{2\pi}{T}t + \phi\right)\\
            &= A f(kx - \omega t + \phi).
        \end{align} 
        Where we have defined the \textbf{wavenumber} (spatial frequency measured in radians/m) $k=2\pi/\lambda$ and recalled $v=f\lambda$, $f=1/T$ and $\omega=2\pi f$.
        \begin{example}
            Suppose we have a sinusoidal wave given by $y(x, t)=A\cos(kx-\omega t+\phi)$.
            What is the particle velocity at fixed position $x$?

            To solve this, we take the \textit{partial derivative} with respect to time (to keep $x$ constant)
            \begin{equation}
                \frac{\partial y(x, t)}{\partial t}=A\omega\sin(kx-\omega t+\phi).
            \end{equation}
            Note that this is different to the propagation speed of the wave itself, which is constant.
        \end{example}

    \section{Superposition of Waves}
        \paragraph{}
        The waves that we are studying follow the principle of \textbf{linear superposition}.
        This means that when multiple waves come together, the amplitude at every point in space and time is determined by the sum of all the waves at that point.
        \begin{example}
            Consider two sinusoidal waves with the travelling with the same frequency and direction.
            Then the superposition is given by
            \begin{align}
                y(x, t) &= A\cos(kx-\omega t) + A\cos(kx-\omega t)\\
                &= 2A\cos(kx-\omega t).
            \end{align}
            So, the resultant wave has the same frequency and direction but double the amplitude.
            % TODO: include diagrams for these examples

            \paragraph{}
            Now consider what happens if one of the waves has a phase shift of $\pi$ radians.
            The resultant wave is
            \begin{align}
                y(x, t) &= A\cos(kx-\omega t) + A\cos(kx-\omega t+\pi)\\
                &= A\cos(kx-\omega t) - A\cos(kx-\omega t)\\
                &= 0.
            \end{align}
            The two waves cancel each other out completely.

            \paragraph{}
            In the general case with a phase shift $\Omega$, we get
            \begin{align}
                y(x, t) &=A\cos(kx-\omega t) + A\cos(kx-\omega t+\Omega)\\
                &= 2A\cos\left(kx-\omega t+\frac{\Omega}{2}\right)\cos\left(-\frac{\Omega}{2}\right)\\
                &= \underbrace{2A\cos\left(\frac{\Omega}{2}\right)}_{\text{Amplitude }\leq 2A}\underbrace{\cos\left(kx-\omega t+\frac{\Omega}{2}\right)}_{\text{Time-dependent part}}.
            \end{align}
            Note that we have used the identity $\cos(\alpha)+\cos(\beta)=2\cos\left(\frac{\alpha+\beta}{2}\right)\cos\left(\frac{\alpha-\beta}{2}\right)$.
        \end{example}

        \paragraph{}
        Now consider what happens if the waves still have the same frequency but are moving in opposite directions.
        In this case the superposition is
        % TODO: put this in an example?
        % TOOD: include diagrams for standing waves
        \begin{align}
            y(x, t) &= A\cos(kx -\omega t) + A\cos(kx + \omega t)\\
            &= 2A\cos\left(\frac{2kx}{2}\right)\cos\left(-\frac{2\omega t}{2}\right)\\
            &= \underbrace{2A\cos(kx)}_{A(x)}\cos(\omega t).
        \end{align}
        So we have a spatially varying amplitude $A(x)$ multiplied by a time-dependent variation.
        This is known as a \textbf{standing wave}.
        \begin{example}
            In the case where one of the waves has a phase shift $\Omega$. The relation above becomes
            \begin{align}
                y(x, t) &=A\cos(kx-\omega t) + A\cos(kx+\omega t+\Omega)\\
                &= 2A\cos\left(\frac{2kx+\Omega}{2}\right)\cos\left(-\frac{w\omega t+\Omega}{2}\right)\\
                &= 2A\cos\left(kx+\frac{\Omega}{2}\right)\cos\left(\omega t+\frac{\Omega}{2}\right).
            \end{align}
        \end{example}
        % TODO: include discussion of phasors for modelling superposition and interference

        \paragraph{}
        The standing waves allowed in a one-dimensional region of length $L$ are given by
        \begin{equation}
            \lambda_\frac{2L}{p}, \quad f_p=p\frac{v}{2L}=pf_1,
        \end{equation}
        where $p\in\mathbb{Z}$.
    
\end{document}
