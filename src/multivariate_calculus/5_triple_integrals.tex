\documentclass[../multivariate_calculus.tex]{subfiles}

\begin{document}

    \section{Integration over Regions Bounded by Planes}
        \paragraph{}
        Just as a double integral of a function over a region of 2D space represents the volume under a surface, the triple integral of a function over a region of 3D space represents the \textit{hypervolume} under a volume.
        To evaluate a triple integral, we extend the methods that we have developed for evaluating double integrals.
        Instead of integrating over an infinitesimal area element, we integrate over a \textbf{volume element}.
        In cartesian coordinates, this has the form $\dd{V}=\dd{x}\dd{y}\dd{z}$.

        \paragraph{}
        When deciding the limits of integration, we make the same considerations that we do for double integrals.
        The limits of the outer integral must be fixed, the limits of the middle integral can depend on the outer variable but \textit{not} on the inner variable.
        The limits of the inner variable can depend on both the middle and outer variables.
        \begin{example}
            Find the coordinates of the centre of the tetrahedron bounded by the $x$-$y$, $x$-$z$, $y$-$z$ planes and the plane $x+y+z=1$.
            % TODO: include diagram for this
            Just like in 2D, the centre of the region, which is simply the average value of the coordinates, is given by
            \begin{equation}
                \bar{x}=\frac{\iiint_V x\dd{V}}{\iiint_V \dd{V}},\dots
            \end{equation}
            We only need to find the average $x$ position in this case since the process will be symmetric for $y$ and $z$.
            Choosing $z$ as the outer variable, the limits are 0 and 1.
            Now consider a cross section of the region $V$ for a constant $z$.
            $x$ and $y$ are related by
            \begin{equation}
                x+y=1-z,
            \end{equation}
            so if we choose $y$ as the middle variable, then the limits are 0 and $1-z$.
            Then similarly, the limits for $x$ are 0 and $1-z-y$.
            Thus, we have
            \begin{align}
                \int_0^1\int_0^{1-z}\int_0^{1-z-y}x\dd{x}\dd{y}\dd{z}&=\int_0^1\int_0^{1-z}\frac{1}{2}(1-z-y)^2\dd{y}\dd{z}\\
                &=\int_0^1\left.-\frac{1}{6}(1-z-y)^3\right|_0^{1-z}\dd{z}\\
                &=\int_0^1\frac{1}{6}(1-z)^3\dd{z}=\left.-\frac{1}{24}(1-z)^4\right|_0^1=\frac{1}{24}.
            \end{align}
            The volume of the tetrahedral region is given by
            \begin{align}
                \int_0^1\int_0^{1-z}\int_0^{1-z-y}\dd{x}\dd{y}\dd{z}&=\int_0^1\int_0^{1-z}(1-z-y)\dd{y}\dd{z}\\
                &=\int_0^1\left.-\frac{1}{2}(1-z-y)^2\right|_0^{1-z}\dd{z}\\
                &=\int_0^1\frac{1}{2}(1-z)^2\dd{z}=\left.-\frac{1}{6}(1-z)^3\right|_0^1=\frac{1}{6}.
            \end{align}
            Hence
            \begin{equation}
                \bar{x}=\bar{y}=\bar{z}=\frac{1/24}{1/6}=\frac{1}{4},
            \end{equation}
            so the centre of the tetrahedron is the point $\left(\frac{1}{4},\frac{1}{4},\frac{1}{4}\right)$.
        \end{example}

    \section{Integrals over Volumes Bounded by Curved Surfaces}
        \paragraph{}
        Just like in 2D, when we have a complicated volume to integrate over making a change of variables can sometimes simplify the problem greatly.
        If a volume is symmetric about an axis, then we can write $\dd{V}=\dd{A}\dd{z}$, where $z$ points along the axis of symmetry and $\dd{A}$ is the differential cross-sectional area.
        \begin{example}
            Find the integral of $z$ over an inverted cone of height $h$ and width $2a$.
            % TODO: include diagram for this
            The cone has a straight edge. The distance between the $z$ axis and the surface of the cone is given by
            \begin{equation}
                R_\text{edge}(z)=a-\frac{a}{h}z=a\left(1-\frac{z}{h}\right).
            \end{equation}
            Hence if we take $z$ as the outer variable, then the integral can be written as
            \begin{equation}
                \iiint_V z\dd{V}=\int_0^h z\iint_{A(z)}\dd{A}(z)\dd{z},
            \end{equation}
            where the inner double integral is simply the area of the 2D region $A(z)$, which is $\pi R_\text{edge}(z)^2$.
            Hence
            \begin{align}
                \int_0^h z\iint_{A(z)}\dd{A}(z)\dd{z}&=\pi a^2\int_0^h z\left(1-\frac{z}{h}\right)^2\dd{z}\\
                &=\pi a^2h^2\int_0^1 u(1-u)^2\dd{u}\\
                &=\pi a^2h^2\left[\frac{1}{2}a^2-\frac{2}{3}u^3+\frac{1}{4}u^4\right]_0^1=\frac{\pi a^2h^2}{12}.
            \end{align}
        \end{example}
        This method applies in general if we have a function and a volume which are both symmetric about an axis.
        Such volume are bounded by surfaces called \textbf{surfaces of revolution}.
        In this case,
        \begin{equation}
            I=\pi\int_{z_1}^{z_2}R_\text{edge}(z)^2 f(z)\dd{z}.
        \end{equation}
        \begin{example}
            Find the integral of the decaying exponential $e^{-z}$ over the volume bounded by the paraboloid $az=x^2+y^2$ with $a>0$ and $0\leq z\leq b$.
            % TODO: include diagram for this
            Note that in plane polar coordinates we have $R^2=x^2+y^2$, so the radius of the volume is given by $R_\text{edge}(z)=\sqrt{az}$.
            Hence by using the general formula, we get
            \begin{align}
                \iiint_V e^{-z}\dd{z}&=\pi a\int_0^b ze^{-z}\dd{z}\\
                &=\pi a\left(\left.-ze^{-z}\right|_0^b+\int_0^b e^{-z}\dd{z}\right)\\
                &=\pi a\left[-be^{-b}-e^{-b}+1\right]=\pi a[1-(1+b)e^{-b}].
            \end{align}
        \end{example}
        
        \paragraph{}
        Common surfaces of revolution have radii given by
        \begin{align}
            \text{Cylinder: }&R=a\\
            \text{Cone: }&R=az\\
            \text{Paraboloid: }&R=a\sqrt{z}\\
            \text{Sphere: }&R=\sqrt{a^2-z^2}\\
            \text{Spheroid: }&R=\sqrt{a^2-b^2z^2}\\
            \text{Hyperboloid of one sheet: }&R=\sqrt{a^2+z^2}\\
            \text{Hyperboloid of two sheets: }&R=\sqrt{z^2-a^2}\quad\abs{z}\geq a.
        \end{align}

        \paragraph{}
        What about general functions over volumes bounded by surfaces of revolution?
        In this case we make a change of variables to cylindrical coordinates with the $z$ axis aligned along the axis of revolution.
        Choosing $z$ as the outer variable, the cross-section at fixed $z$ is a circle which allows us to use plane polar coordinates.
        \begin{example}
            Calculate the integral of $x^2$ over the inverted cone from before.
            Since a cone is bounded by a surface of revolution, we switch to cylindrical coordinates and choose $z$ as the outer variable.
            \begin{equation}
                I=\iiint_V x^2\dd{V}=\int_0^h\iint_{A(z)}x^2\dd{A}(z).
            \end{equation}
            For the inner double integral, the area element is $R\dd{r}\dd{\phi}$ and the limits are $0\leq R\leq a\left(1-\frac{z}{h}\right)$ and $0\leq\phi\leq2\pi$.
            Hence,
            \begin{align}
                I&=\int_0^h\int_0^{2\pi}\int_0^{a\left(1-\frac{z}{h}\right)}R^3\cos^2\phi\dd{r}\dd{\phi}\\
                &=\int_0^h\int_0^{2\pi}\frac{1}{4}\cos^2\phi a^4\left(1-\frac{z}{h}\right)^4\\
                &=\int_0^h\frac{\pi a^4}{4}\left(1-\frac{z}{h}\right)^4\dd{z}\\
                &=\frac{\pi a^4h}{4}\int_0^1(1-u)^4\dd{u}=\frac{\pi a^4h}{4}\left[-\frac{1}{5}(1-u)^5\right]_0^1=\frac{1}{20}\pi a^4h.
            \end{align}
        \end{example}
        \begin{example}
            Find the integral of $xyz$ over a volume bounded by the $x-y$, $x-z$, $y-z$ planes and a sphere of radius $a$.
            % TODO: include diagram for this
            The equation for the sphere is $x^2+y^2+z^2=a^2$. Note that $x^2+y^2=R^2$, hence $R_\text{edge}=\sqrt{a^2-z^2}$.
            Doing this integral in cylindrical coordinates, the limits for $\phi$ are 0 and $\frac{\pi}{2}$.
            \begin{align}
                \iiint_V xyz\dd{V}&=\int_0^a z\int_0^\frac{\pi}{2}\int_0^{\sqrt{a^2-z^2}}R^3\cos\phi\sin\phi\dd{r}\dd{\phi}\dd{z}\\
                &=\int_0^a z\left(\frac{1}{2}\right)\left(\frac{1}{4}\left(\sqrt{a^2-z^2}\right)^4\right)\dd{z}\\
                &=\frac{1}{8}\int_0^a z(a^2-z^2)^2\dd{z}\\
                &=\frac{1}{8}\left[\frac{1}{2}a^4z^2-\frac{1}{2}a^2z^4+\frac{1}{6}z^6\right]_0^a=\frac{1}{8}\left(\frac{1}{6}a^6\right)=\frac{1}{48}a^6.
            \end{align}
        \end{example}

    \section{Integration over Volumes Bounded by Portions of a Sphere}
        \paragraph{}
        If the region of integration is a sphere (or part of one), then changing to spherical coordinates will simplify the problem as the limits of integration will be constant.
        The volume element in spherical coordinates is given by $\dd{V}=\dd{r}\cdot r\dd{\theta}\cdot R\dd{\phi}=r^2\sin\theta\dd{r}\dd{\theta}\dd{\phi}$.
        Another case where spherical coordinates are useful is the volume of integration is the entirety of 3D space and the function to be integrated over depends only on the distance from a point (it has spherical symmetry).
        \begin{example}
            Find the integral of the function $e^{-\frac{r}{h}}$ over $\RR^3$.
            \begin{align}
                \iiint_{\RR^3}e^{-\frac{r}{h}}\dd{V}&=\int_0^{2\pi}\int_0^\pi\int_0^\infty e^{-\frac{r}{h}}r^2\sin\theta\dd{r}\dd{\theta}\dd{\phi}\\
                &=4\pi\int_0^\infty r^2e^{-\frac{r}{h}}\dd{r}\\
                &=4\pi\left[(-hr^2-2h^2r-2h^3)e^{-\frac{r}{h}}\right]_0^\infty=8\pi h^3.
            \end{align}
        \end{example}

\end{document}
