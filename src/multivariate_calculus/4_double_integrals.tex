\documentclass[../multivariate_calculus.tex]{subfiles}

\begin{document}

    \section{Multivariate Integration}
        \paragraph{}
        % TODO: add a section talking about the necessicity for Lebesgue integrals, etc.
        When integrating over more that one variable, we integrate over an \textbf{area element} $\dd{A}$ or \textbf{volume element} $\dd{V}$ as opposed to a \textbf{line element} $\dd{x}$ in 1D.
        Just as an integral of a 1D function represents the \textit{area} under a curve, an integral of a 2D function represents the \textit{volume} under a surface.
        % TODO: add diagram to illustrate this
        We can also find the mean value of a function over a region by
        \begin{equation}
            \bar{f}(x,y)=\frac{\iint_A f(x,y)\dd{A}}{\iint_A\dd{A}}.
        \end{equation}

    \section{Integration over Rectangular Regions}
        \paragraph{}
        A rectangular region is bounded by four sets of cartesian coordinates.
        % TODO: add diagram to illustrate this
        The area element is given by $\dd{A}=\dd{x}\dd{y}$.
        \begin{equation}
            I=\int_{y=b}^{y=d}\int_{x=a}^{x=c}f(x,y)\dd{x}\dd{y}.
        \end{equation}
        We can evaluate this integral by choosing one of the 1D integrals to evaluate first while keeping the other variable constant, just like we do when calculating partial derivatives.
        If we can find $F(x,y)$ satisfying $\pdv{F}{x}=f(x,y)$, then $g(y)=F(c,y)-F(a,y)=F(x,y)+c(y)$.
        \begin{align}
            &\int_a^c f(x,y)\dd{x}=g(y)\\
            &\implies I=\int_b^d g(y)\dd{y}.
        \end{align}
        This is possible due to \textbf{Fubini's theorem}
        % TODO: flesh out this discussion and include discussion of Fubini's theorem
        \begin{example}
            Calculate the integral of $f(x,y)=\sqrt{xy}$ over a rectangular region with opposite corners at $(1,4)$ and $(4,9)$.
            \begin{equation}
                I=\int_4^9\int_1^4\sqrt{xy}\dd{x}\dd{y}=\int_4^9\left.\frac{2}{3}\sqrt{x^3y}\right|_1^4\dd{y}=\int_4^9\frac{14}{3}\sqrt{y}\dd{y}=\left.\frac{37}{9}y^{\frac{3}{2}}\right|_4^9=\frac{532}{9}.
            \end{equation}
        \end{example}
        \begin{example}
            Calculate the integral of $f(x,y)=\frac{y}{(x+y^2)^2}$ over a rectangular region with opposite corners at $(1,4)$ and $(4,9)$.
            \begin{equation}
                I=\int_1^2\int_0^1\frac{y}{(x+y^2)^2}\dd{x}\dd{y}=\int_1^2\left.\frac{-y}{x+y^2}\right|_0^1\dd{y}=\left.\ln y-\frac{1}{2}\ln(1+y^2)\right|_1^2=\frac{1}{2}\ln\frac{8}{5}.
            \end{equation}
        \end{example}
        This second example is a non-seperable function, which influences which order it is easier to do the integrals in.

    \section{Regions Bounded by Lines and Curves}
        \paragraph{}
        To integrate over a non-rectangular region in cartesian coordinates, the order in which we integrate matters.
        This is because the limits of the inner integral will \textit{depend} on the value of the outer variable.
        For example, say we want to calculate the area of a triangular region.
        % TODO: add diagram to illustate this
        Suppose the triangle has corners at the origin, $(1,0)$ and $(0,1)$.
        The slope therefore is given by the equation $x+y=1$.
        It doesn't change the difficulty which variable we choose to integrate over first here so let's pick $x$ as the inner integral.
        The outer limits for $y$ are 0 to 1.
        For a fixed value of $y$, $x$ goes from 0 to $1-y$.
        \begin{equation}
            \implies A=\int_0^1\int_0^{1-y}\dd{x}\dd{y}=\int_0^1(1-y)\dd{y}=y-\left.\frac{1}{2}y^2\right|_0^1=\frac{1}{2}.
        \end{equation}
        \begin{example}
            Now consider integrating the function $f(x,y)=\frac{1-y}{1-x}$ over the region above.
            It will be easier if we switch the order of integration here, so we integrate over $y$ first:
            \begin{align}
                \int_0^1\int_0^{1-x}\frac{1-y}{1-x}\dd{y}\dd{x}&=\int_0^1\frac{1}{1-x}\left[y-\frac{1}{2}y^2\right]_0^{1-x}\dd{x}\\
                &=\int_0^1\frac{1}{2}(1+x)\dd{x}x=\left.\frac{1}{2}\left(x+\frac{1}{2}x^2\right)\right|_0^1=\frac{3}{4}.
            \end{align}
        \end{example}
        \begin{example}
            Find the integral of the function $f(x,y)=e^{-y^2}$ over a triangular region bounded by the lines $x=0$, $y=x$ and $y=1$.
            % TODO: include diagram for this
            This function has no antiderivative, so we need to do the $x$ integral first.
            $y$ goes from 0 to 1 and $x$ goes from 0 to $y$ for a fixed value of $y$.
            \begin{align}
                \int_0^1\int_0^y e^{-y^2}\dd{x}\dd{y}&=\int_0^1\left.xe^{-y^2}\right|_0^y\dd{y}\\
                &=\int_0^1 ye^{-y^2}\dd{y}=\frac{1}{2}\left(1-\frac{1}{e}\right).
            \end{align}
        \end{example}
        \begin{example}
            Find the mean value of $x$ and $y$ over a region bounded by $y=x$ and $y=4x(1-x)$.
            % TODO: include diagram for this
            To avoid having to split the region into two, make $x$ the outer variable.
            Hence, $x$ goes from 0 to $\frac{3}{4}$ (show this) and $y$ goes from $x$ to $4x(1-x)$.
            We have three integrals to evaluate: 1, $x$, and $y$ over the region.
            \begin{align}
                \int_0^\frac{3}{4}\int_x^{4x(1-x)}\dd{y}\dd{x}&=\int_0^\frac{3}{4}(3x-4x^2)\dd{x}=\left[\frac{3}{2}x^2-\frac{4}{3}x^3\right]_0^\frac{3}{4}=\frac{9}{32}\\
                \int_0^\frac{3}{4}\int_x^{4x(1-x)}x\dd{y}\dd{x}&=\int_0^\frac{3}{4}(3x^2-4x^3)\dd{x}=\left[x^3-x^4\right]_0^\frac{3}{4}=\frac{27}{256}\\
                \int_0^\frac{3}{4}\int_x^{4x(1-x)}y\dd{y}\dd{x}&=\int_0^\frac{3}{4}\left(8x^4-16x^3+\frac{15}{2}x^2\right)\dd{x}=\left[\frac{8}{5}x^5-4x^4+\frac{15}{6}x^3\right]_0^\frac{3}{4}\\
                &=\frac{27}{160}.
            \end{align}
            So using the formula from before, we have
            % TODO: reference this formula and also justify it
            \begin{equation}
                \bar{x}=\frac{37/256}{9/32}=\frac{3}{8},\quad\bar{y}=\frac{27/160}{9/32}=\frac{3}{5}.
            \end{equation}
        \end{example}

    \section{Integration using Plane Polar Coordinates}
        \paragraph{}
        Just like in 1D we can make a substitution to simplify the problem, in more than one dimension we can change the coordinate system to make things easier.
        One of the most common changes in coordinates is into polar coordinates.
        In order to change, we need to figure out what the area element, which is given by $\dd{A}=\dd{x}\dd{y}$ in cartesian coordinates, is represented by in polar coordinates.
        
        \paragraph{}
        % TODO: include a diagram to illustrate this
        Consider an annulus of thickness $\dd{r}$ and consider a section (is this the right word?) of the annulus with thickness $\dd{\theta}$.
        The area of the annulus is given by the difference in the area of two circles:
        \begin{equation}
            \pi[(r+\dd{r})^2-r^2]=\pi[2r\dd{r}+\dd{r}^2]=2\pi r\dd{r}\text{ as }\dd{r}\to 0.
        \end{equation}
        The fraction of a circle between $\theta$ and $\theta+\dd{\theta}$ is given by $\frac{\dd{\theta}}{2\pi}$, so the area of the infinitesimal area is
        \begin{equation}
            \dd{A}=\frac{\dd{\theta}}{2\pi}2\pi r\dd{r}=r\dd{r}\dd{\theta}.
        \end{equation}

        \paragraph{}
        % TODO: write an explanation and discussion of the Jacobian
        The more general approach to calculating the new area element in another coordinate system is to use the \textbf{Jacobian}.
        For a transformation between two cordinates $(x,y)\to(u,v)$, the area element $\dd{A}$ is given by
        \begin{equation}
            \dd{A}=J\dd{u}\dd{v}
        \end{equation},
        where
        \begin{equation}
            J=\begin{vmatrix}
                \pdv{x}{u} & \pdv{x}{v} \\
                \pdv{y}{u} & \pdv{y}{v}
            \end{vmatrix}=\left|\pdv{x}{u}\pdv{y}{v}-\pdv{x}{v}\pdv{y}{u}\right|.
        \end{equation}
        Hence in polar coordinates $x=r\cos\theta$, $y=r\sin\theta$, we have
        \begin{equation}
            \dd{A}=\left|\pdv{x}{r}\pdv{y}{\theta}-\pdv{x}{\theta}\pdv{y}{r}\right|\dd{r}\dd{\theta}=\left|r\cos^2\theta-(-r\sin^2\theta)\right|\dd{r}\dd{\theta}=r\dd{r}\dd{\theta}.
        \end{equation}
        \begin{example}
            Find the integral of $\sqrt{x^2+y^2}$ over the sector between $\theta=\frac{\pi}{6}$ and $\theta=\frac{\pi}{3}$ for $r\leq a$.
            This would be a very tricky integral to evaluate in cartesian coordinates but in polar coordinates the region is extremely simple.
            The limits of the integrals don't depend on $r$ and $\theta$, $r$ goes from 0 to $a$ and $\theta$ goes from $\frac{\pi}{6}$ to $\frac{\pi}{3}$.
            % TODO: include diagram for this
            The function also has a very simple form in polar coordinates $\sqrt{x^2+y^2}=r$.
            \begin{equation}
                \int_\frac{\pi}{6}^\frac{\pi}{3}\int_0^a r^2\dd{r}\dd{\theta}=\frac{\pi}{6}\int_0^a r^2\dd{r}=\frac{\pi a^3}{18}.
            \end{equation}
        \end{example}
        \begin{example}
            Calculate the integral of $x^2$ over a circular disc of radius $a$.
            Here, $\theta$ goes from 0 to $2\pi$ and $r$ goes from 0 to $a$.
            % TODO: include diagram for this
            \begin{equation}
                \int_0^{2\pi}\int_0^a r^2\cos^2\theta r\dd{r}\dd{\theta}=\int_0^{2\pi}\frac{a^4}{4}\cos^2\theta\dd{\theta}=\frac{\pi}{4}a^4.
            \end{equation}
        \end{example}
        \begin{example}
            Find the area bounded by two spirals given by $r_1(\theta)=a\theta^\frac{1}{2}$ and $r_2(\theta)=b\theta^\frac{1}{2}$, where $b>a>0$.
            % TODO: include diagram for this
            This is no longer a circularly-symmetric area region, so the order in which we integrate matters because the inner and outer limits will depend on each other.
            If we choose $r$ as the inner variable, then the calculation is easier because $r_1$ and $r_2$ are given purely in terms of $\theta$ and so they will be the inner limits for fixed $\theta$.
            \begin{align}
                A=\int_0^{2\pi}\int_{r_1(\theta)}^{r_2(\theta)}r\dd{r}\dd{\theta}&=\int_0^{2\pi}\left.\frac{1}{2}r^2\right|_{r_1(\theta)}^{r_2(\theta)}\dd{\theta}\\
                &=\frac{1}{2}\int_0^{2\pi}(b^2\theta-a^2\theta)\dd{\theta}\\
                &=\left.\frac{1}{4}(b^2-a^2)\theta^2\right|_0^{2\pi}=\frac{1}{4}(b^2-a^2)(4\pi^2)=\pi^2(b^2-a^2).
            \end{align}
        \end{example}
        \begin{example}
            Calculate the area of an ellipse.
            The equation of an ellipse is $\frac{x^2}{a^2}+\frac{y^2}{b^2}=1$.
            % TODO: include diagram for this
            To do this, we will first simplify the equation using the change of variables $u=\frac{x}{a}$, $v=\frac{y}{b}$.
            Then the equation of an ellipse becomes $u^2+v^2=1$, which is the equation of a circle.
            Note that the area element is $\dd{A}=ab\dd{u}\dd{v}$.
            Now our integration region is a circle so we can use polar coordinates, $u=r\cos\theta$, $v=r\sin\theta$, and the area element is given by $\dd{A}=abr\dd{r}\dd{\theta}$.
            Hence,
            \begin{equation}
                A=\int_0^{2\pi}\int_0^1 abr\dd{r}\dd{\theta}=\left.2\pi ab\frac{1}{2}r^2\right|_0^1=\pi ab.
            \end{equation}
        \end{example}
        \begin{example}
            Consider the integral of the function $(x^2+y^2)$ over the ellipse from the last example.
            Using the same coordinate transformation from last time, $x^2+y^2=a^2r^2\cos^2\theta+b^2r^2\sin^2\theta$.
            The limits are the same as in the last example.
            Hence,
            \begin{align}
                \int_0^{2\pi}\int_0^1 r^3(a^2\cos^2\theta+b^2\sin^2\theta)ab\dd{r}\dd{\theta}&=\int_0^{2\pi}\frac{ab}{4}(a^2\cos^2\theta+b^2\sin^2\theta)\dd{\theta}\\
                &=\frac{\pi ab}{4}(a^2+b^2).
            \end{align}
        \end{example}

\end{document}
