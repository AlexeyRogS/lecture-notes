\documentclass[../multivariate_calculus.tex]{subfiles}

\begin{document}

    \section{Taylor Series of Multivariate Functions}
        \paragraph{}
        First we recap some facts about Taylor series for univariate functions.
        If a single-variable function $f$ is \textbf{analytic}, then it can be represented around a point $x_0$ by its Taylor series:
        \begin{equation}
            f(x)=\sum_{n=0}^\infty\frac{(x-x_0)^n}{n!}\frac{\mathrm{d}^n f}{\mathrm{d}x^n}=f(x_0)+(x-x_0)\left.\dv{f}{x}\right|_{x_0}+\frac{(x-x_0)^2}{2}\left.\dv[2]{f}{x}\right|_{x_0}+\dots.
        \end{equation}
        % TODO: use Lagrange notation instead of Leibniz notation (would make it clearer that derivative is being evaluated at a specific point)
        This power series is valid for any value of $x$ within the \textbf{radius of convergence}, given by
        \begin{equation}
            \abs{x-x_0}<\lim_{n\to\infty}\abs*{\frac{\frac{1}{n!}\frac{\mathrm{d}^n f}{\mathrm{d}x^n}}{\frac{1}{(n+1)!}\frac{\mathrm{d}^{n+1}f}{\mathrm{d}x^{n+1}}}}=\lim_{n\to\infty}\abs*{(n+1)\frac{\frac{\mathrm{d}^n f}{\mathrm{d}x^n}}{\frac{\mathrm{d}^{n+1}f}{\mathrm{d}x^{n+1}}}}=R.
        \end{equation}
        % TODO: add examples of calculating the radius of convergence
        Recall that we can differentiate a power series without changing the radius of convergence.
        If we truncate the power series after a finite numbers of terms $n=k$, then there is an error between the value of the power series and the actual function which is given by
        \begin{equation}
            R_{k+1}=\frac{(x-x_0)^{k+1}}{(k+1)!}\left.\frac{\mathrm{d}^{k+1}f}{\mathrm{d}x^{k+1}}\right|_c,
        \end{equation} 
        where $c$ is a real number between $x$ and $x_0$.
        This is Taylor's theorem and it allows us to obtain bounds on the size of the error.
        \begin{example}
            What is the maximum possible error for the degree 4 Taylor polynomial for the function $e^x$ expanded about $x_0=0$ and used to approximate the function at $x=0.2$?
        \end{example}
        To examine a function at a point outside the radius of convergence, we have to change the expansion point $x_0$.
        We can do this by substituting.
        \begin{example}
            The function $f(u)=\frac{1}{1-u}$ has a power series representation about the point $u_0=0$ which has a radius of convergence $R=1$.
            \begin{equation}
                \frac{1}{1+u}=1-u+u^2-u^3+\dots\quad-1<u<1.
            \end{equation}
            Suppose we want a series representation for this function which convergences for $-1<x<3$.
            We make the substitution $u=\frac{x-1}{2}$ (equation of a straight line)
            \begin{equation}
                \frac{1}{1+\frac{x-1}{2}}=1-\left(\frac{x-1}{2}\right)+\left(\frac{x-1}{2}\right)^2-\left(\frac{x-1}{2}\right)^3+\dots\quad-1<\frac{x-1}{2}<1.
            \end{equation}
            Rearranging this equation gives
            \begin{equation}
                \frac{1}{1+x}=\frac{1}{2}\left[1-\left(\frac{x-1}{2}\right)+\left(\frac{x-1}{2}\right)^2-\left(\frac{x-1}{2}\right)^3+\dots\right].
            \end{equation}
        \end{example}
        % TODO: include note about multiplying power series together

        \paragraph{}
        Lets now try to find a Taylor series for a function of two variables.
        A taylor polynomial is the a polynomial approximation for a function $f(x,y)$ about a point $(x_0,y_0)$.
        So we want to capture how $f$ \textit{varies} in a local area around $(x_0,y_0)$.
        Consider a small line segment given by $x=x_0+ht$, $y=y_0+kt$.
        Then the first and second derivatives with respect to the parameter $t$ is given by
        \begin{align}
            \dv{f}{t}&=\pdv{f}{x}\dv{x}{t}+\pdv{f}{y}\dv{y}{t}=h\pdv{f}{x}+k\pdv{f}{y}\\
            \dv[2]{f}{t}&=\frac{\mathrm{d}}{\mathrm{d}t}\left(h\pdv{f}{x}+k\pdv{f}{y}\right)=h\left(\pdv[2]{f}{x}h+\frac{\partial^2 f}{\partial x\partial y}k\right)+k\left(\frac{\partial^2 f}{\partial x\partial y}h+\pdv[2]{f}{y}k\right)\\
            &=h^2\pdv[2]{f}{x}+2kh\frac{\partial^2 f}{\partial x\partial y}+k^2\pdv[2]{f}{y}.
        \end{align}
        Now we can express $f(x(t),y(t))=z(t)$ and expand $z(t)$ in a Taylor series about $t=0$:
        \begin{equation}
            z(t)=\underbrace{z(0)}_{f(x_0,y_0)}+t\underbrace{\left.\dv{z}{t}\right|_{t=0}}_{f^\prime(x_0,y_0)}+\frac{t^2}{2}\underbrace{\left.\dv[2]{z}{t}\right|_{t=0}}_{f^{\prime\prime}(x_0,y_0)}+\dots.
        \end{equation}
        Note that $ht=x-x_0$, $kt=y-y_0$, so
        \begin{align}
            t\left.\dv{z}{t}\right|_{t=0}&=(x-x_0)\left.\pdv{f}{x}\right|_{(x_0,y_0)}+(y-y_0)\left.\pdv{f}{y}\right|_{(x_0,y_0)}\\
            \frac{t^2}{2}\left.\dv[2]{z}{t}\right|_{t=0}&=\frac{(x-x_0)^2}{2}\left.\pdv[2]{f}{x}\right|_{(x_0,y_0)}+\frac{(y-y_0)^2}{2}\left.\pdv[2]{f}{y}\right|_{(x_0,y_0)}\\
            &\quad+(x-x_0)(y-y_0)\left.\frac{\partial^2 f}{\partial x\partial y}\right|_{(x_0,y_0)}.
        \end{align}
        Hence our Taylor expansion for a function of two variables about a point $(x_0,y_0)$ is
        \begin{align}\label{eq-taylor-series-2D}
            f(x,y)&=f(x_0,y_0)+(x-x_0)\left.\pdv{f}{x}\right|_{(x_0,y_0)}+(y-y_0)\left.\pdv{f}{y}\right|_{(x_0,y_0)}+\frac{(x-x_0)^2}{2}\left.\pdv[2]{f}{x}\right|_{(x_0,y_0)}\\
            &\quad+\frac{(y-y_0)^2}{2}\left.\pdv[2]{f}{y}\right|_{(x_0,y_0)}+(x-x_0)(y-y_0)\left.\frac{\partial^2 f}{\partial x\partial y}\right|_{(x_0,y_0)}
        \end{align}
        % TODO: this section needs some more formal proof
        % TODO: include general formula
        \begin{example}
            Find a second-order Taylor approximation for the function $f(x,y)=\sin xe^y$ about the point $(0,0)$.
            \begin{align}
                f(0,0)=0&\quad f_{xx}(0,0)=0\\
                f_x(0,0)=1&\quad f_{yy}(0,0)=0\\
                f_y(0,0)=0&\quad f_{xy}(0,0)=1.
            \end{align}
            Hence by the formula above the Taylor expansion is
            \begin{equation}
                f(x,y)=x+xy+\dots
            \end{equation}
            % TODO: include diagram for this
        \end{example}

    \section{Stationary Points}
        \paragraph{}
        For functions of one variable the possible types of stationary points are minima, maxima, and inflection points.
        For multivariate functions, we can have all of these as well as \textbf{saddle points}, which are like a minimum in one direction but a maximum in another.
        % TODO: include diagram to illustrate this
        To investigate the nature of stationary points for a 1D function, we look at the second derivative.
        Consider the Taylor series of a 1D function:
        \begin{equation}
            f(x)=f(x_0)+(x-x_0)f^\prime(x_0)+\frac{(x-x_0)^2}{2}f^{\prime\prime}(x_0)+\dots
        \end{equation}
        If we set $f^\prime(x_0)=0$ (stationary point), then we have
        \begin{equation}
            f(x)-f(x_0)=\underbrace{\frac{(x-x_0)^2}{2}}_\text{alway positive}f^{\prime\prime}(x_0)+\dots.
        \end{equation}
        So if $f^{\prime\prime}(x_0)>0$, then $f(x)-f(x_0)>0$ and $f(x_0)$ is a \textbf{local minimum}.
        Similarly if $f^{\prime\prime}(x_0)<0$, then $f(x_0)$ is a \textbf{local maximum}.
        % TODO: this section needs more rigour
        If $f^{\prime\prime}(x_0)=0$ but $f^{\prime\prime\prime}(x_0)\neq0$, then
        \begin{equation}
            f(x)-f(x_0)=f^{\prime\prime\prime}(x_0)\underbrace{\frac{(x-x_0)^3}{6}}_\text{changes sign depending on $x$}+\dots.
        \end{equation}
        % TODO: does this have consequences if f```=0?
        This leads to an inflection point.

        \paragraph{}
        We can use the same technique to investigate stationary points of two-variable functions.
        Using equation \ref{eq-taylor-series-2D}, if $(x_0,y_0)$ is a stationary point then $f_x(x_0,y_0)=f_y(x_0,y_0)=0$ and
        \begin{align}
            f(x,y)-f(x_0,y_0)&=\frac{(x-x_0)^2}{2}f_{xx}(x_0,y_0)+\frac{(y-y_0)^2}{2}f_{yy}(x_0,y_0)\\
            &\quad+(x-x_0)(y-y_0)f_{xy}(x_0,y_0)+\dots.
        \end{align}
        % TODO: rewrite this section assuming that all derivatives are evaluated at (x_0,y_0)? Would get rid of a lot of unnecessary text

        \paragraph{}
        First consider the case where $f_{xx}(x_0,y_0)=f_{yy}(x_0,y_0)=0$.
        Then we have
        \begin{equation}
            f(x,y)-f(x_0,y_0)=(x-x_0)(y-y_0)f_{xy}(x_0,y_0)+\dots.
        \end{equation}
        The sign varies since $x-x_0$ and $y-y_0$ can be positive or negative.
        Therefore this is a \textbf{saddle point}.

        \paragraph{}
        Now consider the case where $f_{xx}(x_0,y_0)\neq0$.
        In this case we can write
        \begin{align}
            \Delta f&=\frac{1}{f_{xx}}\left(\frac{\Delta x^2}{2}f_{xx}^2+\frac{\Delta y^2}{2}f_{yy}f_{xx}+\Delta x\Delta yf_{xy}f_{xx}\right)+\dots\\
            &=\frac{1}{f_{xx}}\left(\frac{\Delta x^2}{2}f_{xx}^2+\Delta x\Delta yf_{xy}f_{xx}+\Delta y^2f_{xy}^2-\Delta y^2f_{xy}^2+\Delta y^2f_{yy}f_{xx}\right)+\dots\\
            &=\frac{1}{f_{xx}}\left\{\left(\frac{\Delta x}{\sqrt{2}}f_{xx}+\frac{\Delta y}{\sqrt{2}}f_{yy}\right)^2+\frac{\Delta y^2}{2}(f_{xx}f_{yy}-f_{xy}^2)\right\}+\dots.
        \end{align}
        The quantity in the first set of parentheses is always positive, whereas the quantity in the second set, often called the \textbf{discriminant} $\Delta=f_{xx}f_{yy}-f{xy}^2$ can be positive or negative.
        If $\Delta>0$ and $f_{xx}>0$, then the whole expression is $>0$ and we have a \textbf{local minimum}.
        If $\Delta>0$ but $f_{xx}<0$, then $\Delta f<0$ and we have a \textbf{local maximum}.
        Finally, if $\Delta<0$ then the sign can change (why?) and we have a \textbf{saddle point}.
        The exact same analysis can be applied to the case $f_{yy}\neq0$.
        % TODO: what happens in the case \Delta=0?
        \begin{example}
            Consider the function $f(x,y)=x^3-6xy+y^3$.
            Stationary points occur when
            \begin{align}
                f_x&=3x^2-6y=0\\
                f_y&=3y^2-6x=0.
            \end{align}
            These equations are satisfied at the points $(0,0)$ and $(2,2)$.
            Let's calculate the discriminant and find the nature of the stationary points.
            \begin{equation}
                \Delta=f_{xx}f_{yy}-f_{xy}^2=(6x)(6y)-(-6)^2=36xy-36.
            \end{equation}
            Hence at $(0,0)$, $\Delta=-36$ so we have a saddle point.
            At $(2,2)$, $\Delta=108$ and $f_{xx}=24$ so we have a local minimum.
        \end{example}
        \begin{example}
            Consider $f(x,y)=x^4+y^4-2(x-y)^2$. We have stationary points when
            \begin{align}
                f_x&=4x^3-4(x-y)=0\\
                f_y&=4y^3+4(x-y)=0.
            \end{align}
            Adding the two equations gives $x=-y$, then using one of the equations gives the solutions $(0,0)$, $(\sqrt{2},-\sqrt{2})$ and $(-\sqrt{2},\sqrt{2})$.
            \begin{equation}
                \Delta=f_{xx}f_{yy}-f_{xy}^2=(12x^2-4)(12y^2-4)-(4)^2=144x^2y^2-48x^2-48y^2.
            \end{equation}
            At $(\sqrt{2},-\sqrt{2})$, $\Delta=384>0$ and $f_{xx}=20>0$ so this is a local minimum.
            By the same argument, there is also a local minimum at $(-\sqrt{2},\sqrt{2})$.
            At $(0,0)$, $\Delta=0$ so we can't say anything about the nature of the stationary point using the discriminant.
            Take $x=y$, then $f(x,y=x)=2x^4\geq0$.
            But now consider small $x$ and $y$ such that $(x-y)^2>x^4+y^4$, then $f(x,y)<0$.
            So since the sign can change, $(0,0)$ must be a saddle point.
        \end{example}

    \section{Optimisation Subject to Constraints}
        \paragraph{}
        Suppose we want to find the minimum or maximum of a functions along a certain path.
        This is equivalent to optimising the function with a \textbf{constraint} on the variables.
        There are many ways of solving these optimisation problems, but two of the most common are the \textbf{substitution method} and the \textbf{method of Lagrange multipliers}.
        The substitution method works well for simple problems where we can separate the variables in the constraint.
        For more complex problems, the method of Lagrange multipliers is very powerful.

        \paragraph{}
        Let's look at the substitution method first.
        Suppose we have a bivariate function $f(x,y)$ and a constraint between the two variables $y=y(x)$.
        Then we can substitution $f(x,y=y(x))$ which defines another function $g(x)$ that we can differentiate to get the extremum.
        % TODO: make this section more general
        \begin{example}
            Minimise $f(x,y)=x^2+y^2$ subject to the constraint $y=1-x$.
            % TODO: include diagram to illustrate this
            We define the composite function $g(x)$ as
            \begin{equation}
                g(x)=f(x,y=1-x)=x^2+(1-x)^2=2x^2-2x+1.
            \end{equation}
            We can now find the derivative and set it equal to 0:
            \begin{equation}
                g^\prime(x)=4x-2=0.
            \end{equation}
            So the function $f(x,y)$ is minimised along the path at $x=\frac{1}{2}$, $y=1-\frac{1}{2}=\frac{1}{2}$.
        \end{example}
        \begin{example}
            Minimise $f(x,y,z)=x^2+2y^2+z$ subject to the constraint $x+y^2-z=0$.
            \begin{equation}
                g(x,y)=f(x,y,z=x+y^2)=x^2+x+3y^2.
            \end{equation}
            \begin{equation}
                \pdv{g}{x}=2x+1=0,\quad\pdv{g}{y}=6y=0,
            \end{equation}
            so $x=-\frac{1}{2}$ and $y=0$. 
            It can be shown using the second derivatives that this is a minimum.
            Hence the minimum of $f(x,y,z)$ subject to the constraint is $f(-\frac{1}{2},0,-\frac{1}{2})=-\frac{1}{4}$.
        \end{example}

    \section{Method of Lagrange Multipliers}
        \paragraph{}
        If we can't solve an optimisation problem using the substitution method then the method of Lagrange multipliers is the next best thing.
        Suppose we have a function of three variables $f(x,y,z)$ and a constraint which we write in the general form $g(x,y,z)=0$.
        At an stationary point on the constraint surface, the directional derivative of $f$ in a direction tangent to the surface will be 0.
        % TODO: why is this?
        Since $g(x,y,z)=0$ defines a level set, the gradient of $g$ is normal to the surface everywhere.
        Thus, at a stationary point on the constraint surface, the gradient of $f$ is \textit{parallel} to the gradient of $g$.
        \begin{equation}
            \exists\lambda\in\RR\text{ such that }\nabla f=\lambda\nabla g.
        \end{equation}
        We call $\lambda$ the \textbf{Lagrange multiplier}.

        \paragraph{}
        In order to optimise a function subject to a constraint using this fact, we construct the function
        \begin{equation}
            F(x,y,z,\lambda)=f(x,y,z)-\lambda g(x,y,z).
        \end{equation} 
        Then setting all the first derivatives equal to 0 gives
        \begin{align}
            \pdv{F}{x}&=\pdv{f}{x}-\lambda\pdv{g}{x}=0\implies\pdv{f}{x}=\lambda\pdv{g}{x}\\
            \pdv{F}{y}&=\pdv{f}{y}-\lambda\pdv{g}{y}=0\implies\pdv{f}{y}=\lambda\pdv{g}{y}\\
            \pdv{F}{z}&=\pdv{f}{z}-\lambda\pdv{g}{z}=0\implies\pdv{f}{z}=\lambda\pdv{g}{z}\\
            \pdv{F}{\lambda}&=-g(x,y,z)=0.
        \end{align}
        % TOOD: this section needs more explanation
        Note that the first three equations are exactly the condition $\nabla f=\lambda\nabla g$.
        \begin{example}
            Minimise $f(x,y)=x^2+y^2$ subject to the constraint $g(x,y)=x+y-1=0$.
            This is the same problem from before which we can solve using the substitution method, but let's now solve it with the method of Lagrange multipliers.
            First we construct the function $F(x,y,\lambda)$, find the derivatives and set them equal to 0.
            \begin{equation}
                F(x,y,\lambda)=f(x,y)-\lambda g(x,y)=x^2-y^2-\lambda(x+y-1).
            \end{equation}
            \begin{align}
                \pdv{F}{x}&=2x-\lambda=0\\
                \pdv{F}{y}&=2y-\lambda=0\\
                \pdv{F}{\lambda}&=x+y-1=0.
            \end{align}
            Now adding the first two equations and substituting the last gives
            \begin{align}
                2x+2y-2\lambda&=0\\
                x+y&=\lambda\\
                \implies\lambda&=1.
            \end{align}
            So using the first two equations gives $x=\frac{1}{2}$ and $y=\frac{1}{2}$ like we found before.
        \end{example}
        \begin{example}
            Find and categorise the stationary points of the function $f(x,y,z)=x+y+z$ on the surface of a sphere centered at the origin with radius 5.
            This constraint takes the form $g(x,y,z)=x^2+y^2+z^2-5^2=0$.
            The function $F(x,y,z,\lambda)$ and its derivatives (set equal to 0) are:
            \begin{align}
                F(x,y,z,\lambda)&=x+y+z-\lambda(x^2+y^2+z^2-25)\\
                &=\pdv{F}{x}=1-2x\lambda=0\\
                &=\pdv{F}{y}=1-2y\lambda=0\\
                &=\pdv{F}{z}=1-2z\lambda=0\\
                &=\pdv{F}{\lambda}=x^2+y^2+z^2-25=0.
            \end{align}
            Thus, we have $x=y=z=\frac{1}{2\lambda}$ by the first three derivatives and $\frac{3}{4\lambda^2}=25$ by the last.
            From this we we obtain $\lambda=\pm\frac{\sqrt{3}}{10}$ which gives $x=y=z=\pm\frac{5}{\sqrt{3}}$.
            We therefore have two stationary points, $(\frac{5}{\sqrt{3}},\frac{5}{\sqrt{3}},\frac{5}{\sqrt{3}})$ and $(-\frac{5}{\sqrt{3}},-\frac{5}{\sqrt{3}},-\frac{5}{\sqrt{3}})$.
            To find the nature of the stationary points, we expand about the stationary points (we will use the constraint to ensure we remain on the sphere):
            \begin{align}
                x&=\frac{5}{\sqrt{3}}+\varepsilon\\
                y&=\frac{5}{\sqrt{3}}+\delta\\
                z&=\sqrt{25-\left(\frac{5}{\sqrt{3}}+\varepsilon\right)^2-\left(\frac{5}{\sqrt{3}}+\delta\right)^2}\\
                &=\frac{5}{\sqrt{3}}\sqrt{1-\frac{2\sqrt{3}}{5}(\varepsilon+\delta)-\frac{2}{25}(\varepsilon^2+\delta^2)}\\
                &\approx\frac{5}{\sqrt{3}}(1-\frac{\sqrt{3}}{5}(\varepsilon+\delta)-\frac{3}{50}(\varepsilon^2+\delta^2)-\frac{\sqrt{3}}{20}(\varepsilon+\delta)^2)\\
            \end{align}
            Where in the last line we have used a Taylor series expansion.
            % TODO: check this section and make it more rigorous
            Hence
            \begin{align}
                f(x,y,z)&=x+y+z\\
                &=\frac{5}{\sqrt{3}}+\varepsilon+\frac{5}{\sqrt{3}}+\delta+\frac{5}{\sqrt{3}}-(\varepsilon+\delta)-\frac{15}{50\sqrt{3}}(\varepsilon^2+\delta^2)-\frac{1}{4}(\varepsilon+\delta)^2\\
                &=5\sqrt{3}-\text{something positive (and small?)}
            \end{align}
            This is $>0$, so we have a local maximum.
        \end{example}
        \begin{example}
            Consider an ellipse $\frac{x^2}{a^2}+\frac{y^2}{b^2}=1$.
            What is the largest rectangle that can fit inside the ellipse?
            % TODO: include diagram to illustrate this
            To turn this into an optimisation problem that we can solve, consider the top right corner of the ellipse $(x,y)$.
            Due to symmetry, the area of a rectangle with a corner at $(x,y)$ is $4xy$, and this is the function that we want to \textit{minimise} under the constraint that $x$ and $y$ lie on the ellipse.
            Using the method of Lagrange multipliers:
            \begin{align}
                F(x,y,\lambda)&=4xy-\lambda\left(\frac{x^2}{a^2}+\frac{y^2}{b^2}-1\right)\\
                \pdv{F}{x}&=4y-\frac{2x\lambda}{a^2}=0\\
                \pdv{F}{y}&=4x-\frac{2y\lambda}{b^2}=0\\
                \pdv{F}{\lambda}&=\frac{x^2}{a^2}+\frac{y^2}{b^2}-1=0.
            \end{align}
            From the $x$ and $y$ derivatives, we get $x=\frac{y\lambda}{2b^2}$ and $y=\frac{x\lambda}{2a^2}$.
            Combining these equations with the last one, we find $\lambda=2ab$.
            Hence by back-substitution we get $x=\frac{a}{\sqrt{2}}$ and $y=\frac{b}{\sqrt{2}}$, which implies $f(x,y)=2ab$.
        \end{example}

\end{document}
