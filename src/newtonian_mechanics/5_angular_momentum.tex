\documentclass[../newtonian_mechanics.tex]{subfiles}

\begin{document}

    \section{Rigid Body Rotation}
        \paragraph{}
        A rigid body is a system of particles which are fixed together such that they move together, no matter the forces applied.
        We have seen in previous chapters that it is possible to treat rigid bodies as point masses when considering linear motion.
        However, when considering rotational effects as well, we need to consider the macroscopic extent of an object.
        Consider a rigid body in motion and look at the total kinetic energy, which is given by the sum of kinetic energy of each constituent particle:
        \begin{equation}
            K=\sum_i\frac{1}{2}m_iv_i^2.
        \end{equation}
        We can split the velocity of each particle into the sum of velocity around the centre of rotation and velocity along the line of motion.
        \begin{align}
            K&=\sum_i\frac{1}{2}m_i(\vec{v}_{i,\text{rot}}+\vec{v}_{i,\text{tan}})^2\\
            &=\sum_i\frac{1}{2}m_i(v_{i,\text{rot}}^2+v_{i,\text{tan}}^2+2\vec{v}_{i,\text{rot}}\cdot\vec{v}_{i,\text{tan}})\\
            &=\underbrace{\sum_i\frac{1}{2}m_iv_{i,\text{rot}}^2}_{K_\text{rotational}}+\underbrace{\sum_i\frac{1}{2}m_iv_{i,\text{tan}}^2}_{K_\text{linear}}.
        \end{align}
        The cross-term in the square cancels out (why?).
        % TODO: explain this fully
        % TODO: include a diagram of this
        This rotational kinetic energy can be written in terms of the angular velocity $\omega$ since it is the same for each particle in a rigid body.
        \begin{align}
            K_\text{rot}&=\sum_i\frac{1}{2}m_iv_{i,\text{rot}}^2=\sum_i\frac{1}{2}m_i(r_i\omega)^2\\
            &=\frac{1}{2}\omega^2\sum_i m_ir_i^2\\
            &=\frac{1}{2}I\omega^2.
        \end{align}
        $I$ is called the \textbf{moment of inertia}, and it is kind of an angular equivalent of mass.
        Notice how it is calculated in a similar way to the centre of mass except with the square of $r$.
        Also notice how in the formula for $K_\text{rot}$, $I$ and $\omega$ play the role of $m$ and $v$ respectively, which shows how they are analogous to the linear quantities.
        However $I$, just like the centre of mass, depends on how the mass is distributed in an object.
        For example, consider a solid disc and a hoop of the same mass.
        % TODO: include a diagram for this
        From the formula for $I$, we can see that since all of the mass in the hoop is concentrated further out, the moment of inertia will be larger than the disc.
        This means that for the same angular velocity, the rotational kinetic energy of the hoop will be larger than the disc.
        $I$ also depends on the axis of rotation.

    \section{Torque}
        \paragraph{}
        Torque is defined as the \textbf{moment} of force, that is, the product of the distance from a reference point and the force.
        % TODO: include a proper discussion of pivot points and what torque does
        In terms of vectors, this is given by the cross product
        \begin{equation}
            \vec{\tau}=\vec{r}\times\vec{F}.
        \end{equation}
        The magnitude of $\vec{\tau}$ is given by
        \begin{align}
            \tau&=\abs{\vec{r}}\abs{\vec{F}}\sin\theta\\
            &=rF_\text{tan}.
        \end{align}
        % TODO: include diagram for this
        This formula tells us that for a fixed radius, the torque has maximum magnitude when $\theta=\pm\frac{\pi}{2}$ i.e. the force acts \textit{perpendicular} to the radius.
        It also tells us that if the force acts along the same line as the radius ($\theta=0$ or $\theta=\pi$), then the torque is equal to 0.
        We can also see from the formula that the magnitude of torque depends on the radius.
        For the same force, if it is applied further away from the centre, the torque will be greater.
    
        \paragraph{}
        Using Newton's second law on the equation above, we can write the equation for torque above as
        \begin{equation}
            \tau=\sum_i\tau_i=\sum_im_ir_ia_i=\sum_im_ir_i^2\alpha=I\alpha.
        \end{equation}
        % TODO: include vector version of this
        This works because for a rigid body, the angular acceleration is the same for all parts of the body, just like angular velocity.
        We can see from this we have an perfect angular analogue of Newton's second law for torques.
        \begin{example}
            Consider two connected masses on a massless pulley with $m_1>m_2$.
            Suppose the system starts from rest and assume the string is massless, inextensible, and lies vertically.
            Find an expression for the magnitude of acceleration of the masses.
            % TODO: include diagram for this
            To do this, we have to analyse the forces acting on the blocks and also the torques acting on the pulley.
            % TODO: complete this example
        \end{example}
        % TODO: discuss how torque depends on coordinate system

    \section{Static Equilibrium}
        \paragraph{}
        In the net force and the net torque on a rigid body are both 0, then it is in \textbf{static equilibrium}.
        % TODO: include discussion of whether objects can be rotating in static equilibrium
        % TODO: include discussion about how static equilibrium implies net torque is zero in ALL inertial reference frames
        \begin{eqnarray}
            \sum_i\vec{F}_i=0\quad\text{and}\quad\sum_i\vec{\tau}_i=0.
        \end{eqnarray}
        Note that we are free to choose any point as the origin to make finding the net torqe easier.
        \begin{example}
            Consider a beam of mass 10kg and length 4m.
            It sits on a fulcrum placed 1m from one end of the beam, and is supported from the other end by a string.
            Find the tension in the string and the force of the beam on the fulcrum.
            % TODO: complete this example
        \end{example}
        \begin{example}
            A ladder weighing 10kg rests on a smooth wall.
            Find the the static friction force between the floor and the ladder.
            % TODO: complete this example
        \end{example}
        \begin{example}
            Consider a sign hanging from a bar attached to a wall supported by a string.
            Find the force between the bar and the wall.
            % TODO: complete this example
        \end{example}

\end{document}
