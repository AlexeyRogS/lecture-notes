\documentclass[../real_analysis.tex]{subfiles}

\begin{document}

    \section{Why do we Need the Real Numbers?}\label{sec:why-do-we-need-real-numbers}
        The rational numbers have many desirable properties by themselves which allow us to do a lot of mathematics with them. One of these is that they are closed under regular arithmetic operations of addition and multiplication. In fact, $\mathbb{Q}$ forms an \textbf{ordered field} when paired with the total order $\leq$. The rational numbers, along with the integers and natural numbers, satisfy the \textbf{Archimedean property},
        \[\forall q\in\mathbb{Q},\ \exists n\in\mathbb{N}\ \mathrm{s.t.}\ n>q,\]
        and they also possess the property of \textbf{density}, which states that between any two rational numbers there exists another rational number.
        % TODO: add footnote for ordered field
        \begin{theorem}
            $\mathbb{Q}$ is dense.
        \end{theorem}
        \begin{proof}
            Let $p,q\in\mathbb{Q}$ with $p<q$. Then let $r=\frac{p+q}{2}$, so $r\in\mathbb{Q}$ and $p<r<q$.
        \end{proof}
        This process can be repeated indefinitely, which shows that there are actually \textit{infinitely many} rational numbers between each rational number. This being the case, there are still many numbers that we need that are not in $\mathbb{Q}$. Specifically, since limits are one of the central topics of study in analysis, it would be nice if we could guarantee the existence of limits for convergent sequences. It turns out that this is equivalent to the completeness property, which we will now define.
        \begin{definition}
            A partially ordered set $X$ is \textbf{complete}\footnote{This property is sometimes called \textbf{Dedekind completeness}, or simply the \textbf{least-upper-bound property}.} if every non-empty subset of $X$ which is bounded above (has an upper bound) has a least upper bound (supremum) in $X$.
        \end{definition}
        It can be shown wih a counterexample that the rational numbers are not complete.
        \begin{theorem}
            Let $S=\{q\in\mathbb{Q}\,:\,q^2<2\}$. Then there does not exist a rational number $r$ such that $r=\sup(S)$.
        \end{theorem}
        \begin{proof}
            $S$ is clearly bounded from above, for example $\frac{3}{2}$ is an upper bound $\left(\left(\frac{3}{2}\right)^2 = \frac{9}{4} > 2\right)$, so by the completeness property we would expect a supremum to exist. Let us assume by way of contradiction that $r=\sup(S)$ and that $r\in\mathbb{Q}$.
            Then, by the trichotomy law, we have $r^2<2$, $r^2>2$, or $r^2=2$.\\
            If $r^2<2$, let $\varepsilon>0$, then
            \begin{align}
                (r+\varepsilon)^2 &= r^2+\varepsilon(2r+\varepsilon)\\
                &< r^2+\varepsilon(2r+1)\tag{$\varepsilon<1$}\\
                &< r^2+2-r^2\tag{$\varepsilon<\frac{2-r^2}{2r+1}$}\\
                &= 2.
            \end{align}
            So there exists a rational number greater than $r$ which is an element of $S$, which means that $r$ is not an upper bound for $S$, contradicting our assumption that $r=\sup(S)$.\\
            If $r^2<2$, let $\varepsilon>0$, then
            \begin{align}
                (r-\varepsilon)^2 &= r^2-\varepsilon(2r+\varepsilon)\\
                &> r^2-\varepsilon(2r+1)\tag{$\varepsilon<1$}\\
                &> r^2-r^2+2\tag{$\varepsilon<\frac{r^2-2}{2r+1}$}\\
                &= 2.
            \end{align}
            So there exists a rational number less than $r$ which is an upper bound for $S$, which contradicts our assumption that $r=\sup(S)$.\\\\
            Thus $r^2=2$, which we have already shown is impossible if $r\in\mathbb{Q}$, meaning that no supremum exists in $\mathbb{Q}$ for $S$.
        \end{proof}
        % TODO: ordered field stuff 4.2.7 and 4.2.9 in tao

    \section{Construction of the Reals}\label{sec:construction-of-the-reals}
        We are giving this real number system a lot of praise, but how can we actually show that it exists?

    \section{Subsets of the Real Line}\label{sec:subsets-of-the-real-line}
        It is helpful to note at this point that the completeness property from before also guarantees the existence of infimums for non-empty subsets that are bounded below.
        \begin{theorem}
            Every non-empty subset of $\mathbb{R}$ that is bounded below has an infimum in $\mathbb{R}$.
        \end{theorem}
        \begin{proof}
            Let $S$ be a non-empty subset of $\mathbb{R}$ that is bounded above. Then by the completeness property there exists $\ell\in\mathbb{R}$ such that $\ell=\sup(S)$.
            Now consider the set $S^\prime=\{-s\,:\,s\in S\}$. It is clear that $S^\prime$ is a non-empty subset of $\mathbb{R}$ which is bounded below, and that $-\ell=\inf(S)$.
        \end{proof}

        \subsection{Intervals}\label{subsec:intervals}
            Quite often we will need some compact notation for denoting a continuous subset of $\mathbb{R}$, such as the positive real numbers, all real numbers between -1 and 4, etc. These subsets are called \textit{intervals}, and we will define them now.
            \begin{definition}
                Let $a, b \in \mathbb{R}$ with $a<b$. Then the \textbf{closed interval} from $a$ to $b$ is given by
                \[[a, b] = \{x \in \mathbb{R} : a \leq x \leq b\},\]
                and the \textbf{open interval} from $a$ to $b$ is given by
                \[(a, b) = \{x \in \mathbb{R} : a < x < b\}.\]
                Likewise the \textbf{half-open intervals} may be defined similarly as
                \[(a, b] = \{x \in \mathbb{R} : a < x \leq b\},\]
                \[[a, b) = \{x \in \mathbb{R} : a \leq x < b\}.\]
            \end{definition}
            To denote a half-unbounded or fully unbounded interval, we can use a $\infty$ symbol. For example:
            \[[a, \infty) = \{x \in \mathbb{R} : a \leq x\},\]
            or
            \[(-\infty, 0) = \{x \in \mathbb{R} : x < 0\},\]
            and we could also say
            \[(-\infty, \infty) = \mathbb{R}.\]
            It is also useful to note that
            \[\sup((a, b))=\sup([a, b])=b,\quad\inf((a, b))=\inf([a, b])=a\].
            % TODO: heine-borel theorem for the line (Theorem 9.1.24 in tao)

    \section{Absolute Value}\label{sec:absolute-value}
        \begin{definition}
            Let $x \in \mathbb{R}$. Then the \textbf{absolute value of $x$} is defined as
            \[\abs{x} = \begin{cases}
                            x & \text{if}\ x \geq 0 \\
                            -x & \text{if}\ x < 0
                        \end{cases}.\]
        \end{definition}
        The absolute value has several useful properties.
        \begin{theorem}[Properties of Absolute Value]
            Let $x, y \in \mathbb{R}$, then
            \begin{enumerate}[label={\upshape(\roman*)}]
                \item $\abs{x} \geq 0$,
                \item $\abs{x} = 0 \iff x = 0\ \big(\abs{x-y} = 0 \iff x=y\big)$,
                \item $\abs{-x} = \abs{x} = \max(x, -x)$,
                \item $x \leq \abs{x}$,
                \item $\abs{x} \leq a \iff -a \leq x \leq a,\ \text{for}\ a \in \mathbb{R}$,
                \item Similarly, $\abs{x} \geq a \iff x \leq -a$ or $x \geq a$,
                \item $\abs{xy} = \abs{x}\abs{y}$.
            \end{enumerate}
        \end{theorem}
        \begin{proof}
            (i), (ii), (iii), and (iv) all follow immediately from the definition.
            \begin{itemize}
                \item[(v)] Since $\abs{x} \geq 0, a \geq 0$. Therefore clearly $x \geq -a$ and $x \leq \abs{x} \leq a$.
                \item [(vi)] Follows similarly.
                \item[(vii)] The result follows immediately if $x$ or $y$ is 0, or if $x$ and $y$ are both either positive negative. Lets assume without loss of generality that $x < 0$, then, noting that $xy = -\abs{xy}$ and $x = -\abs{x}$ we find
                \[\abs{xy} = -xy = (-x)(y) = \abs{x}\abs{y}.\]
            \end{itemize}
        \end{proof}
        These basic properties allow us to prove a basic theorem about inequalities with absolute values which will be absolutely indispensable in more advanced proofs.
        \begin{theorem}[Triangle Inequalities]
            Let $x, y \in \mathbb{R}$, then we have
            \begin{enumerate}[label={\upshape(\roman*)}]
                \item (Triangle Inequality) $\abs{x+y} \leq \abs{x} + \abs{y}$,
                \item (Reverse Triangle Inequality) $\big\lvert\abs{x} - \abs{y}\big\rvert \leq \abs{x - y}$.
            \end{enumerate}
        \end{theorem}
        \begin{proof}
            The proof of these statements follow directly from the basic properties of the absolute value, with the reverse triangle inequality following directly from the normal triangle inequality.
            \begin{itemize}
                \item[(i)] From (iv) we have $-\abs{x} \leq x \leq \abs{x}$ and similarly for $y$. Adding these two inequalities we get
                \begin{align}
                    -\abs{x} -\abs{y} &\leq x + y \leq \abs{x} + \abs{y} \\
                    - \big(\abs{x} + \abs{y}\big) &\leq x + y \leq \abs{x} + \abs{y}, \\
                \end{align}
                and thus from (v) we have $\abs{x+y} \leq \abs{y}$.
                \item[(ii)] Following from the triangle equality, note that
                \begin{align}
                    \abs{x} &= \abs{(x-y)+y} \leq \abs{x-y} + \abs{y} \\
                    \abs{y} &= \abs{(y-x)+x} \leq \abs{y-x} + \abs{x}.
                \end{align}
                Rearranging each equation in turn, we get
                \begin{align}
                    \abs{x} - \abs{y} &\leq \abs{x-y} \\
                    \abs{y} - \abs{x} = -\big(\abs{x} - \abs{y}\big) &\leq \abs{y-x} = \abs{x-y},
                \end{align}
                and so using (vi) we obtain $\big\lvert\abs{x} - \abs{y}\big\rvert \leq \abs{x - y}$.
            \end{itemize}
        \end{proof}
        % TODO: add note about how absolute value is analogous to "distance" between numbers
        % TODO: add simpler definition of bounded using absolute value

    \section{Inequalities}\label{sec:inequalities}
        Manipulating inequalities is an extremely useful trick in analysis because they appear everywhere. We have already discovered and proved the triangle inequality, and now we will discuss a few more.
        \begin{theorem}[General Triangle Inequality]
            % https://math.stackexchange.com/questions/195582/general-proof-for-the-triangle-inequality
        \end{theorem}
        \begin{theorem}[AM-GM Inequality]
            % https://en.wikipedia.org/wiki/Inequality_of_arithmetic_and_geometric_means
        \end{theorem}
        \begin{theorem}[Cauchy-Swartz Inequality]

        \end{theorem}

\end{document}
