\documentclass[../real_analysis.tex]{subfiles}

\begin{document}

    \section{Sequences \& Convergence}
        \paragraph{}
        A sequence is an infinitely long ordered list of real numbers. The precise definition is as follows.
        \begin{definition}
            A \textbf{sequence} of real numbers is a function
            \[f:\mathbb{N}\to\mathbb{R}.\]
            We denote $f(n)$ as $x_n$ and write the sequence as
            \[(x_n)_n,\quad(x_n)_{n\in\mathbb{N}},\quad(x_n)_{n=1}^\infty,\quad(x_1,x_2,\dots).\]
        \end{definition}
        The elements of a sequence can be given by a formula for the $n$-th term, a recurrence relation, or a pattern given by the first few terms.
        \begin{equation*}
            (2,4,6,8,\dots)=(2n)_n\ \mathrm{i.e.}\ x_n=2n,\quad\left(1,\frac{1}{2},\frac{1}{3},\frac{1}{4},\dots\right)=\left(\frac{1}{n}\right)_n\ \mathrm{i.e.}\ x_n=\frac{1}{n}.
        \end{equation*}
        Some sequences converge to a fixed value, or \textbf{limit}, meaning that as $n$ gets larger and larger, $x_n$ gets closer and closer to the limit. An important question to ask is how do we define if a sequence converges or not, or how do we find the limit if it is only reached after an infinite amount of steps? For example, the sequence $\left(\frac{1}{n}\right)_n$ from above intuitively converges to 0 as $n$ tends to infinity, but how can we prove this mathematically?
        % TODO: add definition of e-closeness as stepping stone (def 4.3.4 in tao)
        \begin{definition}
            Let $(x_n)_n$ be a sequence of real numbers and let $x\in\mathbb{R}$.
            \begin{enumerate}
                \item $(x_n)_n$ is a \textbf{convergent sequence} with \textbf{limit} $x$ if
                \begin{equation*}
                    \forall\varepsilon>0,\ \exists N\in\mathbb{N}\ :\ n\geq N\implies\abs{x_n-x}\leq\varepsilon.
                \end{equation*}
                In english this says that `for all $\varepsilon>0$, there exists a natural number $N$ such that for $n\geq N$ the distance between $x_n$ and the limit is less than $\varepsilon$'. We say $(x_n)_n$ \textbf{tends to $x$ as $n$ tends to infinity}.
                \item $(x_n)_n$ \textbf{tends to infinity} as $n$ tends to infinity if
                \begin{equation*}
                    \forall K>0,\ \exists N\in\mathbb{N}\ :\ n\geq N\implies x_n\geq K.
                \end{equation*}
                \item Likewise, $(x_n)_n$ \textbf{tends to minus infinity} as $n$ tends to infinity if
                \begin{equation*}
                    \forall K>0,\ \exists N\in\mathbb{N}\ :\ n\geq N\implies x_n\leq -K.
                \end{equation*}
                \item A sequence is \textbf{divergent} if the limit does not exist or it tends to plus/minus infinity.
            \end{enumerate}
        \end{definition}
        % TODO: introduce limit notation here as well
        % TODO: upper and lower limits (definition 3.16 in rudin)
        Let's take a closer look at definition 4.2.1. Being unfamiliar with techniques in analysis, it might look very confusing, but it is actually quite a simple and clever idea. The key thing to realise is that converging sequences \textit{never actually reach their limit} (unless they are constant), so in order to capture this idea of reaching a limit after infinitely many terms in concrete mathematics, we introduce this idea of bounding the elements of the sequence within a certain distance of the limit. Therefore, what this definition is really saying is that \textit{no matter how small} epsilon is, there exists a step in the sequence such that from that point onwards all values in the sequence are inside the interval $(x-\varepsilon,x+\varepsilon)$.
        % TODO: include number line diagram
        Once this idea is understood, everything in analysis starts to make sense! Let us now look at a few examples.
        \begin{example}
            Claim: The sequence $\left(\frac{1}{n}\right)_n$ tends to 0 as $n$ tends to infinity.\\
            \textit{Proof}. Let $\varepsilon>0$ and choose $N>\frac{1}{\varepsilon}$. Then for $n\geq N$ we have
            \begin{equation*}
                \abs{x_n-x}=\abs*{\frac{1}{n}-0}=\frac{1}{n}\leq\frac{1}{N}<\varepsilon.
            \end{equation*}
        \end{example}
        \begin{example}
            Claim: $\left(\frac{1}{n^2}\right)_n\to0$ as $n\to\infty$.\\
            \textit{Proof}. Let $\varepsilon>0$ and choose $N>\frac{1}{\sqrt{\varepsilon}}$. Then for $n\geq N$ we have
            \begin{equation*}
                \abs{x_n-x}=\abs*{\frac{1}{n^2}-0}=\frac{1}{n^2}\leq\frac{1}{N^2}<\varepsilon.
            \end{equation*}
            It is important to note that we do not have to always use the most optimal value for $N$. In this proof, choosing $N>\frac{1}{\varepsilon}$ would have been sufficient because $\frac{1}{N^2}\leq\frac{1}{N}$.
        \end{example}
        \begin{example}
            Claim: $x_n=\frac{2n+1}{3n+5}$, $(x_n)_n\to\frac{3}{2}$ as $n\to\infty$.\\
            \textit{Proof}. Let $\varepsilon>0$ and choose $N>\frac{7}{9}\varepsilon$. Then for $n\geq N$ we have
            \begin{align*}
                \abs{x_n-x}&=\abs*{\frac{2n+1}{3n+5}-\frac{2}{3}}=\abs*{\frac{3(2n+1)-2(3n+5)}{3(3n+5)}}\\
                &=\abs*{\frac{6n-6n+3-10}{9n+15}}=\abs*{\frac{-7}{9n+15}}=\frac{7}{9n+15}\\
                &\leq\frac{7}{9N+15}\leq\frac{7}{9N}<\varepsilon.
            \end{align*}
        \end{example}
        \begin{example}
            Claim: $(n^3)_n\to\infty$ as $n\to\infty$.\\
            \textit{Proof}. Let $K>0$ and choose $N>\sqrt[3]{K}$. Then for $n\geq N$ we have
            \begin{equation*}
                x_n=n^3\geq N^3>K.
            \end{equation*}
        \end{example}
        \begin{example}
            Claim: $x_n=(-1)^n$, $(x_n)_n$ is divergent.\\
            \textit{Proof}. Suppose by way of contradiction that there is a limit $x\in\mathbb{R}$.\\
            Let $\varepsilon=\frac{1}{2}$, then since the limit exists there exists $N$ such that $n\geq N$ implies $\abs{x_n-x}<\frac{1}{2}$.\\
            Since there are arbitrarily large even numbers, there is always some $n\geq N$ (for example $n=2N$) such that $x_n=1$, so $\abs{1-x}<\frac{1}{2}\implies x>\frac{1}{2}$.\\
            Also, since there are arbitrarily large odd numbers, there is always some $n\geq N$ ($n=2N+1$) such that $x_n=-1$, so $\abs{-1-x}<\frac{1}{2}\implies x<-\frac{1}{2}$.\\
            But then $\frac{1}{2}<x<-\frac{1}{2}$, which is impossible, so the limit cannot exist.
        \end{example}
        \begin{theorem}[Standard Sequences]\label{std-seq}
            Let $a\in\mathbb{R}$.
            \begin{enumerate}[label={\upshape(\roman*)}]
                \item Let $x_n=a$, then $(x_n)_n\to a$ as $n\to\infty$ (constant sequence).
                \item If $\abs{a}<1$, then $(a^n)_n\to0$ as $n\to\infty$.
                \item If $a>1$, then $(a^n)_n\to\infty$ as $n\to\infty$.
            \end{enumerate}
        \end{theorem}
        \begin{proof}
            \begin{enumerate}[label={\upshape(\roman*)}]
                \item Let $\varepsilon>0$. Then given $N\in\mathbb{N}$, $n\geq N$ implies
                \begin{equation*}
                    \abs{x_n-x}=\abs{x-x}=0<\varepsilon.
                \end{equation*}
                \item Let $\varepsilon>0$ and choose $N>\frac{\ln{\varepsilon}}{\ln{\abs{a}}}$. Then for $n\geq N$ we have
                \begin{equation*}
                    \abs{x_n-x}=\abs{a^n-0}=\abs{a^n}\leq\abs{a^N}<\varepsilon.
                \end{equation*}
                \item Let $K>0$ and choose $N>\frac{\ln{K}}{\ln{a}}$. Then for $n\geq N$ we have
                \begin{equation*}
                    a^n\geq a^N>K.
                \end{equation*}
            \end{enumerate}
        \end{proof}

    \section{Properties of Limits}
        \paragraph{}
        Now that we have established how to prove the existence of limits, we will consider the uniqueness of them, as well as some algebraic properties of convergent sequences.
        \begin{theorem}
            If a sequence converges, its limit is unique.
        \end{theorem}
        \begin{proof}
            Let $(x_n)_n$ be a convergent sequence and suppose that $(x_n)_n\to x$ and $(x_n)_n\to y$. Let $\varepsilon>0$. Then there exists $N_1\in\mathbb{N}$ such that $n\geq N_1\implies\abs{x_n-x}<\varepsilon$, but there also exists $N_2\in\mathbb{N}$ such that $n\geq N_2\implies\abs{x_n-y}<\varepsilon$.
            Now,
            \begin{align*}
                \abs{x-y} &= \abs{x-x_n+x_n-y}\tag{for $n\geq\max\{N_1, N_2\}$}\\
                &\leq\abs{x-x_n}+\abs{x_n-y}\\
                &<\varepsilon+\varepsilon=2\varepsilon.
            \end{align*}
            Therefore, $\abs{x-y}=0$, so $x=y$ (since $\forall\varepsilon>0$, $0\leq a<\varepsilon\implies a=0\ \forall a\in\mathbb{R}$). Thus, the two limits are in fact the same.
        \end{proof}
        % TODO: don't need to recall, just cite proof
        The next theorem will serve as a stepping stone that will make some more useful theorems much easier to prove. First, recall that a set is bounded if and only if it is bounded both above and below. We extend this definition to the elements of sequences, saying that a sequence is \textbf{bounded} if there exists $M\geq0$ such that $\abs{x_n}\leq M\ \forall n\in\mathbb{N}$. This is a very quick and easy way to establish the range of a sequence that will come in handy later.
        % TODO: move footnote outside theorem
        \begin{theorem}\label{cvg-seq-bounded}
            Every convergent sequence is bounded\footnote{Note that the converse of this theorem is obviously not true, $((-1)^n)_n$ is bounded, but does not converge.}.
        \end{theorem}
        \begin{proof}
            Let $(x_n)_n$ be a convergent sequence with $x_n\to x$. Then there exists $N\in\mathbb{N}$ such that $n\geq N\implies\abs{x_n-x}<1$. Consider the terms in the sequence $x_1, x_2,\dots,x_{N-1}$ and let $M=\max\{\abs{x_1},\abs{x_2},\dots,\abs{x_{N-1}}\}$.
            Now, for all $n\geq N$, $\abs{x_n}\leq\max\{M, \abs{x}+1\}$ (since $\abs{x_n-x}\geq\abs{x_n}-\abs{x}<1\implies\abs{x_n}<1+\abs{x}$), which is a constant, hence $(x_n)_n$ is bounded.
        \end{proof}
        Now that we have this tool, we can proceed to some more interesting results about how limits of sequences behave when we add and multiply sequences together.
        \begin{theorem}\label{seq-lim-props}
            Let $a\in\mathbb{R}$, $(x_n)_n$ and $(y_n)_n$ converging sequences with $x_n\to x$ and $y_n\to y$.
            \begin{enumerate}[label={\upshape(\roman*)}]
                \item $x_n+y_n\to x+y$
                \item $x_ny_n\to xy$
                \item If $\forall n, y_n\neq0$ and $y\neq0$, then $\frac{1}{y_n}\to\frac{1}{y}$
                \item If $\forall n, y_n\neq0$ and $y\neq0$, then $\frac{x_n}{y_n}\to\frac{x}{y}$
                \item $ax_n\to ax$
                \item If $\forall n, x_n\leq y_n$, then $x\leq y$.
            \end{enumerate}
        \end{theorem}
        \begin{proof}
            For all parts of this proof, begin by letting $\varepsilon>0$.
            \begin{enumerate}[label={\upshape(\roman*)}]
                \item Since $x_n\to x,\ \exists N_1\in\mathbb{N}$ such that $n\geq N_1\implies\abs{x_n-x}<\frac{\varepsilon}{2}$ and since $y_n\to y,\ \exists N_2\in\mathbb{N}$ such that $n\geq N_2\implies\abs{y_n-y}<\frac{\varepsilon}{2}$. Choose $N=\max\{N_1, N_2\}$, now for $n\geq N$ we have
                \begin{align*}
                    \abs{(x_n+y_n)-(x+y)} &= \abs{(x_n-x)+(y_n-y)}\\
                    &\leq\abs{x_n-x}+\abs{y_n-y}\\
                    &<\frac{\varepsilon}{2}+\frac{\varepsilon}{2}=\varepsilon.
                \end{align*}
                \item Since $(x_n)_n$ is bounded, $\exists M\geq0$ such that $\abs{x_n}\leq M\ \forall n\in\mathbb{N}$. Since $x_n\to x,\ \exists N_1\in\mathbb{N}$ such that $n\geq N_1\implies\abs{x_n-x}<\frac{\varepsilon}{2M}$ and since $y_n\to y,\ \exists N_2\in\mathbb{N}$ such that $n\geq N_2\implies\abs{y_n-y}<\frac{\varepsilon}{2\abs{y}}$. Now, choosing $N=\max\{N_1, N_2\}$, for $n\geq N$ we have
                \begin{align*}
                    \abs{x_ny_n-xy}&=\abs{x_ny_n-x_ny+x_ny-xy}\\
                    &=\abs{x_n(y_n-y)+y(x_n-x)}\\
                    &\leq\abs{x_n}\abs{y_n-y}+\abs{y}\abs{x_n-x}\\
                    &<\abs{x_n}\frac{\varepsilon}{2M}+\abs{y}\frac{\varepsilon}{2\abs{y}}\\
                    &\leq M\frac{\varepsilon}{2M}+\abs{y}\frac{\varepsilon}{2\abs{y}}=\frac{\varepsilon}{2}+\frac{\varepsilon}{2}=\varepsilon.
                \end{align*}
                \item Since $y_n\to y,\ \exists N_1\in\mathbb{N}$ such that $n\geq N_1\implies\abs{y_n-y}<\frac{\abs{y}}{2}$. Then for $n\geq N_1$
                \begin{equation*}
                    \abs{y}=\abs{y_n+(y-y_n)}\leq\abs{y_n}+\abs{y_n-y}<\abs{y_n}+\frac{\abs{y}}{2},
                \end{equation*}
                so $\abs{y_n}>\frac{\abs{y}}{2}$. Now, let $N_2$ be such that $n\geq N_2\implies\abs{y_n-y}<\frac{\varepsilon\abs{y}^2}{2}$, then for $n\geq\max\{N_1, N_2\}$ we have
                \begin{align*}
                    \abs*{\frac{1}{y_n}-\frac{1}{y}}&=\abs*{\frac{y-y_n}{y_ny}}=\frac{\abs{y_n-y}}{\abs{y_n}\abs{y}}\\
                    &<\frac{\varepsilon\abs{y}^2}{2\abs{y_n}\abs{y}}<\frac{2\varepsilon\abs{y}^2}{2\abs{y}^2}=\varepsilon.
                \end{align*}
                \item Write $\frac{x_n}{y_n}$ as $x_n\cdot\frac{1}{y_n}$, then the result follows from parts (ii) and (iii).
                \item Let $y_n=a$ (so $(y_n)_n$ is the constant sequence $(a, a, \dots)$ with limit $a$), then the result follows from part (ii).
                \item We will make use of a minor result to assist in proving this.
                \begin{lemma}
                    Suppose $(a_n)_n$ is a convergent sequence with $a_n\to a$. If $a_n>0\ \forall n\in\mathbb{N}$ then $a\geq0$.
                \end{lemma}
                \begin{proof}
                    Suppose by way of contradiction that $a<0$ and let $\varepsilon=-a>0$. Then since $(a_n)_n$ converges, $\exists N\in\mathbb{N}$ such that $n\geq N\implies\abs{a_n-a}<-a$. This means that
                    \begin{align*}
                        a<\, &a_n-a<-a\\
                        \implies &a_n-a+a<-a+a\\
                        \implies &a_n<0
                    \end{align*}
                    which is a contradiction, hence $a$ must be greater than 0.
                \end{proof}
                Now, consider $y_n-x_n\geq0\ \forall n\in\mathbb{N}$. The sequence $(y_n-x_n)_n$ converges to $y-x$ by part (i) (and part (v) with $a=-1$) and so by our claim $y-x\geq0$, which implies $x\leq y$ as required.
            \end{enumerate}
        \end{proof}
        A little observation from the last part of this theorem is the \textbf{sandwich principle}, which is that if we have three sequences --- $(a_n)_n$, $(x_n)_n$ and $(b_n)_n$ with limits $a$, $x$ and $b$ respectively --- with $a_n\leq x_n\leq b_n\ \forall n\in\mathbb{N}$, then $a\leq x\leq b$. In the special case that $a=b$, we can use this principle to determine the limit of an unknown sequence $(x_n)_n$ as in this case $a=x=b$. We will finish off this section with a couple of examples about how to use the arithmetic properties of limits.
        \begin{example}
            Let $(x_n)_n=\left(1+\frac{1}{n^p}\right)_n$ with $p\in\mathbb{N}$. Show that $x_n\to1$ as $n\to\infty$.\\
            First, let $a_n=1$ and $b_n=\frac{1}{n^p}$, then note that we already know that the sequence $\left(\frac{1}{n}\right)_n$ converges to 0, so by theorem \ref{seq-lim-props} (ii) (applied repeatedly),
            \begin{equation*}
                \underbrace{\frac{1}{n}\cdot\frac{1}{n}\cdots\frac{1}{n}}_{p\text{ times}}=\frac{1}{n^p}\to0\text{ as }n\to\infty.
            \end{equation*}
            Then $(x_n)_n=(a_n+b_n)_n$ which, by theorem \ref{seq-lim-props} (i), converges to 1 since $a_n\to1$ and $b_n\to0$ as $n\to\infty$.
        \end{example}
        \begin{example}
            Show that the sequence $(x_n)_n=\left(\frac{n+5n^2+2n^3}{n^3+3}\right)$ converges to 2.\\
            \begin{equation*}
                \frac{2n^3+5n^2+n}{n^3+3}=\frac{2+5\left(\frac{1}{n}\right)+\left(\frac{1}{n}\right)^2}{1+3\left(\frac{1}{n}\right)^3}\to\frac{2+5(0)+0}{1+3(0)}=2\text{ as }n\to\infty.
            \end{equation*}
        \end{example}

    \section{Monotone Sequences}
        \paragraph{}
        So far, in all the examples of sequences we have seen so far we have had to know what the limit of the sequence is in order to prove the existence of it, but sometimes this is tricky to know beforehand if we are dealing with a novel sequence. We will now introduce some tools and theorems that will enable us to prove the existence of a limit for a sequence without knowing what the limit is.
        \begin{definition}
            Let $(x_n)_n$ be a sequence of real numbers.
            \begin{itemize}
                \item $(x_n)_n$ is \textbf{increasing} if $x_1\leq x_2\leq x_3\leq\dots$
                \item $(x_n)_n$ is \textbf{decreasing} if $x_1\geq x_2\geq x_3\geq\dots$
                \item $(x_n)_n$ is \textbf{monotone} if it is either increasing or decreasing.
            \end{itemize}
        \end{definition}
        \begin{example}
            Here are some examples of sequences with their characterisation by this definition.
            \begin{itemize}
                \item $(a_n)_n=(n)_n$ is increasing.
                \item $(b_n)_n=\left(\frac{1}{n}\right)_n$ is decreasing.
                \item $(c_n)_n=(p, p, p,\dots),\ p\in\mathbb{R}$ is both increasing and decreasing.
                \item $(d_n)_n=(-2n)_n$ is decreasing.
                \item $(e_n)_n=((-1)^n)_n$ is not monotone.
                \item $(f_n)_n=\left(\frac{(-1)^n}{n}\right)_n$ is not monotone.
            \end{itemize}
        \end{example}
        We can make an important observation from these examples. The bounded sequences ($(b_n)_n, (c_n)_n, (e_n)_n, (f_n)_n$) that are also monotone ($(b_n)_n, (c_n)_n$) are all convergent, whereas none of the unbounded sequences are convergent (which we would expect by theorem \ref{cvg-seq-bounded}). Although a sequence does not have to be monotone to converge (look at the last example above), it is a general result that if a monotone sequence is bounded then it does converge, which we will now prove.
        \begin{theorem}[Monotone Convergent Theorem (MCT)]\label{MCT}
            Let $(x_n)_n$ be a monotone sequence. Then $(x_n)_n$ is convergent if and only if $(x_n)_n$ is bounded.
        \end{theorem}
        \begin{proof}
            $(\implies)$: This follows from theorem \ref{cvg-seq-bounded}.\\
            $(\impliedby)$: Suppose $(x_n)_n$ is bounded and without meaningful loss of generality assume $(x_n)_n$ is increasing. Recall that since $\mathbb{R}$ is \textit{complete}, every bounded set of real numbers has a supremum that is also a real number, i.e. if we let
            \begin{equation*}
                X\coloneqq\{x_n:n\in\mathbb{N}\}
            \end{equation*}
            then $X$ is a bounded subset of the real numbers and we can let $x=\sup X\in\mathbb{R}$. Now we want to show that $(x_n)_n$ converges to $x$. Let $\varepsilon>0$ and note that $x-\varepsilon$ is not an upper bound since $x$ is the supremum. Therefore, there exists $N\in\mathbb{N}$ such that $x_N\geq x-\varepsilon$. Now since $(x_n)_n$ is increasing, for $n\geq N$ we have
            \begin{equation*}
                \abs{x_n-x}=x-x_n\leq x-x_N\leq\varepsilon,
            \end{equation*}
            and hence $x_n\to x$ as $n\to\infty$.
        \end{proof}
        This theorem makes intuitive sense, if we have a sequence which is always increasing or decreasing and is bounded then it must reach a limit before it surpasses the bound, and in fact this limit is the least upper/greatest lower bound of the set of elements of the sequence.
        \begin{corollary}
            Suppose $(x_n)_n$ is a bounded sequence, then
            \begin{enumerate}[label={\upshape(\roman*)}]
                \item If $(x_n)_n$ is increasing, $\lim_{n\to\infty}x_n=\sup_n x_n$.
                \item If $(x_n)_n$ is decreasing, $\lim_{n\to\infty}x_n=\inf_n x_n$.
            \end{enumerate}
        \end{corollary}
        \begin{proof}
            The proof follows directly from the proof of MCT.
        \end{proof}
        %TODO, fig here
        \begin{example}
            Let $(x_n)_n$ be a sequence given by
            \begin{equation*}
                x_n=\frac{1}{1^2}+\frac{1}{2^2}+\frac{1}{3^2}+\cdots+\frac{1}{n^2}=\sum_{k=1}^n\frac{1}{k^2}.
            \end{equation*}
            Show that this sequence is convergent.\\
            First notice that $(x_n)_n$ is increasing,
            \begin{equation*}
                x_n=\frac{1}{1^2}+\frac{1}{2^2}+\frac{1}{3^2}+\cdots+\frac{1}{n^2}<\frac{1}{1^2}+\frac{1}{2^2}+\frac{1}{3^2}+\cdots+\frac{1}{n^2}+\frac{1}{(n+1)^2}=x_{n+1},
            \end{equation*}
            and also the elements of $(x_n)_n$ are bounded between 0 and 2.
            \begin{align*}
                0&<\frac{1}{1^2}+\frac{1}{2^2}+\frac{1}{3^2}+\frac{1}{4^2}+\cdots+\frac{1}{n^2}\\
                &\leq1+\frac{1}{1\cdot2}+\frac{1}{2\cdot3}+\frac{1}{3\cdot4}+\cdots+\frac{1}{(n-1)n}\\
                &=1+\left(\frac{1}{1}-\frac{1}{2}\right)+\left(\frac{1}{2}-\frac{1}{3}\right)+\left(\frac{1}{3}-\frac{1}{4}\right)+\cdots+\left(\frac{1}{n-1}-\frac{1}{n}\right)\\
                &=1+1-\frac{1}{n}<2.
            \end{align*}
            Hence by MCT, this sequence must converge.
            In fact, it can be shown that this sequence converges to $\frac{\pi^2}{6}$, but the proof of that is a bit more involved.
        \end{example}

    \section{Subsequences}
        \paragraph{}
        Now we want to introduce the idea of a subsequence, which is a sequence contained within another sequence. We can do this by taking a regular old sequence of real numbers and `picking' certain entries out that we want to be part of the new subsequence, for example: every 2nd entry, every 5th entry, every entry whose index is a power of 10. We can do this by defining a new sequence of natural numbers, and then using this sequence to index our original one as follows.
        \begin{definition}
            Let $(x_n)_n$ be a sequence of real numbers and $(m_k)_k$ be a sequence of natural numbers which is strictly increasing $(m_1<m_2<m_3<\dots)$. Then $(x_{m_k})_k$ is a \textbf{subsequence} of $(x_n)_n$.
        \end{definition}
        So, for example, we could start with the sequence $(x_n)_n=\left(\frac{1}{n^2}\right)_n$ and take $m_k=2k$, then $x_{m_k}=x_{2k}=\frac{1}{(2k)^2}$. One important thing to note from this definition is that we \textit{always} have $m_k\geq k\ \forall k\in\mathbb{N}$, this will be a useful fact going forward. Now, how can we know if a subsequence converges? We can use the methods we have discovered already, but also it turns out that if a sequence is convergent then any subsequence of it will also converge to the same limit.
        \begin{theorem}
            Let $(x_n)_n$ be a convergent subsequence with $x_n\to x$ as $n\to\infty$. Then any subsequence $(x_{m_k})_k$ also has $x_{m_k}\to x$ as $k\to\infty$.
        \end{theorem}
        \begin{proof}
            Let $\varepsilon>0$. Since $x_n\to x,\ \exists N\in\mathbb{N}$ such that $n\geq N\implies\abs{x_n-x}<\varepsilon$. Since $(x_{m_k})_k$ is a subsequence of $(x_n)_n$, we can let $K\in\mathbb{N}$ such that for $k\geq K$ we have $m_k\geq N$ (for example, choose $K=N$ since $m_k\geq k\geq N$). Then we have
            \begin{equation*}
                \abs{x_{m_k}-x}<\varepsilon\ \forall k\geq K,
            \end{equation*}
            which means that the subsequence converges to $x$.
        \end{proof}
        \begin{theorem}[Monotone Subsequence Theorem]\label{MST}
            Any sequence of real numbers has a monotone subsequence.
        \end{theorem}
        \begin{proof}
            Let $(x_n)_n$ ba a sequence of real numbers. Before proving this theorem we make a quick definition. We say that a natural number $p$ is a \textit{peak} if $x_p\geq x_n\ \forall n\geq p$, i.e. it is the index of the greatest number in the sequence from that point onwards. Now, there are two cases for a general sequence: either it has infinitely many peaks, or it has finitely many.\\
            \textbf{Case 1}: $(x_n)_n$ has infinitely many peaks. List out the peaks in order of which they occur --- $p_1<p_2<p_3\dots$ --- and notice that by definition we have
            \begin{equation*}
                x_{p_1}\geq x_{p_2}\geq x_{p_3}\geq\dots,
            \end{equation*}
            and thus $(x_{p_k})_k$ is a decreasing subsequence (which is monotone) as required.\\
            \textbf{Case 2}: $(x_n)_n$ has finitely many peaks. Again we can list them in order: $p_1<p_2<p_3<\dots<p_N$. Now let $t_1>p_N$ be a natural number. Since $t_1$ is not a peak by definition, there exists $t_2>t_1$ such that $x_{t_2}>x_{t_1}$. Since $t_2$ is also not a peak, there exists $t_3>t_2$ such that $x_{t_3}>x_{t_2}$. Continuing in this way we generate a sequence $t_1<t_2<t_3<\dots$ with $x_{t_1}<x_{t_2}<x_{t_3}<\dots$, hence $(x_{t_k})_k$ is an increasing subsequence (which is monotone) as required.
        \end{proof}
        Notice that if we wanted to create an increasing subsequence from a sequence with infinitely many peaks, or a decreasing subsequence from one with finitely many, we need only switch round the definition of a peak to that of a `trough' by flipping the order relation and also flipping all the order relations in the body of the proof. Thus no generality is lost. We will now prove one of the major results of real analysis using all of the tools we have gathered so far.
        \begin{theorem}[Bolzano-Weierstrass Theorem]\label{BWT}
            Any \textit{bounded} sequence of real numbers has a \textit{convergent} subsequence.
        \end{theorem}
        \begin{proof}
            Let $(x_n)_n$ be any bounded sequence of real numbers. Then by the Monotone Subsequence Theorem (\ref{MST}), $(x_n)_n$ has a monotone subsequence $(x_{m_k})_k$. But now this subsequence is both monotone and bounded, so by the Monotone Convergence Theorem (\ref{MCT}), this subsequence is convergent.
        \end{proof}
        A simple example of this theorem in action would be the sequence $((-1)^n)_n$, which is bounded but does not converge, however the subsequence $((-1)^{2k})_k=(1)_k$ does converge.
        %TODO add axis translation example of subsequence

    \section{Cauchy Sequences}
        \paragraph{}
        Our goal was to develop tools that allow us to determine if a sequence is convergent without knowing what the limit of sequence is. We know that if a sequence is monotone and bounded then it is convergent, and that if a sequence is  a subsequence of a bounded sequence, then it is convergent. But what if a sequence does not fulfill these properties? Our current definition of convergence requires knowledge of the limit in order to prove that a sequence is convergent, so how about we create a new definition of a convergent sequence where we don't need to know what the limit is. The intuition for this defintion is that we want to express mathematically that for a sequence to converge the terms have to get closer and closer together as $n\to\infty$. We will name this new type of convergent sequence \textit{Cauchy} after the 19th-century French mathematician Augustin-Louis Cauchy.
        \begin{definition}
            Let $(x_n)_n$ be a sequence of real numbers. Then $(x_n)_n$ is a \textbf{Cauchy sequence} if
            \begin{equation*}
                \forall\varepsilon>0,\ \exists N\in\mathbb{N}\ :\ n,m\geq N\implies\abs{x_m-x_n}\leq\varepsilon.
            \end{equation*}
        \end{definition}
        What this defintion is encoding for us is that for any small $\varepsilon$, there exists a step in the sequence such that from that step onwards, \textbf{all terms} of the sequence are $\varepsilon$-close to each other. It is actually quite important that we specify `all terms' as we want this definition to be as close as possible to our original definition of convergence and if we say that, for example, only consecutive terms must get $\varepsilon$-close to each other then we admit some sequences which we would not consider convergent under the original definition. For example, the sequence $(x_n)_n=(\sqrt{n})_n$. Successive terms get arbitrarily close to each other ($x_{n+1}-x_n=\sqrt{n+1}-\sqrt{n}=\frac{1}{\sqrt{n+1}+\sqrt{n}}<\frac{1}{2\sqrt{n}}$) but not \textit{all others} (the terms still get arbitrarily large as $n\to\infty$ so for any index $n$ and distance $M$ we can always find a large enough index $m$ such that $x_m-x_n>M$), and thus the sequence is not Cauchy.
        \begin{example}
            Show that $\left(\frac{1}{n}\right)_n$ is a Cauchy sequence.\\
            Let $\varepsilon>0$ and choose $N>\frac{2}{\varepsilon}$. Then for $m,n\geq N$ we have
            \begin{equation*}
                \abs{x_m-x_n}=\abs*{\frac{1}{m}-\frac{1}{n}}\leq\frac{1}{m}+\frac{1}{n}\leq\frac{1}{N}+\frac{1}{N}=\frac{2}{N}<\varepsilon.
            \end{equation*}
        \end{example}
        \begin{example}
            Show that $\left(\frac{n}{1+n}\right)_n$ is a Cauchy sequence.\\
            Let $\varepsilon>0$ and choose $N>\frac{1}{\varepsilon}$. Then for $m\geq n\geq N$ we have
            \begin{align*}
                \abs{x_m-x_n}&=\abs*{\frac{m}{1+m}-\frac{n}{1+n}}=\abs*{\frac{m-n}{(1+m)(1+n)}}\\
                &\leq\frac{m-n}{mn}=\frac{1}{n}-\frac{1}{m}\leq\frac{1}{n}\leq\frac{1}{N}<\varepsilon.
            \end{align*}
        \end{example}
        \begin{example}
            Show that the sequence given by $x_n=1+\frac{1}{2}+\dots+\frac{1}{n}$ is not Cauchy.\\
            To prove this, we will show that the sequence fulfills the negation of the Cauchy sequence definition ($\exists\varepsilon>0$ such that $\forall N\in\mathbb{N}\ \exists m,n\geq N$ such that $\abs{x_m-x_n}>\varepsilon$).\\
            Choose $\varepsilon<\frac{1}{2}$, let $N\in\mathbb{N}$, and choose $n=N$ and $m=2N$. Then
            \begin{align*}
                &\abs{x_m-x_n}=\abs{x_{2N}-x_N}\\
                &=\abs*{\left(1+\frac{1}{2}+\dots+\frac{1}{N}+\frac{1}{N+1}+\dots+\frac{1}{2N}\right)-\left(1+\frac{1}{2}+\dots+\frac{1}{N}\right)}\\
                &=\frac{1}{N+1}+\dots+\frac{1}{2N}\geq\frac{1}{2N}+\dots+\frac{1}{2N}=N\frac{1}{2N}=\frac{1}{2}>\varepsilon.
            \end{align*}
        \end{example}
        Notice that in these examples, the sequences that are Cauchy are also convergent, and the sequence that is not Cauchy is not convergent. This correspondence turns out to be general and in fact it is an important result that the criteria for being convergent and being Cauchy are equivalent.
        \begin{theorem}
            Let $(x_n)_n$ be a sequence of real numbers. Then $(x_n)_n$ is a convergent sequence if and only if it is also a Cauchy sequence.
        \end{theorem}
        \begin{proof}\\
            $(\implies)$: We will make use of a lemma.
            \begin{lemma}
                Any Cauchy sequence is bounded.
            \end{lemma}
            \begin{proof}
                Let $(x_n)_n$ be a Cauchy sequence. Then $\exists N\in\mathbb{N}$ such that $n\geq N\implies\abs{x_n-x_N}<1$ (this is equivalent to the definition of Cauchy with $m=N$). Then
                \begin{equation*}
                    \abs{x_n}=\abs{x_n-x_N+x_N}\leq\abs{x_n-x_N}+\abs{x_N}\leq 1+\abs{x_N}.
                \end{equation*}
                Therefore, for
                \begin{equation*}
                    M=\max\{1+\abs{x_N},\abs{x_1},\abs{x_2},\dots,\abs{x_{N-1}}\},
                \end{equation*}
                and thus for \textit{all} $n\in\mathbb{N}$ we have $\abs{x_n}\leq M$, so $(x_n)_n$ is bounded.
            \end{proof}
            Now suppose $(x_n)_n$ is a Cauchy sequence, then by the lemma above, $(x_n)_n$ is bounded and thus by the Bolzano-Weierstrass Theorem (\ref{BWT}) there exists a convergent subsequence $(x_{m_k})_k$ which converges to some limit $x$. We now want to show that the Cauchy sequence $(x_n)_n$ also converges to $x$. Let $\varepsilon>0$, Since $(x_n)_n$ is Cauchy $\exists N\in\mathbb{N}$ such that $m,n\geq N\implies\abs{x_m-x_n}<\frac{\varepsilon}{2}$. Also, since $x_{m_k}\to x,\ \exists K\in\mathbb{N}$ such that $k\geq K\implies\abs{x_{m_k}-x}<\frac{\varepsilon}{2}$. Then, for $n\geq\max\{N,K\}$ we have
            \begin{equation*}
                \abs{x_n-x}=\abs{x_n-x_{m_n}+x_{m_n}-x}\leq\abs{x_n-x_{m_k}}+\abs{x_{m_k}-x}<\frac{\varepsilon}{2}+\frac{\varepsilon}{2}=\varepsilon,
            \end{equation*}
            and thus $(x_n)_n$ is convergent with limit $x$.\\
            $(\impliedby)$: Suppose $(x_n)_n$ is a convergent sequence with limit $x$. Let $\varepsilon>0$, then $\exists N\in\mathbb{N}$ such that $n\geq N\implies\abs{x_n-x}<\frac{\varepsilon}{2}$. For $m,n\geq N$ we have
            \begin{equation*}
                \abs{x_m-x_n}=\abs{x_m-x+x-x_n}\leq\abs{x_m-x}+\abs{x-x_n}<\frac{\varepsilon}{2}+\frac{\varepsilon}{2}=\varepsilon,
            \end{equation*}
            and thus $(x_n)_n$ is a Cauchy sequence.
        \end{proof}

\end{document}
