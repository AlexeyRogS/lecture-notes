% !TeX root = ..\real_analysis.tex
\documentclass[../real_analysis.tex]{subfiles}

\begin{document}

    \section{What is Analysis?}
        Real analysis is the rigorous study of concepts that involve limits of infinite processes.
        Some examples of these that appear in all areas of mathematics are convergence of sequences or series, limits of functions, or derivatives and integrals.
        Until the 18th/19th century, our understanding of these concepts was based on intuition;
        such as the idea that a continuous function is one that can be drawn without removing the pen off the page, or that a differentiable function is one of which the graph has no sharp corners;
        and while these intuitions can take us very far, eventually the discovery of many `paradoxes' which contradict these intuitions lead to the creation of a new foundation for mathematics based on rigour and mathematical logic.

        An example of one of these contradictions comes from a Fourier series, which is an infinite summation of sine functions.
        It is pretty clear that the sine function is continuous, so therefore it would make sense that any superposition of sine functions is also continuous, right? It turns out that this is not the case.
        \begin{example}
            The function $S$ defined by:
            \begin{align}
                S(x) & = \sum_{n=1}^\infty\frac{(-1)^{(n+1)}}{n}\sin{nx}        \\
                    & = \sin{x}-\frac{1}{2}\sin{2x}+\frac{1}{3}\sin{3x}-\cdots
            \end{align}
            is not continuous. We can see this very clearly if we look at the graph.
        \end{example}
        % TODO: include graph

        In order to reconcile these apparent contradictions, we have to use a number system which is `complete' i.e. there are no gaps in the number line.
        It is a well-known fact that the rational numbers $\mathbb{Q}$ do not fulfill this requirement, as can be seen simply drawing a right-angled triangle with short sides of length 1.
        % TODO: add figure
        In this case the length of the hypotenuse, by the Pythagorean theorem, squares to 2, and it can be shown that this quantity cannot be expressed as a ratio of two integers.
        \begin{theorem}
            There is no rational number $x$ such that $x^2=2$.
        \end{theorem}
        \begin{proof}
            Suppose $x=\frac{p}{q}$, with $p,q\in\mathbb{N}$ and $\mathrm{gcd}(p, q)=1$ ($x$ is expressed as a rational number in lowest terms).
            Since $x^2=2$,
            \begin{equation}
                \frac{p^2}{q^2}=2\implies p^2=2q^2,
            \end{equation}
            so $p$ is an even number (a square can only be even if the number itself is even), thus $\exists k\in\mathbb{N}$ such that $p=2k$.
            Hence,
            \begin{equation}
                (2k)^2=2q^2\implies 2k^2=q^2,
            \end{equation}
            so $q$ is also an even number, but this is a contradiction of our assumption that $\mathrm{gcd}(p,q)=1$, and so the initial assumption that $x$ can be expressed as a rational number must be false.
        \end{proof}
        What this simple example shows is that to do analysis, we need to create a new \textit{complete} number system, the \textbf{real numbers} $\mathbb{R}$, but before we do that, we need to lay down a few definitions about sets.

\end{document}
