\documentclass[../real_analysis.tex]{subfiles}

\begin{document}

    \section{The Basic Definition of a Function}\label{sec:basic-definition-of-a-function}
        The concept of a function is one that is fundamental to all area of mathematics. We will be using functions thoroughly from now on so it's a good time to review exactly what a function is and how we can manipulate them.
        \begin{definition}
            Let $D$ and $R$ be non-empty sets. Then $f \subseteq D \times R$ is a function if $\forall x \in D, \forall y, z \in R,\ (x, y) \in f$ and $(x, z) \in f \implies y = z.$
        \end{definition}
        In English this says that a function is a binary relation on two sets, and what makes a function different from any old binary relation is the property that every $x \in D$ is paired with a \textit{unique} $y \in R$.
        \begin{definition}
            Let $f \subseteq D \times R$ be a function.
            Then we say $f: D \to R$ and denote the unique $y \in R$ associated with $x \in D$ as $f(x)$.
            \begin{itemize}
                \item $D$ is \textbf{domain} of the function, denoted $\dom(f)$.
                \item $R$ is the \textbf{range}\footnote{Some people use the term \textbf{range} in an ambiguous way. A less ambiguous term is \textbf{codomain}, which means the same thing as we have just defined for range.} of the function, denoted $\range(f)$.
                \item The \textbf{image} of $f$, $\im(f)$, is defined as ${\{f(x):x \in D\}}$, or more explicitly ${\{y \in R : \exists x \in D, f(x)=y\}}$. The image is a subset of the range, and may or may not be a proper subset.
                \item For a given $x$, $f(x)$ may be referred to as the \textbf{image} of $x$ under $f$, or the \textbf{value} of $f$ at $x$, or the \textbf{output} of $f$ for the \textbf{input} $x$.
            \end{itemize}
        \end{definition}
        To express what the image of some function $f$ actually is, we need some kind of formula for the image of an arbitrary point $x \in D$. This can be given by a straightforward formula such as
        \begin{equation}
            f(x) = 2x \quad \text{or}\quad x \mapsto x^2 + 1,
        \end{equation}
        or a piecewise definition like
        \begin{equation}
            f(x) = \begin{cases}
                        x & x \leq 1 \\
                        x^2 + 1 & x > 1,
                    \end{cases}
        \end{equation}
        or something even more abstract, such as a power series or recurrence relation (Recall that sequences are defined as functions).

        A real function is a function that has real numbers as its inputs and outputs. We can visualise a real functions behaviour by treating the ordered pairs $\big(x, f(x)\big) \in f$ as Cartesian coordinates in the plane, which is what we all know as a graph.
        \begin{example}
            Let $f: \mathbb{R} \to \mathbb{R}$ be given by $f(x) = \sqrt{x}$. \\
            This is not a function since $\sqrt{x}$ is not defined in $\mathbb{R}$ for $x<0$. Even if we restrict the domain of $f$ to $[0, \infty)$, it is still technically not a function since there are two possible values of $\sqrt{x}$ for all $x \in [0, \infty)$. We can fix this by taking the absolute value ($f: [0, \infty) \to \mathbb{R}$ given by $f(x) = \lvert\sqrt{x}\rvert$ is a function).
        \end{example}
        \begin{example}
            Let $f: \mathbb{R} \to \mathbb{R}$ be given by
            \[f(x) = \begin{cases}
                        1 & \text{if}\ x \in \mathbb{Q} \\
                        0 & \text{if}\ x \in \mathbb{R}\setminus\mathbb{Q}.
                    \end{cases}\]
            This is again a valid function, and it has numerical values, but drawing a convincing graph of this function is left as a challenge to the reader.
        \end{example}

    \section{Operations on Functions}\label{sec:operations-on-functions}
        Now we will define how to combine functions together to create new functions. Let $f, g$ be functions with domains and ranges in $\mathbb{R}$.
        \begin{definition}[Basic Arithmetic]
            Firstly, it is useful to have a working definition of equality.
            \begin{itemize}
                \item $f = g \iff D:= \dom(f) = \dom(g)\ \text{and}\ f(x) = g(x)\ \forall x \in D$.
            \end{itemize}
            For simple operations like the sum and product, we define the new domain simply as the common domain of the operands.
            \begin{itemize}
                \item $\dom(f+g) = \dom(f \cdot g) = \dom(f) \cap \dom(g)$.
                \item $(f+g)(x) = f(x) + g(x)$.
                \item $(f \cdot g)(x) = f(x)g(x)$.
            \end{itemize}
            In the special case of multiplying by a constant function $c$ given by $x \mapsto k, k \in \mathbb{R}$,
            \begin{itemize}
                \item $(c \cdot f)(x) = c(x)f(x) = kf(x)$.
            \end{itemize}
            Using this definition with the constant function $x \mapsto -1$, we can define the differences of functions.

            The reciprocal $1 / g$ is defined where $g$ is not zero.
            \begin{itemize}
                \item $\dom\left(\frac{1}{g}\right) = \{x \in \dom(g) : g(x) \ne 0\}$.
                \item $\left(\frac{1}{g}\right)(x) = \frac{1}{g(x)}$.
            \end{itemize}
            Using this we can define quotients of functions as a product of a function with a reciprocal.
        \end{definition}
        \begin{definition}[Composition]
            The composition $f \circ g: \dom(g) \to \range(f)$ is defined only where $\im(g)$ overlaps with $\dom(f)$, so we say:
            \begin{itemize}
                \item $\dom(f \circ g) = \{x \in \dom(g) : g(x) \in \dom(f)\}$.
                \item $(f \circ g)(x) = f\big(g(x)\big)$.
            \end{itemize}
        \end{definition}

    \section{Classes of Functions}\label{sec:classes-of-functions}
        There are certain properties of functions that we can use to group functions together into similarly-behaving types. Let $f$ be a function with domain and range in $\mathbb{R}$.
        \begin{definition}
            \begin{itemize} \item[]
                \item $f$ is an \textbf{even} function $\iff f(x) = f(-x)\ \forall x \in \dom(f)$. This can be visualised as symmetry about the y-axis of the graph.
                \item $f$ is an \textbf{odd} function $\iff -f(x) = f(-x)\ \forall x \in \dom(f)$. This can be visualised as $180^\circ$ rotational symmetry about the origin of the graph.
            \end{itemize}
            It is possible for a function to be neither even nor odd.
        \end{definition}
        \begin{definition}
            \begin{itemize} \item[]
                \item $f$ is an \textbf{injection} (or \textbf{one-to-one}) $\iff$
                \[\forall x, x^\prime \in \dom(f),\ {f(x) = f(x^\prime) \implies x = x^\prime}.\] Every element in the range is the image of \textit{at most one} element from the domain.
                \item $f$ is a \textbf{surjection} (or \textbf{onto}) $\iff$
                \[\forall y \in \range(f),\ \exists x \in \dom(f)\  \text{s.t.}\ y = f(x).\]
                Every element in the range is reached by \textit{at least one} element from the domain.
                \item f is a \textbf{bijection} (or \textbf{one-to-one correspondence}) $\iff f$ is \textbf{injective} and $f$ is \textbf{surjective}. Each element of the range is mapped to by \textit{exactly one} element from the domain.
            \end{itemize}
            It is possible for a function to be neither  injective nor surjective.
        \end{definition}

    \section{Limits}\label{sec:limits}
        % left limits, right limits
        % move boundedness of function to this section

    \section{Continuity}\label{sec:continuity}
        Very informally, a continuous function is one whose graph can be drawn without lifting the pen of the page. Intuitively this means that a small change in the input of the function leads to a small change to the output. To write a rigorous definition, we will use what we have learned from sequences.
        \begin{definition}
            Let $f: D \to R$, then $f$ is \textbf{continuous at} $x_0\in D$ if is \textbf{continuous at} $x_0\in D$ if
            \begin{equation}
                \forall\varepsilon>0,\ \exists\delta>0\text{ such that }\forall x\in D,\ \abs{x-x_0}\leq\delta\implies\abs{f(x)-f(x_0)}\leq\varepsilon.
            \end{equation}
            We say $f$ is \textbf{continuous} if $f$ is continuous at all $x_0\in D$.
        \end{definition}
        What this means is that for a fixed point $x_0$ and \textit{any} value of $\varepsilon$ we choose, we can find a $\delta$ such that when the inputs of the function are $\delta$-close, the outputs of the function are $\varepsilon$-close. If this is true for all points in the functions domain, then we call the whole function continuous.
        % TODO: insert a diagram to help here
        When we are doing a proof of continuity, we must first fix $x_0$, then $\varepsilon$, then we may choose $\delta$ similarly to the way we chose $N$ when proving convergence of a sequence. Let's do an example.
        \begin{example}
            Let $f: \RR\to\RR$ be given by $f(x)=2x$.
            Claim: $f$ is continuous.\\
            Let $x_0\in\RR$, let $\varepsilon>0$, choose $\delta\in(0,\frac{\varepsilon}{2})$.\\
            Consider $x\in\RR$ with $\abs{x-x_0}<\delta$, then
            \begin{align}
                \abs{f(x)-f(x_0)}&=\abs{2x-2x_0}\\
                &=2\abs{x-x_0}\\
                &<2\delta\\
                &<\varepsilon.
            \end{align}
            Since the value of $x_0$ was arbitrary, $f$ is continuous.
        \end{example}
        
        \subsection{Sequential Definition of Continuity}\label{subsec:sequential-definition-of-continuity}
            Just like when proving the convergence of a sequence, when we are first constructing a proof of continuity the value of $\delta$ is unknown until the proof is complete. Once the value is known, we go back and place it at the start so that the proof works.
            % TODO: thomae's function
            It turns out that there is a very strong link between the continuous functions and convergent sequences that we will now prove.
            \begin{theorem}[Sequential Definition of Continuity]\label{thm:seq-cont}
                Let $D\subseteq\RR$, let $f: D\to\RR$, then the following are equivalent:
                \begin{enumerate}[label={\upshape(\roman*)}]
                    \item $f$ is continuous at $x_0\in D$.
                    \item If $(x_n)_n$ is a sequence in $D$ with limit $x_0$, then the sequence $(f(x_n))_n$ in $\RR$ has limit $f(x_0)$.
                \end{enumerate}
            \end{theorem}
            \begin{proof}\\
                $(\implies)$: Suppose $f$ is continuous at $x_0$, $(x_n)_n$ is a sequence in $D$ with limit $x_0$.\\
                Let $\varepsilon>0$, since $f$ is continuous at $x_0$, $\exists\delta>0$ such that $\abs{x-x_0}<\delta\implies\abs{f(x)-f(x_0)}<\varepsilon$.\\
                Now, since $x_n\to x_0$, $\exists N\in\NN$ such that $n\geq N\implies\abs{x_n-x_0}<\delta$, therefore $\abs{f(x_n)-f(x_0)}<\varepsilon$.\\
                $(\impliedby)$: Suppose by way of contradiction that (ii) is true, but (i) is false. So $\exists\varepsilon>0$ such that $\forall\delta>0, \exists x\in D$ where $\abs{x-x_0}<\delta\implies\abs{f(x)-f(x_0)}\geq\varepsilon$ (negation of continuity).\\
                Choose $\varepsilon$ such that the above is satisfied, then for $n\in\NN$, let $\delta=\frac{1}{n}$ and $x_n\in D$ such that $\abs{x_n-x_0}<\frac{1}{n}$ but $\abs{f(x_n)-f(x_0)}\geq\varepsilon$.\\
                Thus $x_n\to x_0$, but $f(x_n)\not\to f(x_0)$ since $\varepsilon$ is fixed, which is a contradiction of our assumptions that (ii) is true, hence $f$ must be continuous at $x_0$.
            \end{proof}
            What this theorem is telling us is that we now have two equivalent definitions of continuity; the $\epsilon-\delta$ definition which we defined originally and the sequential definition, ``$\forall(x_n)_n\in D$ such that $x_n\to x_0$ we have $f(x_n)\to f(x_0)$''. Which one we will want to use depends on the situation. Let's look at an example where using the sequential definition makes proving continuity much easier.
            \begin{example}
                Let $f: \RR\to\RR$ be given by $f(x)=2x^2+1$. Claim: $f$ is continuous.\\
                \underline{Using $\varepsilon-\delta$ definition of continuity}:\\
                Let $x_0\in\RR$, let $\varepsilon>0$, choose $\delta<\min\left\{1, \frac{\varepsilon}{4\abs{x_0}+2}\right\}$.\\
                Consider $x\in\RR$ such that $\abs{x-x_0}<\delta$, then
                \begin{align}
                    \abs{f(x)-f(x_0)}&=\abs{2x^2+1-(2x_0^2+1)}\\
                    &=2\abs{x^2-x_0^2}\\
                    &=2\abs{(x-x_0)(x+x_0)}\\
                    &<2\delta\abs{x+x_0}\\
                    &<2\delta(\abs{x}+\abs{x_0})\\
                    &<2\delta(2\abs{x_0}+1)\tag{provided $\delta<1$ so $\abs{x}<\abs{x_0}+1$}\\
                    &<\varepsilon.\tag{provided $\delta<\frac{\varepsilon}{4\abs{x_0}+2}$}
                \end{align}
                \underline{Using the sequential definition of continuity}:\\
                Let $x_0\in\RR$, let $(x_n)_n$ be a sequence in $\RR$ such that $x_n\to x_0$. Then
                \begin{equation}
                    f(x_n)=2x_n^2+1\to2x_0^2+1=f(x_0).\tag{by theorem \ref{thm:seq-lim-props}}
                \end{equation}
                Therefore $f$ is continuous.
            \end{example}
            Here are a couple of examples proving discontinuity using the sequential definition.
            \begin{example}
                Let $f: \RR\to\RR$ be given by $f(x)=\begin{cases}
                    -1 & x<0\\
                    1 & x\geq0.
                \end{cases}$ Claim: $f$ is discontinuous at $x_0=0$.\\
                Let $(x_n)_n=\left(-\frac{1}{n}\right)_n$ with $x_n\to0$. Since $x_n<0$, $f(x_n)=-1$. But $f(0)=1\neq-1$, so $f$ cannot be continuous at $x=0$.
            \end{example}
            \begin{example}
                Let $f: \RR\to\RR$ be given by $f(x)=\begin{cases}
                    x & x\in\QQ\\
                    0 & x\in\RR\setminus\QQ.
                \end{cases}$ Claim: $f$ is continuous only at $x=0$.
                % TODO: prove
            \end{example}

        \subsection{Operations on Continuous Functions}\label{subsec:operations-on-continuous-functions}
            In these styles of proof, the arguments can often be greatly simplified by breaking down the given function into smaller building blocks and using previous results. Recall that this was one of the main benefits of the comparison test for proving the convergence of a series. In order to use this technique, we need to know if the continuity of functions is preserved under the operations we can do with them. Thankfully, they are.
            \begin{theorem}\label{thm:cts-func-props}
                Let $D\subseteq\RR$, $x_0\in D$, $f, g: D\to\RR$ two functions continuous at $x_0$, $\lambda\in\RR$.
                \begin{enumerate}[label={\upshape(\roman*)}]
                    \item $(f+g): D\to\RR$ is continuous at $x_0$.
                    \item $(fg): D\to\RR$ is continuous at $x_0$.
                    \item $(\lambda f): D\to\RR$ is continuous at $x_0$.
                    \item $\min\{f, g\}: D\to\RR$ is continuous at $x_0$.
                    \item Similarly for $\max\{f, g\}$.
                    \item $\abs{f}: D\to\RR$ is continuous at $x_0$.
                    \item If $g(x)\neq0$ $\forall x\in D$, $\left(\frac{f}{g}\right): D\to\RR$ is continuous at $x_0$.
                \end{enumerate}
            \end{theorem}
            \begin{proof}\\
                \begin{itemize}
                    \item[(i)] \underline{Using the $\varepsilon-\delta$ definition}:\\
                    Let $\varepsilon>0$, since $f$ is continuous at $x_0$, $\exists\delta_f>0$ so that $\abs{x-x_0}<\delta_f\implies\abs{f(x)-f(x_0)}<\frac{\varepsilon}{2}$. Similarly for $g$, $\exists\delta_g$.\\
                    Now choose $\delta=\min\{\delta_f,\delta_g\}$, then
                    \begin{align}
                        \abs{x-x_0}<\delta\implies\abs{(f+g)(x)-(f+g)(x_0)}&=\abs{f(x)+g(x)-(f(x_0)+g(x_0))}\\
                        &=\abs{f(x)-f(x_0)+g(x)-g(x_0)}\\
                        &\leq\abs{f(x)-f(x_0)}+\abs{g(x)-g(x_0)}\\
                        &\leq\frac{\varepsilon}{2}+\frac{\varepsilon}{2}\\
                        &=\varepsilon.
                    \end{align}
                    \underline{Using the sequential definition}:\\
                    Consider a sequence $(x_n)_n$ in $D$ with limit $x_0$. Since $f$ and $g$ are continuous at $x_0$, $f(x_n)\to f(x_0)$ and $g(x_n)\to g(x_0)$.\\
                    Then $(f+g)(x_n)=f(x_n)+g(x_n)\to f(x_0)+g(x_0)=(f+g)(x_0)$.
                    \item (ii), (iii), (vi), and (vii) follow similarly.
                    \item To prove (iv) and (v), note that
                    \begin{equation}
                        \min\{f,g\}=\frac{f+g-\abs{f-g}}{2},\quad \max\{f,g\}=\frac{f+g+\abs{f-g}}{2},
                    \end{equation}
                    then use the results of (i), (iii), and (vi).
                \end{itemize}
            \end{proof}
            A notable consequence of this theorem is that since $f: \RR\to\RR$ defined by $f(x)=x$ is a continuous function (the proof is trivial), any polynomial is a continuous function.
            \begin{theorem}
                Let $D_f, D_g\subseteq\RR$, $f: D_f\to\RR$ with $f(D_f)\subseteq(D_g)$, continuous at $x_0\in D_f$, $g: D_g\to\RR$ continuous at $f(x_0)\in D_g$. Then the composition $(g\circ f):D_f\to\RR$ is continuous at $x_0$.
            \end{theorem}
            \begin{proof}
                Let $(x_n)_n\in D_f$ with limit $x_0$. Since $f$ is continuous at $x_0$, $f(x_n)\to f(x_0)$. Since $g$ is continuous at $f(x_0)$,
                \begin{equation}
                    (g\circ f)(x_n)=g(f(x_n))\to g(f(x_0))=(g\circ f)(x_0).
                \end{equation}
            \end{proof}

        \subsection{The Intermediate Value Theorem}\label{subsec:intermediate-value-theorem}
            We will now move onto the meat of this chapter, some intuitive but nonetheless very important results about continuous functions which will be very important when we get into more advanced topics.
            \begin{definition}
                A function $f: D\to\RR$ is \textbf{bounded} if $\exists M\in\RR$ such that
                \begin{equation}
                    \forall x\in D,\ \abs{f(x)}\leq M.
                \end{equation}
            \end{definition}
            The first theorem we will see is so simple that it is not clear why we need to prove it. It states that a continuous function on a closed interval is bounded and has a maximum and minimum value. As we will see after proving the theorem, there are discontinuous functions which satisfy these properties but there are also many which don't.
            \begin{theorem}[Extreme Value Theorem]\label{thm:evt}
                Given two real numbers $a, b$ with $a<b$, let $f: [a, b]\to\RR$ be a continuous function. Then
                \begin{enumerate}[label={\upshape(\roman*)}]
                    \item $f$ is bounded.
                    \item $f$ attains a maximum, i.e. $\exists x_0\in[a,b]$ such that $f(x)\leq f(x_0)\ \forall x\in[a,b]$.
                    \item $f$ attains a minimum, i.e. $\exists x_0\in[a,b]$ such that $f(x)\geq f(x_0)\ \forall x\in[a,b]$.
                \end{enumerate}
            \end{theorem}
            \begin{proof}\\
                \begin{enumerate}[label={\upshape(\roman*)}]
                    \item Assume by way of contradiction that $f$ is not bounded. Then let $(x_n)_n$ be a sequence in $[a,b]$ such that $\forall n\in\NN$ we have $\abs{f(x_n)}>n$.\\
                    $(x_n)_n$ is a bounded sequence (all the terms are contained within an interval), so by the Bolzano-Weierstrass theorem (\ref{thm:BWT}), there exists a convergent subsequence $(x_{n_k})_k\in[a,b]$ with limit $x_0\in[a,b]$.\\
                    Then since $f$ is continuous, $f(x_{n_k})\to f(x_0)$, hence $(f(x_{n_k}))_k$ is a bounded sequence by theorem \ref{thm:cvg-seq-bounded}, but this is a contradiction since $\abs{f(x_{n_k})}>n_k\ \forall k$, so $f$ must be bounded.
                    \item Let $M=\sup\{f(x):\ x\in[a,b]\}$. $M$ exists since all bounded sets in $\RR$ have suprema and we know the set is bounded by (i).\\
                    By definition of $\sup$, $\forall n\in\NN\ \exists x_n\in[a,b]$ such that $M-\frac{1}{n}\leq f(x_n)\leq M$, which implies that $f(x_n)\to M$ as $n\to\infty$ (however, $(x_n)_n$ is by no means necessarily convergent, so we are not done yet).\\
                    By the Bolzano-Weierstrass theorem (\ref{thm:BWT}), there exists a convergent subequence $(x_{n_k})_k\in[a,b]$ with limit $x_0\in[a,b]$, and since $f$ is continuous, $f(x_{n_k})\to f(x_0)$.\\
                    Therefore since $(x_{n_k})_k$ is a subsequence of $(x_n)_n$, $f(x_0)=M$.
                \end{enumerate}
                The proof of (iii) follows similarly from (ii).
            \end{proof}
            % TODO: example of step function satisfying the theorem and 1/x not satisfying
            The next theorem is the most important result for continuous functions. It cements the intuitive notion that continuous functions have no gaps or jumps. The proof uses the completeness property of the real numbers, which ensures that there are no infinitesimal gaps in the function.
            % TODO: include diagram
            \begin{theorem}[Intermediate Value Theorem]\label{thm:ivt}
                Let $I\subseteq\RR$ be an interval, $f: I\to\RR$ a continuous function. Let $a,b\in I$ with $a<b$ and let $y\in\RR$ lie between $f(a)$ and $f(b)$ (i.e. $f(a)<y<f(b)$ or $f(b)<y<f(a)$). Then $\exists x\in(a,b)$ such that
                \begin{equation}
                    f(x)=y.
                \end{equation}
            \end{theorem}
            \begin{proof}
                Assume without meaningful loss of generality that $f(a)<y<f(b)$. Let $S=\{x\in[a,b]:\ f(x)<y\}$ (since $a\in S$, $S\neq\emptyset$). Let $x_0=\sup S$.\\
                Claim: $f(x_0)=y$.
                \begin{enumerate}[label={\upshape(\roman*)}]
                    \item $\forall n\in\NN\ x_0-\frac{1}{n}<x_0$, so $x_0-\frac{1}{n}$ is not an upper bound for $S$. Hence $\exists s_n\in S$ such that $x_0-\frac{1}{n}<s_n<x_0$. Then $s_n\to x_0$ as $n\to\infty$ and therefore since $f$ is continuous $f(s_n)\to f(x_0)$.\\
                    Since $s_n\in S$, $f(s_n)<y$, therefore $f(x_0)\leq y$.
                    \item It can be shown similarly that $f(x_0)\geq y$, so $f(x_0)=y$.
                \end{enumerate}
                Note that since $f(a)<f(x_0)<f(b)$, $x_0\in(a,b)$.
            \end{proof}
            % TODO: include diagram
            It should be noted that although it may seem as if this theorem may be taken to be the definition of a continuous function, the converse of the theorem is not true.
            % TODO: counterexample sin(1/x)
            \begin{example}
                Let $f:[0,1]\to[0,1]$ be continuous. Then $f$ has a fixed point ($\exists x\in[0,1]$ such that $f(x)=x$).\\
                \textit{Proof}. Let $g$ be given by $g(x)=f(x)-x$. Note that by theorem \ref{thm:cts-func-props}, $g$ is continuous on $[0,1]$. Also note that
                \begin{align}
                    g(0)&=f(0)-0=f(0)\geq0\\
                    g(1)&=f(1)-1\leq1-1=0
                \end{align}
                So $0\in[g(1),g(0)]$. If $g(0)=0$ or $g(1)=0$, then $0$ or $1$ are fixed points. Therefore, assume that $g(0)>0>g(1)$, and hence by the intermediate value theorem (\ref{thm:ivt}), $\exists x_0\in(0,1)$ so that $g(x_0)=0$, i.e. $f(x_0)-x_0=0$, so $x_0$ is a fixed point.
            \end{example}
            \begin{corollary}
                Let $I\subseteq\RR$ be an interval, $f: I\to\RR$ a continuous function. Then the range of $f$, $f(I)$, is either an interval or a single point.
            \end{corollary}
            \begin{proof}
                Suppose there are two or more distinct points in $f(I)$, let $y\in(\inf f(I),\sup f(I))$. By the intermediate value theorem (\ref{thm:ivt}), $\exists x\in I$ such that $f(x)=y$, hence $f(I)$ is an interval.
            \end{proof}
            % TODO: uniform continuity, section 3.6 in Howie
\end{document}
