\documentclass[../quantum_mechanics.tex]{subfiles}

\begin{document}

    \section{Introduction to the Quantum Harmonic Oscillator}\label{sec:introduction-to-the-quantum-harmonic-oscillator}
        The square well has introduced to the key concepts of quantum mechanical systems.
        However, it is not the most realistic model in the world, although it has some uses.
        In this chapter, we will study the quantum analogue of the simple harmonic oscillator from classical mechanics, which is a very useful quantum mechanical model.

        \subsection{The Importance of the Harmonic Oscillator}\label{subsec:importance-of-the-harmonic-oscillator}
            It is not an understatement to say that the simple harmonic oscillator is quite possibly the most important model in classical mechanics.
            Why is this?

            The prototypical example of a system exhibiting simple harmonic motion is a mass on a spring, for which the force acting on the mass is given by Hooke's law $F=-kx$.
            This leads to sinusoidal motion with frequency $\omega=\sqrt{\frac{k}{m}}$.

            The reason why Hooke's law works so well in so many scenarios is that it is always a good approximation around a stable equilibrium point.
            To see why this is, we take the Taylor expansion of a potential energy function (it does not matter what shape it is) about a minimum located at $x_0$,
            \begin{equation}
                V(x)=V(x_0)+\dv{V(x_0)}{x}(x-x_0)+\frac{1}{2}\dv[2]{V(x_0)}{x}(x-x_0)^2+\dots.
            \end{equation}
            The term with the first derivative vanishes since $x_0$ is a minimum, so for $x$ close to $x_0$, the potential is approximately given by
            \begin{equation}
                V(x)\approx V(x_0)+\frac{1}{2}\dv[2]{V(x_0)}{x}(x-x_0)^2.
            \end{equation}
            This gives rise to a force with the same form as Hooke's law:
            \begin{equation}
                F=-\dv{V(x)}{x}=-\dv[2]{V(x_0)}{x}(x-x_0).
            \end{equation}

            If we apply this to situations in quantum mechanics, we could approximate the behaviour of, for example, vibrations in diatomic molecules that consist of two bound atoms such as hydrogen chloride.
            These interatomic forces can be modelled classically using the Lennard-Jones potential, but at the equilibrium point we can approximate the potential as parabolic.
            This means that the quantum harmonic oscillator will be a good approximation of the ground state and the first few excited states of this system!
            % TODO: include a diagram of this

        \subsection{Setting up the Quantum Harmonic Oscillator}\label{subsec:setting-up-the-quantum-harmonic-oscillator}
            For the classical harmonic oscillator, the potential energy is
            \begin{equation}
                V(x)=\frac{1}{2}kx^2=\frac{1}{2}m\omega^2x^2.
            \end{equation}
            For the quantum harmonic oscillator, we will write the potential energy with the natural frequency $\omega$ because it is more helpful than referring to an abstract ``spring constant'' $k$.
            
            In quantum mechanics, our position variable $x$ gets promoted to an operator $\hat{x}$, so the potential energy operator for the quantum harmonic oscillator would be
            \begin{equation}
                \hat{V}(x)=\frac{1}{2}m\omega^2\hat{x}^2.
            \end{equation}
            The Hamiltonian operator for the system is then
            \begin{equation}
                \hat{H}=\frac{\hat{p}^2}{2m}+\frac{1}{2}m\omega^2\hat{x}^2,
            \end{equation}
            which, when written in the position basis, becomes
            \begin{equation}
                \hat{H}=-\frac{\hbar^2}{2m}\pdv[2]{}{x}+\frac{1}{2}m\omega^2x^2.
            \end{equation}
            The time-dependent Schrodinger equation is therefore
            \begin{equation}
                -\frac{\hbar^2}{2m}\pdv[2]{\psi(x,t)}{x}+\frac{1}{2}m\omega^2x^2\psi(x,t)=i\hbar\pdv{\psi(x,t)}{t}.
            \end{equation}
            
            Just like in the last chapter, we can separate the time-dependence out to get the time-dependent Schrodinger equation for the QHO:
            \begin{equation}\label{eq:qho-tise}
                -\frac{\hbar^2}{2m}\dv[2]{u(x)}{x}+\frac{1}{2}m\omega^2x^2u(x)=Eu(x),
            \end{equation}
            where the full wavefunction is then given by $\psi(x,t)=u(x)e^{-\frac{iEt}{\hbar}}$.
            % TODO: mention that this is Weber's equation

    \section{Solving the Quantum Harmonic Oscillator}\label{sec:solving-the-quantum-harmonic-oscillator}
        We will now go through all the stages of solving equation~\ref{eq:qho-tise} to find the energy eigenstates.
        
        \subsection{Changing the Dimensionless Quantities}\label{sec:solving-the-quantum-harmonic-oscillator:subsec:changing-to-dimensionless-quantities}
            The first step is change our variable from $x$ to some dimensionless variable $y$.
            This will simplify the notation somewhat, which makes it easier to see what is going on.

            We have three physical constants, $\hbar$, $m$, and $\omega$ in equation~\ref{eq:qho-tise}.
            These have units of \unit{\kilogram\meter\squared\per\second}, \unit{\kilogram}, and \unit{\per\second} respectively.
            If we combine these in the form $\frac{\hbar}{m\omega}$, this has units of \unit{\meter\squared}.
            Therefore, if we define
            \begin{equation}
                x_0=\sqrt{\frac{\hbar}{m\omega}},
            \end{equation}
            this is a natural length scale for the problem, so we can define our new dimensionless length parameter $y$ as
            \begin{equation}
                y=\frac{x}{x_0}=\sqrt{\frac{m\omega}{\hbar}}x.
            \end{equation}
            Note that $x_0$ has nothing to do with equilibrium position.
            For a given oscillator, this is a constant.
            Substituting $x=yx_0$ into equation~\ref{eq:qho-tise} and expanding the definition of $x_0$, we get
            \begin{align}
                -\frac{\hbar^2}{2m}\frac{1}{x_0^2}\dv[2]{u}{y}+\frac{1}{2}m\omega^2x_0^2y^2u&=Eu\\
                -\frac{\hbar^2}{2m}\frac{m\omega}{\hbar}\dv[2]{u}{y}+\frac{1}{2}m\omega^2\frac{\hbar}{m\omega}y^2u&=Eu\\
                -\frac{\hbar\omega}{2}\left(\dv[2]{u}{y}-y^2u\right)&=Eu\\
                \dv[2]{u}{y}-y^2u&=-\varepsilon u,\label{eq:qho-tise-dimensionless}
            \end{align}
            where we have defined
            \begin{equation}
                \varepsilon=\frac{2E}{\hbar\omega},
            \end{equation}
            as the dimensionless energy parameter.

        \subsection{Simplifying Using Asymptotic Analysis}\label{sec:solving-the-quantum-harmonic-oscillator:subsec:simplifying-using-asymptotic-analysis}
            As $y\to\pm\infty$, we can neglect $\varepsilon u$ compared to $y^2u$ as the former will be negligible.
            Equation~\ref{eq:qho-tise-dimensionless} then becomes
            \begin{equation}
                \dv[2]{u}{y}=y^2u.
            \end{equation}
            If we guess a solution of the form $Ae^{-\frac{y^2}{2}}$, then we get
            \begin{align}
                \dv{}{y}Ae^{-\frac{y^2}{2}}&=-Aye^{-\frac{y^2}{2}}\\
                \dv[2]{}{y}Ae^{-\frac{y^2}{2}}&=-Ae^{-\frac{y^2}{2}}+Ay^2e^{-\frac{y^2}{2}}\approx Ay^2e^{-\frac{y^2}{2}},
            \end{align}
            where in the last part we have neglected terms that are small when $y\to\pm\infty$.
            Therefore, this function is a solution to the TISE for large $y$.
            Note that $Be^{+\frac{y^2}{2}}$ is also a valid solution, but is does not have a finite limit as $y\to\pm\infty$ and is therefore not normalisable, so we will throw it out.

            So the behaviour of the full solution to the TISE must be Gaussian for large $y$, but what about the rest of the function?
            We now suppose that it takes the form
            \begin{equation}
                u(y)=h(y)e^{-\frac{y^2}{2}},
            \end{equation}
            and substitute this into equation~\ref{eq:qho-tise-dimensionless} to see what we get.

            Taking derivatives, we find
            \begin{align}
                \dv{u}{y}&=\dv{h}{y}e^{-\frac{y^2}{2}}-yhe^{-\frac{y^2}{2}}\\
                \dv[2]{u}{y}&=\dv[2]{h}{y}e^{-\frac{y^2}{2}}-y\dv{h}{y}e^{-\frac{y^2}{2}}-he^{-\frac{y^2}{2}}-y\dv{h}{y}e^{-\frac{y^2}{2}}+y^2he^{-\frac{y^2}{2}}\\
                &=\dv[2]{h}{y}e^{-\frac{y^2}{2}}-2y\dv{h}{y}e^{-\frac{y^2}{2}}-he^{-\frac{y^2}{2}}+y^2he^{-\frac{y^2}{2}}.
            \end{align}
            Substituting these in, we get
            \begin{align}
                &\dv[2]{h}{y}e^{-\frac{y^2}{2}}-2y\dv{h}{y}e^{-\frac{y^2}{2}}-he^{-\frac{y^2}{2}}+y^2he^{-\frac{y^2}{2}}+(\varepsilon-y^2)he^{-\frac{y^2}{2}}=0\\
                &\dv[2]{h}{y}e^{-\frac{y^2}{2}}-2y\dv{h}{y}e^{-\frac{y^2}{2}}+(\varepsilon-1)he^{-\frac{y^2}{2}}=0\\
                &\dv[2]{h}{y}-2y\dv{h}{y}+(\varepsilon-1)h=0\label{eq:qho-tise-h}.
            \end{align}
            If we solve this differential equation for $h(y)$, we get the full solution for equation~\ref{eq:qho-tise-dimensionless}.
            Luckily, this equation is a well-known equation in mathematics called ``Hermite's Equation'' (with $2\lambda=\varepsilon-1$)!
            We will now go through the solution via a series method.
        
        \subsection{Solving for $h(y)$ via a Series Expansion}\label{sec:solving-the-quantum-harmonic-oscillator:subsec:solving-for-h(y)-via-a-series-expansion}
            Assuming that $h(y)$ can be written as an infinite polynomial (a power series), we can substitute the following into equation~\ref{eq:qho-tise-h}:
            \begin{equation}
                h(y)=\sum_{n=0}^\infty a_ny^n.
            \end{equation}
            What we find is a recurrence relation for the coefficients:
            \begin{equation}
                \frac{a_{j+2}}{a_j}=\frac{2j+1-\varepsilon}{(j+2)(j+1)}.
            \end{equation}

            The limiting behaviour for the ratio of the subsequent coefficients is
            \begin{equation}
                \lim_{j\to\infty}\frac{a_{j+2}}{a_j}=\frac{2}{j},
            \end{equation}
            which diverges more quickly than the Gaussian converges.
            % TODO: show this in more detail please
            This means that for the solutions to be valid wavefunctions, meaning for them to be normalisable, the power series must terminate.
            For each solution with a series terminating at $j=1$, $j=2$, $j=3$, etc., we obtain one valid eigenstate.

            For some $j$ to be the highest non-vanishing coefficient, we must have the numerator of the recurrence relation be zero, i.e.
            \begin{equation}
                2j+1-\varepsilon=0.
            \end{equation}
            If we substitute back in $\varepsilon=\frac{2E}{\hbar\omega}$, this gives us the energy eigenvalues of the quantum harmonic oscillator:
            \begin{equation}
                E_n=\hbar\omega\left(n+\frac{1}{2}\right),
            \end{equation}
            where we have replaced $j$ with $n$.
            Note that in contrast to the infinite square well, the label for the energy eigenvalues starts at 0 instead of 1.
            This is just a matter of preference for how the equations look and doesn't mean anything physically.

            The ground state energy is $E_0=\frac{1}{2}\hbar\omega$, the first excited state is $E_1=\frac{3}{2}\hbar\omega$, and so on.
            The energy levels are equally spaced, increasing in single units of $\hbar\omega$.

            So what are the polynomials $h_n(y)$?
            To determine the coefficients, we need $a_0$ to calculate all the even coefficients and $a_1$ to calculate all the odd coefficients.
            Both the even and odd coefficients must terminate for the whole series to terminate, but the value of $E$ only allows us to terminate one or the other.
            Therefore, valid eigenstates will either have $a_0=0$ and only odd powers of $x$ or $a_1=0$ and only even power of $x$ in $h(y)$.

            For the ground state, only $a_0\neq 0$, so $h(y)=a_0$ and the ground state eigenfunction is
            \begin{equation}
                u_0(y)=a_0e^{-\frac{y^2}{2}}.
            \end{equation}
            For the first excited state, only $a_1\neq 0$, so $h(y)=a_1y$ and the eigenfunction is therefore
            \begin{equation}
                u_1(y)=a_1ye^{-\frac{y^2}{2}}.
            \end{equation}
            For the second excited state, $a_0$ and $a_2$ are nonzero so $h(y)=a_0+a_2x^2$.
            The dimensionless energy has the value
            \begin{equation}
                \varepsilon=\frac{2E}{\hbar\omega}=\frac{2\times\frac{5}{2}\hbar\omega}{\hbar\omega}=5,
            \end{equation}
            so the recurrence relation gives
            \begin{equation}
                \frac{a_2}{a_0}=\frac{2(0)+1-5}{(0+2)(0+1)}=-\frac{4}{2}=-2.
            \end{equation}
            Hence the eigenfunction takes the form
            \begin{equation}
                u_2(y)=a_0(1-2y^2)e^{-\frac{y^2}{2}}.
            \end{equation}
            The process carries on like this.
            The polynomials generated are called \textbf{Hermite polynomials}, and denoted $h_n(y)$.
            Note that it is conventional to introduce a minus sign to the normalisation constant $a_0$ or $a_1$ so that the leading-order terms in the polynomial have a positive sign.
            The first few Hermite polynomials are
            \begin{align}
                h_0(y)&=1\\
                h_1(y)&=2y\\
                h_2(y)&=4y^2-2\\
                h_3(y)&=8y^3-12y\\
                h_4(y)&=16y^4-48y^2+12\\
                h_5(y)&=32y^5-160y^3+120y.
            \end{align}
            These coefficients are slightly different to the ones calculated above, but it doesn't matter since we have to normalise the whole wavefunction anyway.

    \section{Energy Eigenstates of the QHO}\label{chap:quantum-harmonic-oscillator:sec:energy-eigenstates}
        There is one final step to writing out the eigenstates, which is to return to physical units and remove the dimensionless length $y$.
        Once we do this, the energy eigenstates (including time dependence) have the form:
        \begin{equation}
            \psi_n(x,t)=N_nh_n\left(\sqrt{\frac{m\omega}{\hbar}}x\right)e^{-\frac{m\omega}{2\hbar}x^2}e^{-i\omega t\left(n+\frac{1}{2}\right)}.
        \end{equation}
        The normalisation constant $N_n$ turns out to be
        \begin{equation}
            N_n=\frac{1}{\sqrt{2^nn!}}\sqrt[4]{\frac{m\omega}{\pi\hbar}}.
        \end{equation}
        % TODO: check and calculate this
        
        The spatial part of the first few eigenstates is
        \begin{align}
            u_0(x)&=N_0e^{-\frac{m\omega}{2\hbar}x^2}\\
            u_1(x)&=N_1xe^{-\frac{m\omega}{2\hbar}x^2}\\
            u_2(x)&=N_2\left(\frac{2m\omega}{\hbar}x^2-1\right)e^{-\frac{m\omega}{2\hbar}x^2}\\
            u_3(x)&=N_3\left(8\sqrt{\left(\frac{m\omega}{\hbar}\right)^3}y^3-12\sqrt{\frac{m\omega}{\hbar}}y\right)e^{-\frac{m\omega}{2\hbar}x^2}\\
        \end{align}
        % TODO: calculate all normalisations and plot
            
        \subsection{Comparison of the Quantum to Classical Harmonic Oscillator}\label{subsec:comparison-of-the-quantum-to-classical-harmonic-oscillator}
            Here are the probability density of the first few eigenstates, with the classical probability density of a simple harmonic oscillator with energy $E=\hbar\omega\left(n+\frac{1}{2}\right)$ overlayed.
            We can see that as $n$ increases, the spatial probability density of the oscillator is more spread out, just like how the amplitude of them classical turning points increase for higher energy.
            The quantum probability density is actually nonzero beyond the classical turning point, so we have some tunnelling, but it decays exponentially in this region.
            Also note that the wavelength of $\psi_n$ is shortest at the center of the well, which is where a classical particle would be travelling the fastest.
            % http://tinyurl.com/qu-SHO

        \subsection{Parity}\label{sec:energy-eigenstates:subsec:parity}
            Recall that we saw in section~\ref{sec:solving-the-quantum-harmonic-oscillator:subsec:solving-for-h(y)-via-a-series-expansion} that the Hermite polynomials are either even or odd.
            This property is passed on to the eigenstates as well.
            We can investigate this symmetry by defining an operator which will detect whether a given wavefunction is odd or even.
            
            \begin{definition}
                The \textbf{parity} operator switches $x$ to $-x$, and can be defined by its action on a wavefunction as
                \begin{equation}
                    \hat{P}\psi(x,t)=\psi(-x,t).
                \end{equation}
            \end{definition}

            What are the possible eigenvalues of parity?
            If we apply parity twice, we get
            \begin{equation}
                \hat{P}^2\psi(x)=\hat{P}\hat{P}\psi(x)=\hat{P}\psi(-x)=\psi(x),
            \end{equation}
            so \textit{all} wavefunctions are eigenfunctions of $\hat{P}^2$ with eigenvalue $+1$.
            Using this fact, we can show that the eigenvalues of $\hat{P}$ are $\pm 1$.
            Suppose that $\psi(x)$ is an eigenfunction of $\hat{P}$ with eigenvalue $P$, then
            \begin{equation}
                \hat{P}^2\psi(x)=\hat{P}P\psi(x)=P^2\psi(x),
            \end{equation}
            implies that $P^2=1$, hence $P=\pm 1$.
            If a wavefunction is an eigenfunction of parity with $\hat{P}\psi(x)=\psi(x)$, then this means that $\psi(x)$ is an even function (symmetric about $x=0$) since this implies $\psi(-x)=\psi(x)$.
            Likewise, if we have $\hat{P}\psi(x)=-\psi(x)$, then $\psi(x)$ must be odd (antisymmetric about $x=0$) since this implies $\psi(x)=-\psi(-x)$.
            Not all wavefunctions are even or odd, therefore not all wavefunctions are eigenfunctions of parity.

            Is parity Hermitian?
            We can test this by calculating the adjoint, note that the integral has to be from $-\infty$ to $\infty$ for the harmonic oscillator.
            \begin{align}
                \langle\hat{P}^\dagger\rangle&=\int_{-\infty}^\infty(\hat{P}\psi(x))^\ast\psi(x)\dd{x}\\
                &=\int_{-\infty}^\infty\psi^\ast(-x)\psi(x)\dd{x},
            \end{align}
            now we make a substitution $y=-x$ to get
            \begin{align}
                \langle\hat{P}^\dagger\rangle&=-\int_\infty^{-\infty}\psi^\ast(y)\psi(-y)\dd{y}\\
                &=\int_{-\infty}^\infty\psi^\ast(y)\hat{P}\psi(y)\dd{y}\\
                &=\langle\hat{P}\rangle,
            \end{align}
            hence $\hat{P}$ is Hermitian and therefore it is observable.
            % TODO: talk about how we can observe parity

        \subsection{Superposition States}\label{chap:quantum-harmonic-oscillator:sec:energy-eigenstates:subsec:superposition-states}
            We can write superposition states just like we can for the infinite square well, using an expansion of energy eigenfunctions.
            \begin{equation}
                \psi(x,t)=\sum_{n=0}^\infty c_nu_n(x)e^{-\frac{iE_nt}{\hbar}}.
            \end{equation}
            The only difference between this formula and the one for the infinite square well is that the index $n$ starts from 0 instead of 1.
            % TODO: talk about coherent states

\end{document}
