\documentclass[../quantum_mechanics.tex]{subfiles}

\begin{document}

    \section{Introduction to the Quantum Harmonic Oscillator}\label{sec:introduction-to-the-quantum-harmonic-oscillator}
        The square well has introduced to the key concepts of quantum mechanical systems.
        However, it is not the most realistic model in the world, although it has some uses.
        In this chapter, we will study the quantum analogue of the simple harmonic oscillator from classical mechanics, which is a very useful quantum mechanical model.

        \subsection{The Importance of the Harmonic Oscillator}\label{subsec:importance-of-the-harmonic-oscillator}
            It is not an understatement to say that the simple harmonic oscillator is quite possibly the most important model in classical mechanics.
            Why is this?

            The prototypical example of a system exhibiting simple harmonic motion is a mass on a spring, for which the force acting on the mass is given by Hooke's law $F=-kx$.
            This leads to sinusoidal motion with frequency $\omega=\sqrt{\frac{k}{m}}$.

            The reason why Hooke's law works so well in so many scenarios is that it is always a good approximation around a stable equilibrium point.
            To see why this is, we take the Taylor expansion of a potential energy function (it does not matter what shape it is) about a minimum located at $x_0$,
            \begin{equation}
                V(x)=V(x_0)+\dv{V(x_0)}{x}(x-x_0)+\frac{1}{2}\dv[2]{V(x_0)}{x}(x-x_0)^2+\dots.
            \end{equation}
            The term with the first derivative vanishes since $x_0$ is a minimum, so for $x$ close to $x_0$, the potential is approximately given by
            \begin{equation}
                V(x)\approx V(x_0)+\frac{1}{2}\dv[2]{V(x_0)}{x}(x-x_0)^2.
            \end{equation}
            This gives rise to a force with the same form as Hooke's law:
            \begin{equation}
                F=-\dv{V(x)}{x}=-\dv[2]{V(x_0)}{x}(x-x_0).
            \end{equation}

            If we apply this to situations in quantum mechanics, we could approximate the behaviour of, for example, vibrations in diatomic molecules that consist of two bound atoms such as hydrogen chloride.
            These interatomic forces can be modelled classically using the Lennard-Jones potential, but at the equilibrium point we can approximate the potential as parabolic.
            This means that the quantum harmonic oscillator will be a good approximation of the ground state and the first few excited states of this system!
            % TODO: include a diagram of this

        \subsection{Setting up the Quantum Harmonic Oscillator}\label{subsec:setting-up-the-quantum-harmonic-oscillator}
            For the classical harmonic oscillator, the potential energy is
            \begin{equation}
                V(x)=\frac{1}{2}kx^2=\frac{1}{2}m\omega^2x^2.
            \end{equation}
            For the quantum harmonic oscillator, we will write the potential energy with the natural frequency $\omega$ because it is more helpful than referring to an abstract ``spring constant'' $k$.
            
            In quantum mechanics, our position variable $x$ gets promoted to an operator $\hat{x}$, so the potential energy operator for the quantum harmonic oscillator would be
            \begin{equation}
                \hat{V}(x)=\frac{1}{2}m\omega^2\hat{x}^2.
            \end{equation}
            The Hamiltonian operator for the system is then
            \begin{equation}\label{eq:qho-hamiltonian}
                \hat{H}=\frac{\hat{p}^2}{2m}+\frac{1}{2}m\omega^2\hat{x}^2,
            \end{equation}
            which, when written in the position basis, becomes
            \begin{equation}
                \hat{H}=-\frac{\hbar^2}{2m}\pdv[2]{}{x}+\frac{1}{2}m\omega^2x^2.
            \end{equation}
            The time-dependent Schrodinger equation is therefore
            \begin{equation}
                -\frac{\hbar^2}{2m}\pdv[2]{\psi(x,t)}{x}+\frac{1}{2}m\omega^2x^2\psi(x,t)=i\hbar\pdv{\psi(x,t)}{t}.
            \end{equation}
            
            Just like in the last chapter, we can separate the time-dependence out to get the time-dependent Schrodinger equation for the QHO:
            \begin{equation}\label{eq:qho-tise}
                -\frac{\hbar^2}{2m}\dv[2]{u(x)}{x}+\frac{1}{2}m\omega^2x^2u(x)=Eu(x),
            \end{equation}
            where the full wavefunction is then given by $\psi(x,t)=u(x)e^{-\frac{iEt}{\hbar}}$.
            % TODO: mention that this is Weber's equation

    \section{Solving the Quantum Harmonic Oscillator}\label{sec:solving-the-quantum-harmonic-oscillator}
        We will now go through all the stages of solving equation~\ref{eq:qho-tise} to find the energy eigenstates.
        
        \subsection{Changing the Dimensionless Quantities}\label{sec:solving-the-quantum-harmonic-oscillator:subsec:changing-to-dimensionless-quantities}
            The first step is change our variable from $x$ to some dimensionless variable $y$.
            This will simplify the notation somewhat, which makes it easier to see what is going on.

            We have three physical constants, $\hbar$, $m$, and $\omega$ in equation~\ref{eq:qho-tise}.
            These have units of \unit{\kilogram\meter\squared\per\second}, \unit{\kilogram}, and \unit{\per\second} respectively.
            If we combine these in the form $\frac{\hbar}{m\omega}$, this has units of \unit{\meter\squared}.
            Therefore, if we define
            \begin{equation}
                x_0=\sqrt{\frac{\hbar}{m\omega}},
            \end{equation}
            this is a natural length scale for the problem, so we can define our new dimensionless length parameter $y$ as
            \begin{equation}
                y=\frac{x}{x_0}=\sqrt{\frac{m\omega}{\hbar}}x.
            \end{equation}
            Note that $x_0$ has nothing to do with equilibrium position.
            For a given oscillator, this is a constant.
            Substituting $x=yx_0$ into equation~\ref{eq:qho-tise} and expanding the definition of $x_0$, we get
            \begin{align}
                -\frac{\hbar^2}{2m}\frac{1}{x_0^2}\dv[2]{u}{y}+\frac{1}{2}m\omega^2x_0^2y^2u&=Eu\\
                -\frac{\hbar^2}{2m}\frac{m\omega}{\hbar}\dv[2]{u}{y}+\frac{1}{2}m\omega^2\frac{\hbar}{m\omega}y^2u&=Eu\\
                -\frac{\hbar\omega}{2}\left(\dv[2]{u}{y}-y^2u\right)&=Eu\\
                \dv[2]{u}{y}-y^2u&=-\varepsilon u,\label{eq:qho-tise-dimensionless}
            \end{align}
            where we have defined
            \begin{equation}
                \varepsilon=\frac{2E}{\hbar\omega},
            \end{equation}
            as the dimensionless energy parameter.

        \subsection{Simplifying Using Asymptotic Analysis}\label{sec:solving-the-quantum-harmonic-oscillator:subsec:simplifying-using-asymptotic-analysis}
            As $y\to\pm\infty$, we can neglect $\varepsilon u$ compared to $y^2u$ as the former will be negligible.
            Equation~\ref{eq:qho-tise-dimensionless} then becomes
            \begin{equation}
                \dv[2]{u}{y}=y^2u.
            \end{equation}
            If we guess a solution of the form $Ae^{-\frac{y^2}{2}}$, then we get
            \begin{align}
                \dv{}{y}Ae^{-\frac{y^2}{2}}&=-Aye^{-\frac{y^2}{2}}\\
                \dv[2]{}{y}Ae^{-\frac{y^2}{2}}&=-Ae^{-\frac{y^2}{2}}+Ay^2e^{-\frac{y^2}{2}}\approx Ay^2e^{-\frac{y^2}{2}},
            \end{align}
            where in the last part we have neglected terms that are small when $y\to\pm\infty$.
            Therefore, this function is a solution to the TISE for large $y$.
            Note that $Be^{+\frac{y^2}{2}}$ is also a valid solution, but is does not have a finite limit as $y\to\pm\infty$ and is therefore not normalisable, so we will throw it out.

            So the behaviour of the full solution to the TISE must be Gaussian for large $y$, but what about the rest of the function?
            We now suppose that it takes the form
            \begin{equation}
                u(y)=h(y)e^{-\frac{y^2}{2}},
            \end{equation}
            and substitute this into equation~\ref{eq:qho-tise-dimensionless} to see what we get.

            Taking derivatives, we find
            \begin{align}
                \dv{u}{y}&=\dv{h}{y}e^{-\frac{y^2}{2}}-yhe^{-\frac{y^2}{2}}\\
                \dv[2]{u}{y}&=\dv[2]{h}{y}e^{-\frac{y^2}{2}}-y\dv{h}{y}e^{-\frac{y^2}{2}}-he^{-\frac{y^2}{2}}-y\dv{h}{y}e^{-\frac{y^2}{2}}+y^2he^{-\frac{y^2}{2}}\\
                &=\dv[2]{h}{y}e^{-\frac{y^2}{2}}-2y\dv{h}{y}e^{-\frac{y^2}{2}}-he^{-\frac{y^2}{2}}+y^2he^{-\frac{y^2}{2}}.
            \end{align}
            Substituting these in, we get
            \begin{align}
                &\dv[2]{h}{y}e^{-\frac{y^2}{2}}-2y\dv{h}{y}e^{-\frac{y^2}{2}}-he^{-\frac{y^2}{2}}+y^2he^{-\frac{y^2}{2}}+(\varepsilon-y^2)he^{-\frac{y^2}{2}}=0\\
                &\dv[2]{h}{y}e^{-\frac{y^2}{2}}-2y\dv{h}{y}e^{-\frac{y^2}{2}}+(\varepsilon-1)he^{-\frac{y^2}{2}}=0\\
                &\dv[2]{h}{y}-2y\dv{h}{y}+(\varepsilon-1)h=0\label{eq:qho-tise-h}.
            \end{align}
            If we solve this differential equation for $h(y)$, we get the full solution for equation~\ref{eq:qho-tise-dimensionless}.
            Luckily, this equation is a well-known equation in mathematics called ``Hermite's Equation'' (with $2\lambda=\varepsilon-1$)!
            We will now go through the solution via a series method.
        
        \subsection{Solving for $h(y)$ via a Series Expansion}\label{sec:solving-the-quantum-harmonic-oscillator:subsec:solving-for-h(y)-via-a-series-expansion}
            Assuming that $h(y)$ can be written as an infinite polynomial (a power series), we can substitute the following into equation~\ref{eq:qho-tise-h}:
            \begin{equation}
                h(y)=\sum_{n=0}^\infty a_ny^n.
            \end{equation}
            What we find is a recurrence relation for the coefficients:
            \begin{equation}
                \frac{a_{j+2}}{a_j}=\frac{2j+1-\varepsilon}{(j+2)(j+1)}.
            \end{equation}

            The limiting behaviour for the ratio of the subsequent coefficients is
            \begin{equation}
                \lim_{j\to\infty}\frac{a_{j+2}}{a_j}=\frac{2}{j},
            \end{equation}
            which diverges more quickly than the Gaussian converges.
            % TODO: show this in more detail please
            This means that for the solutions to be valid wavefunctions, meaning for them to be normalisable, the power series must terminate.
            For each solution with a series terminating at $j=1$, $j=2$, $j=3$, etc., we obtain one valid eigenstate.

            For some $j$ to be the highest non-vanishing coefficient, we must have the numerator of the recurrence relation be zero, i.e.
            \begin{equation}
                2j+1-\varepsilon=0.
            \end{equation}
            If we substitute back in $\varepsilon=\frac{2E}{\hbar\omega}$, this gives us the energy eigenvalues of the quantum harmonic oscillator:
            \begin{equation}\label{eq:qho-eigenvalues}
                E_n=\hbar\omega\left(n+\frac{1}{2}\right),
            \end{equation}
            where we have replaced $j$ with $n$.
            Note that in contrast to the infinite square well, the label for the energy eigenvalues starts at 0 instead of 1.
            This is just a matter of preference for how the equations look and doesn't mean anything physically.

            The ground state energy is $E_0=\frac{1}{2}\hbar\omega$, the first excited state is $E_1=\frac{3}{2}\hbar\omega$, and so on.
            The energy levels are equally spaced, increasing in single units of $\hbar\omega$.

            So what are the polynomials $h_n(y)$?
            To determine the coefficients, we need $a_0$ to calculate all the even coefficients and $a_1$ to calculate all the odd coefficients.
            Both the even and odd coefficients must terminate for the whole series to terminate, but the value of $E$ only allows us to terminate one or the other.
            Therefore, valid eigenstates will either have $a_0=0$ and only odd powers of $x$ or $a_1=0$ and only even power of $x$ in $h(y)$.

            For the ground state, only $a_0\neq 0$, so $h(y)=a_0$ and the ground state eigenfunction is
            \begin{equation}
                u_0(y)=a_0e^{-\frac{y^2}{2}}.
            \end{equation}
            For the first excited state, only $a_1\neq 0$, so $h(y)=a_1y$ and the eigenfunction is therefore
            \begin{equation}
                u_1(y)=a_1ye^{-\frac{y^2}{2}}.
            \end{equation}
            For the second excited state, $a_0$ and $a_2$ are nonzero so $h(y)=a_0+a_2x^2$.
            The dimensionless energy has the value
            \begin{equation}
                \varepsilon=\frac{2E}{\hbar\omega}=\frac{2\times\frac{5}{2}\hbar\omega}{\hbar\omega}=5,
            \end{equation}
            so the recurrence relation gives
            \begin{equation}
                \frac{a_2}{a_0}=\frac{2(0)+1-5}{(0+2)(0+1)}=-\frac{4}{2}=-2.
            \end{equation}
            Hence the eigenfunction takes the form
            \begin{equation}
                u_2(y)=a_0(1-2y^2)e^{-\frac{y^2}{2}}.
            \end{equation}
            The process carries on like this.
            The polynomials generated are called \textbf{Hermite polynomials}, and denoted $h_n(y)$.
            Note that it is conventional to introduce a minus sign to the normalisation constant $a_0$ or $a_1$ so that the leading-order terms in the polynomial have a positive sign.
            The first few Hermite polynomials are
            \begin{align}
                h_0(y)&=1\\
                h_1(y)&=2y\\
                h_2(y)&=4y^2-2\\
                h_3(y)&=8y^3-12y\\
                h_4(y)&=16y^4-48y^2+12\\
                h_5(y)&=32y^5-160y^3+120y.
            \end{align}
            These coefficients are slightly different to the ones calculated above, but it doesn't matter since we have to normalise the whole wavefunction anyway.

    \section{Energy Eigenstates of the QHO}\label{chap:quantum-harmonic-oscillator:sec:energy-eigenstates}
        There is one final step to writing out the eigenstates, which is to return to physical units and remove the dimensionless length $y$.
        Once we do this, the energy eigenstates (including time dependence) have the form:
        \begin{equation}
            \psi_n(x,t)=N_nh_n\left(\sqrt{\frac{m\omega}{\hbar}}x\right)e^{-\frac{m\omega}{2\hbar}x^2}e^{-i\omega t\left(n+\frac{1}{2}\right)}.
        \end{equation}
        The normalisation constant $N_n$ turns out to be
        \begin{equation}
            N_n=\frac{1}{\sqrt{2^nn!}}\sqrt[4]{\frac{m\omega}{\pi\hbar}}.
        \end{equation}
        % TODO: check and calculate this
        
        The spatial part of the first few eigenstates is
        \begin{align}
            u_0(x)&=N_0e^{-\frac{m\omega}{2\hbar}x^2}\\
            u_1(x)&=N_1xe^{-\frac{m\omega}{2\hbar}x^2}\label{eq:qho-first-excited-state}\\
            u_2(x)&=N_2\left(\frac{2m\omega}{\hbar}x^2-1\right)e^{-\frac{m\omega}{2\hbar}x^2}\\
            u_3(x)&=N_3\left(8\sqrt{\left(\frac{m\omega}{\hbar}\right)^3}y^3-12\sqrt{\frac{m\omega}{\hbar}}y\right)e^{-\frac{m\omega}{2\hbar}x^2}\\
        \end{align}
        % TODO: calculate all normalisations and plot
            
        \subsection{Comparison of the Quantum to Classical Harmonic Oscillator}\label{subsec:comparison-of-the-quantum-to-classical-harmonic-oscillator}
            Here are the probability density of the first few eigenstates, with the classical probability density of a simple harmonic oscillator with energy $E=\hbar\omega\left(n+\frac{1}{2}\right)$ overlayed.
            We can see that as $n$ increases, the spatial probability density of the oscillator is more spread out, just like how the amplitude of them classical turning points increase for higher energy.
            The quantum probability density is actually nonzero beyond the classical turning point, so we have some tunnelling, but it decays exponentially in this region.
            Also note that the wavelength of $\psi_n$ is shortest at the center of the well, which is where a classical particle would be travelling the fastest.
            % http://tinyurl.com/qu-SHO

        \subsection{Parity}\label{sec:energy-eigenstates:subsec:parity}
            Recall that we saw in section~\ref{sec:solving-the-quantum-harmonic-oscillator:subsec:solving-for-h(y)-via-a-series-expansion} that the Hermite polynomials are either even or odd.
            This property is passed on to the eigenstates as well.
            We can investigate this symmetry by defining an operator which will detect whether a given wavefunction is odd or even.
            
            \begin{definition}
                The \textbf{parity} operator switches $x$ to $-x$, and can be defined by its action on a wavefunction as
                \begin{equation}
                    \hat{P}\psi(x,t)=\psi(-x,t).
                \end{equation}
            \end{definition}

            What are the possible eigenvalues of parity?
            If we apply parity twice, we get
            \begin{equation}
                \hat{P}^2\psi(x)=\hat{P}\hat{P}\psi(x)=\hat{P}\psi(-x)=\psi(x),
            \end{equation}
            so \textit{all} wavefunctions are eigenfunctions of $\hat{P}^2$ with eigenvalue $+1$.
            Using this fact, we can show that the eigenvalues of $\hat{P}$ are $\pm 1$.
            Suppose that $\psi(x)$ is an eigenfunction of $\hat{P}$ with eigenvalue $P$, then
            \begin{equation}
                \hat{P}^2\psi(x)=\hat{P}P\psi(x)=P^2\psi(x),
            \end{equation}
            implies that $P^2=1$, hence $P=\pm 1$.
            If a wavefunction is an eigenfunction of parity with $\hat{P}\psi(x)=\psi(x)$, then this means that $\psi(x)$ is an even function (symmetric about $x=0$) since this implies $\psi(-x)=\psi(x)$.
            Likewise, if we have $\hat{P}\psi(x)=-\psi(x)$, then $\psi(x)$ must be odd (antisymmetric about $x=0$) since this implies $\psi(x)=-\psi(-x)$.
            Not all wavefunctions are even or odd, therefore not all wavefunctions are eigenfunctions of parity.

            Is parity Hermitian?
            We can test this by calculating the adjoint, note that the integral has to be from $-\infty$ to $\infty$ for the harmonic oscillator.
            \begin{align}
                \langle\hat{P}^\dagger\rangle&=\int_{-\infty}^\infty(\hat{P}\psi(x))^\ast\psi(x)\dd{x}\\
                &=\int_{-\infty}^\infty\psi^\ast(-x)\psi(x)\dd{x},
            \end{align}
            now we make a substitution $y=-x$ to get
            \begin{align}
                \langle\hat{P}^\dagger\rangle&=-\int_\infty^{-\infty}\psi^\ast(y)\psi(-y)\dd{y}\\
                &=\int_{-\infty}^\infty\psi^\ast(y)\hat{P}\psi(y)\dd{y}\\
                &=\langle\hat{P}\rangle,
            \end{align}
            hence $\hat{P}$ is Hermitian and therefore it is observable.
            % TODO: talk about how we can observe parity

        \subsection{Superposition States}\label{chap:quantum-harmonic-oscillator:sec:energy-eigenstates:subsec:superposition-states}
            We can write superposition states just like we can for the infinite square well, using an expansion of energy eigenfunctions.
            \begin{equation}
                \psi(x,t)=\sum_{n=0}^\infty c_nu_n(x)e^{-\frac{iE_nt}{\hbar}}.
            \end{equation}
            The only difference between this formula and the one for the infinite square well is that the index $n$ starts from 0 instead of 1.
            % TODO: talk about coherent states

    \section{Solving the QHO with Ladder Operators}\label{sec:solving-the-qho-with-ladder-operators}
        In this section we are going to introduce a technique to solve for the energy eigenvalues and eigenfunctions without solving the complicated differential equation~\ref{eq:qho-tise}.
        This method will be to factorise the Hamiltonian using two new operators, which we will call ladder operators for reasons that will become clear, and then see what their effect on an unknown eigenfunction is.

        \subsection{Factorising the Hamiltonian}\label{sec:solving-the-qho-with-ladder-operators:subsec:factorising-the-hamiltonian}
            Notice that the QHO Hamiltonian (equation~\ref{eq:qho-hamiltonian}) is a sum of squares of operators.
            We would like to write it in a more symmetric form and then factorise.

            We can do this by removing the units.
            Define the dimensionless position operator by
            \begin{equation}
                \hat{X}=\frac{\hat{x}}{x_0},
            \end{equation}
            where $x_0=\sqrt{\frac{\hbar}{m\omega}}$ is the natural length scale created using the constants in the Hamiltonian.
            We can define a dimensionless momentum operator by creating a nautral momentum scale, which is created by dividing $\hbar$ by $x_0$:
            \begin{equation}
                p_0=\frac{\hbar}{x_0}=\sqrt{\hbar m\omega}.
            \end{equation}
            Then we can define
            \begin{equation}
                \hat{P}=\frac{\hat{x}}{p_0},
            \end{equation}
            as the dimensionless momentum operator.

            Substituting these into equation~\ref{eq:qho-hamiltonian}, we get
            \begin{align}
                \hat{H}&=\frac{p_0^2\hat{P}^2}{2m}+\frac{1}{2}m\omega^2x_0^2\hat{X}^2\\
                &=\frac{\hbar m\omega}{2m}\hat{P}^2+\frac{m\omega^2\hbar}{m\omega}\hat{X}^2\\
                &=\frac{\hbar\omega}{2}(\hat{P}^2+\hat{X}^2).
            \end{align}

            Recall that an expression $u^2+v^2$ can be factored using complex numbers as $(u+iv)(u-iv)$.
            When using the same idea with operators, we need to keep in mind that two operators may not necessarily commute.
            The two parentheses expand to
            \begin{align}
                (\hat{U}-i\hat{V})(\hat{U}+i\hat{V})&=\hat{U}^2+\hat{V}^2+i\hat{U}\hat{V}-i\hat{V}\hat{U}\\
                &=\hat{U}^2+\hat{V}^2+i[\hat{U},\hat{V}],
            \end{align}
            therefore for the QHO Hamiltonian, we get
            \begin{equation}
                \hat{H}=\frac{\hbar\omega}{2}((\hat{X}-i\hat{P})(\hat{X}+i\hat{P})-i[\hat{X},\hat{P}]),
            \end{equation}
            where we have to subtract of $i$ times the commutator of $\hat{X}$ and $\hat{P}$ to account for the cross terms.
            The commutator between dimensionless position and momentum is given by
            \begin{align}
                [\hat{X},\hat{P}]&=\left[\frac{\hat{x}}{x_0},\frac{\hat{p}}{p_0}\right]\\
                &=\frac{1}{x_0p_0}[\hat{x},\hat{p}]\\
                &=\sqrt{\frac{m\omega}{\hbar}}\frac{1}{\sqrt{\hbar m\omega}}i\hbar\\
                &=i,
            \end{align}
            So the Hamiltonian simplifies to
            \begin{equation}
                \hat{H}=\frac{\hbar\omega}{2}((\hat{X}-i\hat{P})(\hat{X}+i\hat{P})+1).
            \end{equation}

            Now we will define new operators to be the combination in parentheses:
            \begin{align}
                \hat{A}&=\frac{1}{\sqrt{2}}(\hat{X}+i\hat{P})\\
                \hat{A}^\dagger&=\frac{1}{\sqrt{2}}(\hat{X}-i\hat{P}).
            \end{align}
            These operators are called the \textbf{ladder operators} for reasons that will become very clear in the next section.
            If we substitute these into the Hamiltonian, we get
            \begin{equation}\label{eq:qho-hamiltonian-ladder}
                \hat{H}=\hbar\omega\left(\hat{A}^\dagger\hat{A}+\frac{1}{2}\right).
            \end{equation}
            This equation looks suspiciously like equation~\ref{eq:qho-eigenvalues} for the energy eigenvalues but with operators!
            This implies that the product $\hat{A}^\dagger\hat{A}$ is a ``number operator'', it has the energy eigenstates as its eigenfunctions with their corresponding quantum number $n$ as its eigenvalues.
            % TODO: prove that this is the case
            % TODO: talk about the opposite order of factorisation which gives H=hbar*omega*(AA^+ - 1/2)

            But what do the operators $\hat{A}$ and $\hat{A}^\dagger$ actually do?
            Note that $\hat{A}^\dagger$ is the adjoint of $\hat{A}$, as we can tell by the notation and because it is defined as the ``complex conjugate'' of $\hat{A}$.
            Because $\hat{A}\neq\hat{A}^\dagger$, neither of them are Hermitian and therefore they do not represent observables.
            We will look at the action of these operators on an energy eigenstate in the next section.

    \section{Ladder Operators}\label{sec:ladder-operators}
        To work out the action of the ladder operators on an energy eigenstate $u_n$, we will first calculate some commutators involving them and then do some algebra to find the result.
        
        \subsection{Properties of Ladder Operators}\label{subsec:properties-of-ladder-operators}
            We have already seen that since $\hat{A}$ and $\hat{A}^\dagger$ are different operators, they cannot be Hermitian.

            The first commutator we will need is the commutator of $\hat{A}$ with its adjoint $\hat{A}^\dagger$, which is
            \begin{align}
                [\hat{A},\hat{A}^\dagger]&=\frac{1}{2}[\hat{X}+i\hat{P},\hat{X}-i\hat{P}]\\
                &=\frac{1}{2}([\hat{X},\hat{X}]+i[\hat{P},\hat{X}]-i[\hat{X},\hat{P}]-i^2[\hat{P},\hat{P}])\\
                &=\frac{1}{2}(0-i^2-i^2+0)\\
                &=1,
            \end{align}
            So $\hat{A}$ and $\hat{A}^\dagger$ do not commute.

            The next commutator to calculate is that of the ladder operators with the Hamiltonian.
            Looking at $\hat{A}$ first, this is
            \begin{align}
                [\hat{H},\hat{A}]&=\hbar\omega\left[\left(\hat{A}^\dagger\hat{A}+\frac{1}{2}\right),\hat{A}\right]\\
                &=\hbar\omega\left([\hat{A}^\dagger\hat{A},\hat{A}]+\left[\frac{1}{2},\hat{A}\right]\right)\\
                &=\hbar\omega(\hat{A}^\dagger[\hat{A},\hat{A}]+[\hat{A}^\dagger,\hat{A}]\hat{A})\\
                &=-\hbar\omega\hat{A}.
            \end{align}
            % TODO: make sure all these steps are using identities that have been defined previously (addition, multiplication, constants inside commutator, antisymmetry)
            By a similar calculation, it can be shown that $[\hat{H},\hat{A}^\dagger]=+\hbar\omega\hat{A}^\dagger$.

        \subsection{The Lowering Operator}\label{subsec:the-lowering-operator}
            Suppose we have the an energy eigenstate $u_i$, so $\hat{H}u_i=E_iu_i$.
            We will now calculate what $\hat{A}$ when it acts on $u_i$ by acting the Hamiltonian on $\hat{A}u_i$.
            When doing this calculation, we will deliberately ``add zero'' (add and subtract the same term) in order to introduce the commutators we have calculated above.
            \begin{align}
                \hat{H}(\hat{A}u_i)&=(\hat{H}\hat{A}-\hat{A}\hat{H}+\hat{A}\hat{H})u_i\\
                &=[\hat{H},\hat{A}]u_i+\hat{A}(\hat{H}u_i)\\
                &=-\hbar\omega\hat{A}u_i+E_i\hat{A}u_i\\
                &=(E_i-\hbar\omega)\hat{A}u_i.
            \end{align}
            This does not really look like we have made much progress, but notice that the last line is an eigenvalue equation, which implies that $\hat{A}u_i$ is an energy eigenfunction with energy $E_i-\hbar\omega$.
            Of course, since we know what the energy eigenvalues are, we can see that $\hat{A}u_i$ must be proportional to $u_{i-1}$ because $E_i-\hbar\omega=E_{i-1}$!

            Let us apply $\hat{A}$ twice to $u_i$ and see what happens.
            Using the same trick, we get
            \begin{align}
                \hat{H}(\hat{A}^2u_i)&=\hat{H}\hat{A}(\hat{A}u_i)\\
                &=(\hat{H}\hat{A}-\hat{A}\hat{H}+\hat{A}\hat{H})(\hat{A}u_i)\\
                &=[\hat{H},\hat{A}](\hat{A}u_i)+\hat{A}\hat{H}(\hat{A}u_i)\\
                &=-\hbar\omega\hat{A}^2u_i+(E_i-\hbar\omega)\hat{A}^2u_i\\
                &=(E_i-2\hbar\omega)\hat{A}^2u_i,
            \end{align}
            so we see, in the same way, that $\hat{A}^2u_i$ is proportional to $u_{i-2}$ since $E_i-2\hbar\omega=E_{i-2}$.

            One can show that in general,
            \begin{equation}
                \hat{H}(\hat{A}^nu_i)=(E_i-n\hbar\omega)\hat{A}^nu_i.
            \end{equation}

            So the action of $\hat{A}$ is to \textit{lower} an eigenstate $u_i$ by one energy unit of $\hbar\omega$.
            For this reason, $\hat{A}$ is called the \textbf{lowering operator}.

        \subsection{The Raising Operator}\label{subsec:the-raising-operator}
            If $\hat{A}$ lowers an energy eigenstate by one unit of $\hbar\omega$, one would expect that its adjoint $\hat{A}^\dagger$ raises it up by one.
            Indeed, we find that this is the case, so $\hat{A}^\dagger$ is called the \textbf{raising operator}.

            We can show that this is the case by doing a similar calculation to above.
            \begin{align}
                \hat{H}(\hat{A}^\dagger u_i)&=(\hat{H}\hat{A}^\dagger-\hat{A}^\dagger\hat{H}+\hat{A}^\dagger\hat{H})u_i\\
                &=[\hat{H},\hat{A}^\dagger]u_i+\hat{A}^\dagger(\hat{H}u_i)\\
                &=\hbar\omega\hat{A}^\dagger u_i+E_i\hat{A}^\dagger u_i\\
                &=(E_i+\hbar\omega)\hat{A}^\dagger u_i,\label{eq:qho-ladder-raising-action}
            \end{align}
            as predicted.

            Again, one can show in general that
            \begin{equation}
                \hat{H}(\hat{A}^{\dagger\,n}u_i)=(E_i+n\hbar\omega)\hat{A}^{\dagger\,n}u_i.
            \end{equation}

            One easy way to remember which operator raises and which one lowers is by imagining the Hermitian dagger $\dagger$ as a little ``+'' symbol, which denotes raising.

        \subsection{The Ladder of Energy Levels}\label{subsec:the-ladder-of-energy-levels}
            The raising and lowering operators imply that we have a ``ladder'' of energy levels, equally spaced apart by $\hbar\omega$.
            However, there is one problem.
            It appears that there is nothing stopping us from applying the lowering operator $\hat{A}$ indefinitely, implying that there are infinitely many rungs in both directions!
            We know that this is impossible, since the potential energy $V(x)$ has a minimum value of $V=0$, which means there are \textit{no bound states} with $E_i<0$.

            We can also show directly that the energy must be greater than or equal to zero.
            Recall that the expectation value of the Hamiltonian for an energy eigenstate $u_n$ is $E_n$:
            \begin{equation}
                \expval{\hat{H}}{u_n}=\int_{-\infty}^\infty u_n^\ast\hat{H}u_n\dd{x}=E_n\int_{-\infty}^\infty u_n^\ast u_n\dd{x}=E_n.
            \end{equation}
            Then if we insert equation~\ref{eq:qho-hamiltonian-ladder}, we get
            \begin{align}
                E_n&=\hbar\omega\int_{-\infty}^\infty u_n^\ast\left(\hat{A}^\dagger\hat{A}+\frac{1}{2}\right)u_n\dd{x}\\
                &=\hbar\omega\left[\int_{-\infty}^\infty u_n^\ast\hat{A}^\dagger\hat{A}u_n\dd{x}+\frac{1}{2}\int_{-\infty}^\infty u_n^\ast u_n\dd{x}\right]\\
                &=\hbar\omega\left[\int_{-\infty}^\infty(\hat{A}u_n)^\ast\hat{A}u_n\dd{x}+\frac{1}{2}\right]\\
                &=\hbar\omega\left[\int_{-\infty}^\infty\abs{\hat{A}u_n}^2\dd{x}+\frac{1}{2}\right]\\
                &\geq 0.
            \end{align}
            Note that in the third-last line, we used the fact that $u_n^\ast\hat{A}^\dagger$ is the complex conjugate of $\hat{A}u_n$, and to go from the second-last line to the last we recall that probability density must be non-negative everywhere.
            % TODO: include Dirac notation proof: <n|N|n>=<n|A^+ A|n>=(A|n>)^* A|n> geq 0

            All this is to say that we need to impose that there exists a ground state with a minimum energy.
            To do this, we define the lowering operator to completely annihilate the ground state wavefunction
            \begin{equation}
                \hat{A}u_0=0.
            \end{equation}
            This is known as the \textbf{ladder termination condition}.
            It implies that any repeated application of the lowering operator will eventually end up returning zero (since $\hat{A}^n u_0=0$ for any $n$).
            % TODO: do any problems arise if we "forget" to impose this condition?
            This actually gives us the value of the ground state energy.
            If we apply $\hat{H}$ to $u_0$, we get
            \begin{align}
                \hat{H}u_0&=\hbar\omega\left(\hat{A}^\dagger\hat{A}+\frac{1}{2}\right)u_0\\
                &=\hbar\omega\hat{A}^\dagger(\hat{A}u_0)+\frac{1}{2}\hbar\omega u_0\\
                &=\frac{1}{2}\hbar\omega u_0,
            \end{align}
            since the first term becomes zero.
            So the ground state energy is $\frac{1}{2}\hbar\omega$, which is what we found before by solving the Schrodinger equation.

            This, along with equation~\ref{eq:qho-ladder-raising-action}, gives us the full spectrum of energy levels
            \begin{equation}
                E_n=\hbar\omega\left(n+\frac{1}{2}\right),
            \end{equation}
            as found previously.
            % TODO: necessary to explicitly show this?

        \subsection{Finding the Eigenfunctions Using Ladder Operators}\label{subsec:finding-the-eigenfunctions-using-ladder-operators}
            The only thing we haven't recovered yet from our knowledge of the QHO using this method is the form of the energy eigenstates in the position basis.
            According to equation~\ref{eq:qho-ladder-raising-action}, $\hat{A}^\dagger u_n$ is proportional to $u_{n+1}$.
            We can calculate the constant of proportionality by setting $\hat{A}^\dagger u_n=Cu_{n+1}$ and using the fact that the energy eigenstates are normalised.
            \begin{align}
                1&=\int_{-\infty}^\infty u_{n+1}^\ast u_{n+1}\dd{x}\\
                &=\frac{1}{C^2}\int_{-\infty}^\infty(\hat{A}^\dagger u_n)^\ast\hat{A}^\dagger u_n\dd{x}\\
                &=\frac{1}{C^2}\int_{-\infty}^\infty u_n^\ast\hat{A}\hat{A}^\dagger u_n\dd{x}.
            \end{align}
            We will now use a nifty trick that comes in handy when manipulating expressions with ladder operators.
            With the commutator $[\hat{A},\hat{A}^\dagger]=1$, we can rewrite
            \begin{equation}
                \hat{A}\hat{A}^\dagger=\hat{A}\hat{A}^\dagger-\hat{A}^\dagger\hat{A}+\hat{A}^\dagger\hat{A}=[\hat{A},\hat{A}^\dagger]+\hat{A}^\dagger\hat{A}=1+\hat{A}^\dagger\hat{A}.
            \end{equation}
            Using this and substituting the Hamiltonian, we get
            \begin{align}
                1&=\frac{1}{C^2}\int_{-\infty}^\infty u_n^\ast(1+\hat{A}^\dagger\hat{A})u_n\dd{x}\\
                &=\frac{1}{C^2}\int_{-\infty}^\infty u_n^\ast\left(1+\frac{\hat{H}}{\hbar\omega}-\frac{1}{2}\right)u_n\dd{x}\\
                &=\frac{1}{C^2}\int_{-\infty}^\infty u_n^\ast\left(\frac{\hat{H}}{\hbar\omega}u_n+\frac{1}{2}u_n\right)\dd{x}\\
                &=\frac{1}{C^2}\int_{-\infty}^\infty u_n^\ast\left(\left(n+\frac{1}{2}\right)u_n+\frac{1}{2}u_n\right)\dd{x}\\
                &=\frac{n+1}{C^2}\int_{-\infty}^\infty u_n^\ast u_n\dd{x}\\
                &=\frac{n+1}{C^2},
            \end{align}
            which implies $C=\sqrt{n+1}$.
            Thus, we have
            \begin{equation}\label{eq:qho-ladder-raising-normalised}
                \hat{A}^\dagger u_n=\sqrt{n+1}u_{n+1}.
            \end{equation}
            % TODO: again, show Dirac notation proof, e.g. bottom of this section https://en.wikipedia.org/wiki/Quantum_harmonic_oscillator#Ladder_operator_method

            By a similar calculation, one can show that
            \begin{equation}
                \hat{A} u_n=\sqrt{n}u_{n-1}.
            \end{equation}

            Using the ladder termination condition and expanding $\hat{A}$ in the position basis, we can find the form of the ground state wavefunction in the position basis.
            \begin{align}
                &\hat{A}u_0=0\\
                &\frac{1}{\sqrt{2}}(\hat{X}+i\hat{P})u_0=0\\
                &\left(\frac{\hat{x}}{x_0}+i\frac{\hat{p}}{p_0}\right)u_0=0\\
                &\sqrt{\frac{m\omega}{\hbar}}u_0+\frac{i}{\sqrt{\hbar m\omega}}\left(-i\hbar\dv{u_0}{x}\right)=0
            \end{align}
            This gives us a first-order differential equation which we can solve by integrating:
            \begin{align}
                \dv{u_0}{x}&=-\frac{m\omega}{\hbar}xu_0\\
                \int\frac{\dd{u_0}}{u_0}&=-\frac{m\omega}{\hbar}\int x\dd{x}\\
                \ln u_0&=-\frac{m\omega}{2\hbar}x^2+c\\
                u_0&=N_0e^{-\frac{m\omega}{2\hbar}x^2},
            \end{align}
            which is exactly the wavefunction we found by solving the Schrodinger equation.
            Note that if we were solving this problem from scratch using ladder operators, this would be the \textit{only} differential equation we have to solve, and also the only time that we have to refer back to the definition of the ladder operators in terms of $\hat{X}$ and $\hat{P}$!

            The rest of the eigenstates can be found using equation~\ref{eq:qho-ladder-raising-normalised} and by expanding the definition of $\hat{A}^\dagger$.
            For example, for the first excited state we have $\hat{A}^\dagger u_0=\sqrt{0+1}u_1=u_1$, so we get
            \begin{align}
                u_1&=\frac{1}{\sqrt{2}}(\hat{X}-i\hat{P})u_0\\
                &=\frac{1}{\sqrt{2}}\left(\sqrt{\frac{m\omega}{\hbar}}x-\sqrt{\frac{\hbar}{m\omega}}\dv{}{x}\right)N_0e^{-\frac{m\omega}{2\hbar}x^2}\\
                &=N_0\sqrt{\frac{2m\omega}{\hbar}}xe^{-\frac{m\omega}{2\hbar}x^2},
            \end{align}
            which is the same as equation~\ref{eq:qho-first-excited-state}.
            Note that when using this method for generating the eigenfunctions, they are already normalised!
            % TODO: inductively prove the form of the wavefunctions for all n

    % TODO: include formula for ladder operators in terms of dimensionful operators
    % TODO: talk about ladder operators being generators of rotation in phase space for classical oscillator

\end{document}
