\documentclass[../quantum_mechanics.tex]{subfiles}

\begin{document}

    \section{The 3D Quantum Harmonic Oscillator}\label{sec:the-3d-quantum-harmonic-oscillator}
        % TODO: rewrite main principles of this section as solving the infinite square well in 3D
        Solving the quantum harmonic oscillator in one dimension proved to be quite a slog, but we will see that in more than one dimension it is much easier!
        This is because we can use the method of separation of variables to break down the problem into copies of the one-dimensional problem, for which we already know the solution.

        \subsection{Schrodinger Equation for the 3D QHO}\label{subsec:schrodinger-equation-for-the-3d-qho}
            The first thing we need to do is write out the potential energy so that we can write down the Schrodinger equation.
            The particle is going to be under the influence of a central force $\vec{F}=-\sqrt{\omega/m}\vec{r}$, which gives a potential energy
            \begin{equation}
                \hat{V}=\frac{1}{2}m\omega^2\hat{r}^2=\frac{1}{2}m\omega^2(\hat{x}^2+\hat{y}^2+\hat{z}^2).
            \end{equation}
            Note that here we are assuming that force grows in magnitude with the same rate in every direction, and therefore the natural frequency $\omega$ is the same in all directions.
            We call an oscillator like this \textbf{isotropic}.

            For kinetic energy, we can write
            \begin{equation}
                \hat{T}=\frac{\hat{p}^2}{2m}=\frac{\hat{\vec{p}}\cdot\hat{\vec{p}}}{2m},
            \end{equation}
            where $\hat{\vec{p}}$ is the 3D momentum operator, which components $\hat{\vec{p}}=(\hat{p}_x,\hat{p}_y,\hat{p}_z)$ given by
            \begin{align}
                \hat{p}_x&=-i\hbar\pdv{}{x}\\
                \hat{p}_y&=-i\hbar\pdv{}{y}\\
                \hat{p}_z&=-i\hbar\pdv{}{z}.
            \end{align}
            Therefore, kinetic energy in 3D can be written as
            \begin{align}
                \hat{T}&=\frac{1}{2m}(\hat{p}_x^2+\hat{p}_y^2+\hat{p}_z^2)\\
                &=-\frac{\hbar^2}{2m}\left(\pdv[2]{}{x}+\pdv[2]{}{y}+\pdv[2]{}{z}\right)\\
                &=-\frac{\hbar^2}{2m}\nabla^2.
            \end{align}
            where $\nabla^2$ is the Laplacian operator.

            Putting these together, the time-independent Schrodinger equation in the position basis is
            \begin{align}
                \left(-\frac{\hbar^2}{2m}\nabla^2+\frac{1}{2}m\omega^2r^2\right)u(\vec{r})&=Eu(\vec{r})\\
                \left(-\frac{\hbar^2}{2m}\left(\pdv[2]{}{x}+\pdv[2]{}{y}+\pdv[2]{}{z}\right)+\frac{1}{2}m\omega^2(x^2+y^2+z^2)\right)u(x,y,z)&=Eu(x,y,z).\label{eq:qho-tise-3d}
            \end{align}
            The TISE is now a partial differential equation, rather than an ordinary one, as the spatial part of the wavefunction now depends on three variables: $x$, $y$, and $z$.
            As in the 1D case, the full energy eigenfunctions are obtained by multiplying the spatial part with the temporal part, so we get
            \begin{equation}
                \psi(x,y,z,t)=u(x,y,z)e^{-\frac{iEt}{\hbar}},
            \end{equation}
            for whatever the energy eigenvalues $E$ happen to be.

        \subsection{Solving the TISE by Separation of Variables}\label{sec:the-3d-quantum-harmonic-oscillator:subsec:solving-the-tise-by-separation-of-variables}
            Note that the potential energy has the form $V(x,y,z)=V_x(x)+V_y(y)+V_z(z)$.
            For a classical oscillator with a potential like this, the motion along each dimension is unaffected by the motion along the others.
            Therefore, we assume that the eigenfunctions are separable, so they take the form
            \begin{equation}
                u(x,y,z)=u_x(x)u_y(y)u_z(z).
            \end{equation}
            
            Inserting this form into equation~\ref{eq:qho-tise-3d} gives
            \begin{equation}
                -\frac{\hbar^2}{2m}\left(u_yu_z\dv[2]{u_x}{x}+u_xu_z\dv[2]{u_y}{y}+u_xu_y\dv[2]{u_z}{z}\right)+\frac{1}{2}m\omega^2(x^2+y^2+z^2)u_xu_yu_z=Eu_xu_yu_z,
            \end{equation}
            then dividing by $u_xu_yu_z$ so that each term only depends on a single variable, we get
            \begin{equation}\label{eq:qho-tise-3d-separated}
                \begin{split}
                &\left(-\frac{\hbar^2}{2m}\frac{1}{u_x}\dv[2]{u_x}{x}+\frac{1}{2}m\omega^2x^2\right)+\left(-\frac{\hbar^2}{2m}\frac{1}{u_y}\dv[2]{u_y}{y}+\frac{1}{2}m\omega^2y^2\right)\\
                &+\left(-\frac{\hbar^2}{2m}\frac{1}{u_z}\dv[2]{u_z}{z}+\frac{1}{2}m\omega^2z^2\right)=E 
                \end{split}
            \end{equation}

            Now, the whole left-hand side must equal the total energy $E$, which is a constant.
            However, $x$, $y$, and $z$ can all be varied completely independently.
            This means that each term in large parentheses must itself be equal to a constant.
            These constants represent the energy along each axis, so we will denote them $E_x$, $E_y$, and $E_z$.
            By equation~\ref{eq:qho-tise-3d-separated}, we have
            \begin{equation}
                E=E_x+E_y+E_z.
            \end{equation}
            We then have three separate ordinary differential equations:
            \begin{align}
                -\frac{\hbar^2}{2m}\dv[2]{u_x}{x}+\frac{1}{2}m\omega^2x^2u_x&=E_xu_x\\
                -\frac{\hbar^2}{2m}\dv[2]{u_y}{y}+\frac{1}{2}m\omega^2y^2u_y&=E_yu_y\\
                -\frac{\hbar^2}{2m}\dv[2]{u_z}{z}+\frac{1}{2}m\omega^2z^2u_z&=E_zu_z.
            \end{align}
            These are three copies of the 1D quantum harmonic oscillator!
            We therefore know what the forms of $u_x$, $u_y$, and $u_z$ will be as we have solved for them already.
            The energy eigenstates of the 3D isotropic quantum harmonic oscillator will then be products of the the one-dimensional eigenstates.

            The solution for each coordinate axis will have its own quantum number to describe which energy level the particle is on along that direction.
            We call these three labels $n_x$, $n_y$, $n_z$.
            Then the energy eigenvalues along each axis are
            \begin{align}
                E_x&=\hbar\omega\left(n_x+\frac{1}{2}\right)\\
                E_y&=\hbar\omega\left(n_y+\frac{1}{2}\right)\\
                E_z&=\hbar\omega\left(n_z+\frac{1}{2}\right),
            \end{align}
            just like in the one-dimensional case.

            This means that the total energy has three quantum numbers which label the eigenvalues, which are given by
            \begin{equation}\label{eq:qho-eigenvalues-3d}
                E_{n_xn_yn_z}=E_x+E_y+E_z=\hbar\omega\left(n_x+n_y+n_z+\frac{3}{2}\right).
            \end{equation}
            Each quantum number starts from zero, and they can all take on any value independent of each other.
            Clearly we will have some combinations of values which have the same energy, for example $n_x=n_y=n_z=1$, or $n_x=2$, $n_y=1$, and $n_z=0$, or $n_x=3$ and $n_y=n_z=0$!
            We will discuss what this means in the next section.

    \section{Degeneracy}\label{sec:degeneracy}
        \subsection{Energy Eigenstates of the 3D Isotropic QHO}\label{subsec:energy-eigenstates-of-the-3d-qho}
            As derived in the last section, the energy eigenvalues of the 3D isotropic quantum harmonic oscillator are
            \begin{equation}
                E_{n_xn_yn_z}=\hbar\omega\left(n_x+n_y+n_z+\frac{3}{2}\right),
            \end{equation}
            and the spatial parts of the eigenfunctions are
            \begin{align}
                u_{n_xn_yn_z}(x,y,z)&=u_{n_x}(x)u_{n_y}(y)u_{n_z}(z)\\
                &=N_{n_x}N_{n_y}N_{n_z}h_{n_x}\left(\frac{x}{x_0}\right)h_{n_y}\left(\frac{y}{x_0}\right)h_{n_z}\left(\frac{z}{x_0}\right)e^{-\frac{m\omega}{2\hbar}(x^2+y^2+z^2)},
            \end{align}
            where we recall the natural length scale $x_0=\sqrt{\frac{\hbar}{m\omega}}$.
            % check and calculate this is correct
            The full time-dependent eigenfunctions will then simply be
            \begin{align}
                \psi_{n_xn_yn_z}(x,y,z,t)&=u_{n_xn_yn_z}(x,y,z)e^{-\frac{iE_{n_xn_yn_z}t}{\hbar}}\\
                &=u_{n_x}(x)e^{-\frac{iE_{n_x}}{\hbar}}u_{n_y}(y)e^{-\frac{iE_{n_y}}{\hbar}}u_{n_z}(z)e^{-\frac{iE_{n_z}}{\hbar}},
            \end{align}
            where in the last line we have shown explicitly that the full wavefunction really is the product of the three individual wavefunctions for each direction using $E_{n_xn_yn_z}=E_{n_x}+E_{n_y}+E_{n_z}$.

            The ground state energy for the 3D QHO is found when $n_x=n_y=n_z=0$, which gives
            \begin{equation}
                E_{000}=\hbar\omega\left(0+0+0+\frac{3}{2}\right)=\frac{3}{2}\hbar\omega.
            \end{equation}
            This is three times as much energy as for a 1D oscillator.
            % TODO: comment on why
            Since the quantum numbers $n_x$, $n_y$, and $n_z$ can vary independently, the spacing between energy levels is still $\hbar\omega$ just like in the one-dimensional case.

            The ground state wavefunction looks like
            \begin{equation}
                \psi_{000}(x,y,z,t)=N_{000}e^{-\frac{m\omega}{2\hbar}(x^2+y^2+z^2)}e^{-\frac{3}{2}i\omega t}.
            \end{equation}

            In dirac notation, we denote the eigenstates by $\ket{n_xn_yn_z}$, so the ground state would be denoted $\ket{000}$.

            The first excited state is the state with the second lowest energy, which would be $E=\frac{5}{2}\hbar\omega$ for the 3D isotropic oscillator.
            However, from equation~\ref{eq:qho-eigenvalues-3d}, we can see that there are several states that have this energy!
            They are:
            \begin{equation}
                E_{100}=E_{010}=E_{001}=\frac{5}{2}\hbar\omega.
            \end{equation}
            This was not possible in for the one-dimensional oscillator, but becomes possible now because of the extra degrees of freedom and because of the symmetry of the situation.

            The three eigenstates that have this eigenvalue have three \textit{distinct} wavefunctions, given by
            \begin{align}
                \psi_{100}(x,y,z,t)&=N_{100}xe^{-\frac{m\omega}{2\hbar}(x^2+y^2+z^2)}e^{-\frac{5}{2}i\omega t}\\
                \psi_{010}(x,y,z,t)&=N_{010}ye^{-\frac{m\omega}{2\hbar}(x^2+y^2+z^2)}e^{-\frac{5}{2}i\omega t}\\
                \psi_{001}(x,y,z,t)&=N_{001}ze^{-\frac{m\omega}{2\hbar}(x^2+y^2+z^2)}e^{-\frac{5}{2}i\omega t}
            \end{align}
            These wavefunctions, which correspond to the states $\ket{100}$, $\ket{010}$, and $\ket{001}$, are still orthonormal and so they represent physically distinct states, yet they have the same energy.
            These are known as \textbf{degenerate states}.
            \begin{definition}
                \textbf{Degenerate states} are states that share the same eigenvalue of a Hermitian operator.

                The number of states with the same eigenvalue is called the \textbf{degeneracy} of the eigenvalue.
                For example, the first excited state of the 3D isotropic quantum harmonic oscillator can correspond to three different eigenstates, so we say that the first excited state has a degeneracy of 3.
            \end{definition}

        % TODO: calculate degeneracy of each level

        % TODO: include a subsection on visualising
        % http://tinyurl.com/qu-SHO2

\end{document}
