\documentclass[../quantum_mechanics.tex]{subfiles}

\begin{document}

    \section{The Free Particle}\label{sec:free-particle}
        Now that we have done plenty of exposition, we will finally study at what the motion of a quantum particle actually looks like.
        We will first analyse the simplest case, which is that of a free particle.

        \subsection{Solving the Schrodinger Equation}\label{sec:free-particle:subsec:solving-the-schrodinger-equation}
            A free particle is under the influence of no forces, so the potential $V(x)$ is zero.
            The TISE (equation~\ref{eq:tise}) then becomes
            \begin{gather}\label{eq:free-tise}
                -\frac{\hbar^2}{2m}\dv[2]{u(x)}{x}=Eu(x)\\
                \implies\dv[2]{u(x)}{x}=-\frac{2mE}{\hbar^2}u(x).
            \end{gather}
            For a free particle, the total energy $E$ is just the kinetic energy $\frac{p^2}{2m}$, so inputting this gives
            \begin{equation}
                \dv[2]{u(x)}{x}=-\frac{p^2}{\hbar^2}u(x)=-k^2u(x),
            \end{equation}
            where we have defined $k^2=\frac{p^2}{\hbar^2}$.

            This happens to be an equation we know very well, the equation for simple harmonic motion!
            This implies that the spatial part of the energy eigenfunctions take the form
            \begin{equation}
                u(x)=Ae^{ikx}.
            \end{equation}
            These solutions are known as \textbf{plane wave} solutions.
            Plane waves of the form $e^{-ikx}$ are also solutions, but we will ignore them for now and bring them back later.

            As we learned in section~\ref{subsec:solving-the-temporal-part-of-the-schrodinger-equation}, the temporal part of the solution is given by equation~\ref{eq:tdse-solution}.
            We multiply this with the spatial part to get the full energy eigenfunctions for the free particle:
            \begin{equation}
                \psi(x,t)=Ae^{\frac{i}{\hbar}(px-Et)}=Ae^{i(kx-\omega t)},
            \end{equation}
            where we have introduced the frequency $\omega$ of the temporal oscillation given by
            \begin{equation}\label{eq:einstein-relation}
                E=\hbar\omega.
            \end{equation}
            This is the well-known Einstein relation.
            % TODO: introduce the Einstein relation when talking about temporal solution to the Schrodinger equation?
            $k$ represents the wavenumber of the oscillation in space, and is related to the particle's momentum by
            \begin{equation}\label{eq:de-broglie-relation}
                p=\hbar k.
            \end{equation}
            This is known as the de Broglie relation.

            Having already derived two seminal results of early quantum mechanics, you might think we are not doing too bad, however there are a couple of problems with these energy eigenfunctions.

            The first issue is that these plane waves clearly do not represent the most general solution to the Schrodinger equation.
            In principle, we should be able to choose any valid initial wavefunction $\psi(x,0)$, which need not even be separable, and then the Schrodinger equation determines its evolution.
            However, for a plane wave, we can only specify $A$ which is its amplitude (and initial phase if $A$ is complex) at $t=0$.

            The second issue is more serious, and it is that the plane wave solutions are \textit{unnormalisable}.
            To see this, we can calculate the integral of probability density over all space:
            \begin{equation}
                \int_{-\infty}^\infty\abs{\psi(x,t)}^2\dd{x}=\int_{-\infty}^\infty\abs{Ae^{i(kx-\omega t)}}^2\dd{x}=\abs{A}^2\int_{-\infty}^\infty\dd{x}=\infty.
            \end{equation}
            % TODO: show this more rigorously
            This happens because plane waves do not decay as $x\to\pm\infty$.
            Physically, this happens because plane waves have \textit{definite} momentum and therefore, by the uncertainty principle, their position is \textit{completely undefined}.
            This reflects the fact that if we calculate the probability density for a plane wave, it is \textit{constant} everywhere in space!
            \begin{equation}
                \abs{\psi(x,t)}^2=\abs{Ae^{i(kx-\omega t)}}^2=\abs{A}^2.
            \end{equation}

            The fact that plane waves have definite momentum is elucidated by the fact that they are eigenstates of the momentum operator.
            This means that if we measure the momentum of a plane wave, we get back a definite result.
            \begin{equation}
                \hat{p}\psi(x,t)=-i\hbar A\pdv{}{x}e^{\frac{i}{\hbar}(px-Et)}=-i\hbar A\frac{ip}{\hbar}e^{\frac{i}{\hbar}(px-Et)}=p\psi(x,t).
            \end{equation}
            To be clear, the fact that momentum eigenstates are unnormalisable means that they are \textbf{unphysical}, a free particle can \textit{never} be found in a stationary state.

            Luckily both of these problems, the lack and generality and the unnormalisability, can be solved using the linearity of the Schrodinger equation, i.e. using the superposition principle.

        \subsection{Wave Packets}\label{sec:free-particle:subsec:wave-packets}
            For a wavefunction to be normalisable, it must be localised to some extent in space.
            Using the principle of superposition, we can add together momentum eigenstates to get a normalisable wavefunction.
            This will have the consequence of the particle no longer having definite momentum, but that means that position will no longer be undefined.

            Let us consider a superposition of two momentum eigenstates with wavenumbers $k_1$ and $k_2$.
            Note that wavenumber $k$ is basically synonymous with momentum in quantum mechanics, as the two are related by the de Broglie relation~\ref{eq:de-broglie-relation}.
            \begin{equation}
                \psi(x,t)=Ae^{i(k_1x-\omega_1t)}+Be^{i(k_2x-\omega_2t)}.
            \end{equation}
            The temporal frequencies $\omega_1$ and $\omega_2$ are given by the Einstein relation~\ref{eq:einstein-relation}, so $\omega_1=\frac{E_1}{\hbar}=\frac{p_1^2}{2m\hbar}$ and similarly for $\omega_2$.
            In the same way that wavenumber is synonymous with momentum, temporal frequency is synonymous with energy.
            This solution is no longer separable, i.e. it is not of the form $\psi(x,t)=u(t)e^{i\omega t}$.
            Now, the probability density is not uniform over all space and this wavefunction is no longer a momentum eigenstate, however this wavefunction is still not normalisable as it is periodic.
            % TODO: show that it is no longer a momentum eigenstate
            
            In fact, we can add together infinitely many plane waves, and this would still be the case.
            \begin{equation}
                \psi(x,t)=\sum_n A_ne^{i(k_nx-\omega_nt)}.
            \end{equation}
            Notice that this is a \textbf{Fourier series}, which are always periodic over the domain $(-\infty,\infty)$.
            % TODO: https://phet.colorado.edu/en/simulations/fourier-making-waves/about

            To fully eliminate the periodicity, we need to make the range of wavenumbers continuous, and so the sum becomes an integral.
            \begin{equation}
                \psi(x,t)=\frac{1}{\sqrt{2\pi}}\int_{-\infty}^\infty A(k)e^{i(kx-\omega t)}\dd{k}.
            \end{equation}
            The intuition for why this is the case is that as the spacing between wavenumbers decreases, the spacing between maxima in the Fourier sum increases.
            So in the limit that the wavenumber spacing goes to zero and becomes continuous, the spacing between maxima goes to infinity.
            Notice that the amplitudes and initial phases of the plane waves $A_n$ have become a function of the wavenumber $A(k)$.
            This definition of $\psi(x,t)$ in terms of an integral of plane waves is known as a \textbf{wave packet}.
            The factor $1/\sqrt{2\pi}$ in front of the integral is placed there because it happens to be the correct normalisation (so the integral of $\abs{\psi(x,t)}^2$ over all space is one), which will be explained in a moment.
            
            If we absorb the time evolution into $A(k)$ so it becomes $A(k,t)=A(k)e^{-i\omega t}$, then this wavefunction becomes
            \begin{equation}\label{eq:psi-fourier}
                \psi(x,t)=\frac{1}{\sqrt{2\pi}}\int_{-\infty}^\infty A(k,t)e^{ikx}\dd{k},
            \end{equation}
            which shows that the wavefunction $\psi(x,t)$ is the \textbf{Fourier transform} of $A(k,t)$.
            Likewise, we can say that the amplitude $A(k,t)$ is the Fourier transform of the wavefunction $\psi(x,t)$:
            \begin{equation}\label{eq:psi-fourier-inverse}
                A(k,t)=\frac{1}{\sqrt{2\pi}}\int_{-\infty}^\infty\psi(x,t)e^{-ikx}\dd{x}.
            \end{equation}
            
            There is a result from mathematics called \textbf{Parseval's theorem}, which says that if $\psi(x,t)$ is normalised, then its Fourier transform $A(k,t)$ is normalised, and vice versa.
            This means that if we enforce
            \begin{equation}
                \int_{-\infty}^\infty\abs{A(k,t)}^2\dd{k}=1,
            \end{equation}
            then $\psi(x,t)$ will be a correctly normalised wavefunction.
            % TODO: write an example proving this

            Note that we can also define this Fourier integral in terms of momentum using the de Broglie relation~\ref{eq:de-broglie-relation}:
            \begin{align}
                \psi(x,t)&=\frac{1}{\sqrt{2\pi\hbar}}\int_{-\infty}^\infty\phi(p,t)e^{\frac{i}{\hbar}px}\dd{p}\label{eq:psi-fourier-momentum}\\
                \phi(p,t)&=\frac{1}{\sqrt{2\pi\hbar}}\int_{-\infty}^\infty\psi(x,t)e^{-\frac{i}{\hbar}px}\dd{x},\label{eq:psi-fourier-momentum-inverse}
            \end{align}
            where the normalisation factor has changed to $1/\sqrt{2\pi\hbar}$.
            % TODO: note that this comes from the scaling property of the Dirac delta

            We can interpret $\abs{A(k,t)}^2$ as the \textbf{wavenumber probability density} and $\abs{\phi(p,t)}^2$ as the \textbf{momentum probability density}, i.e. $\abs{\phi(p,t)}^2\dd{p}$ represents the probability that a measurement of a particle's momentum will be in the range $(p,p+\dd{p})$ at time $t$.

            In summary, we have found that by choosing an appropriate range of momenta and amplitudes for their corresponding plane waves (namely a $\phi(p,t)$ or $A(k,t)$ which is normalisable), we can construct a normalised wave packet $\psi(x,t)$.
            This all followed from the principle of superposition.
            Furthermore, note that the Fourier transform relations above hold for \textit{any} system in quantum mechanics, not just the free particle.
            However, there is an important caveat that $E=p^2/2m$ does not hold in general, only for the free particle.

        % TODO: write general result for an arbitrary initial wavefunction

    \section{Gaussian Wave Packets}\label{sec:gaussian-wave-packets}
        We will now look at a concrete example of a valid state for a free particle constructed as a wave packet and study its properties to get a general idea of how wavefunctions behave in quantum mechanics.
        As discussed in section~\ref{sec:free-particle:subsec:wave-packets} above, this begins with choosing a set of wavenumbers and their amplitudes $A(k)$.
        We will start with constructing the wave packet at $t=0$, then we will discuss how it evolves with time.

        \subsection{Choosing an Amplitude $A(k)$}\label{subsec:choosing-an-amplitude}
            It would be nice to choose $A(k)$ to be a Gaussian shape, because then the Fourier transform $\psi(x,0)$ will also be a Gaussian, and Gaussians are nice to work with.
        
            Recall that a Gaussian function has the general form
            \begin{equation}
                \frac{a}{\sqrt{\pi}}e^{-\frac{(x-c)^2}{b^2}},
            \end{equation}
            which is a bell curve centered at $c$, with height $\frac{a}{\sqrt{\pi}}$, width $b$, and area $ab$.
            This means if we want our wavenumber probability density to be a Gaussian centered at some wavenumber $k_0$ with width $\Delta k$, we should choose $a=\frac{1}{\Delta k}$ so that it will be normalised.
            \begin{equation}\label{eq:gaussian-wavenumber-probability-density}
                A(k)^2=\frac{1}{\Delta k\sqrt{\pi}}e^{-\frac{(k-k_0)^2}{\Delta k^2}}.
            \end{equation}
            The width $\Delta k$ is the distance from the centre at $e^{-1}$ of the maximum height.
            It is a rough measure of the range of wavenumbers of the plane waves that make up the wave packet.
            Note that $\Delta p=\hbar\Delta k$, so the width $\Delta k$ is directly related to the uncertainty in momentum.
            % TODO: include plots!

        \subsection{Calculating the Wavefunction $\psi(x,0)$}\label{subsec:calculating-the-wavefunction}
            Using the Fourier transform given by equation~\ref{eq:psi-fourier}, the wavefunction at $t=0$ is given by
            \begin{align}
                \psi(x,0)&=\frac{1}{\sqrt{2\pi}}\int_{-\infty}^\infty A(k)e^{ikx}\dd{x}\\
                &=\frac{1}{\sqrt{\Delta k\sqrt{\pi}}\sqrt{2\pi}}\int_{-\infty}^\infty e^{-\frac{(k-k_0)^2}{2\Delta k^2}}e^{ikx}\dd{k},
            \end{align}
            where in the second line we inserted the square root of equation~\ref{eq:gaussian-wavenumber-probability-density} for $A(k)$.
            To integrate this we use the standard result for Gaussian integrals:
            \begin{equation}\label{eq:gaussian-standard-integral}
                \int_{-\infty}^\infty e^{-\alpha q^2}e^{-\beta q}\dd{q}=\sqrt{\frac{\pi}{\alpha}}e^{\frac{\beta^2}{4\alpha}}.
            \end{equation}
            With $q=k-k_0$ (so $\dd{q}=\dd{k}$), we get
            \begin{align}
                \psi(x,0)&=\frac{1}{\sqrt{\Delta k\sqrt{\pi}}\sqrt{2\pi}}\int_{-\infty}^\infty e^{-\frac{q^2}{2\Delta k^2}}e^{i(q+k_0)x}\dd{q}\\
                &=\frac{1}{\sqrt{\Delta k\sqrt{\pi}}\sqrt{2\pi}}e^{ik_0x}\int_{-\infty}^\infty e^{-\frac{q^2}{2\Delta k^2}}e^{iqx}\dd{q}.
            \end{align}
            We can now use the standard integral with $\alpha=\frac{1}{2\Delta k^2}$ and $\beta=-ix$ to get
            \begin{align}
                \psi(x,0)&=\frac{\sqrt{2\pi\Delta k^2}}{\sqrt{\Delta k\sqrt{\pi}}\sqrt{2\pi}}e^{ik_0x}e^{-\frac{\Delta k^2x^2}{2}}\\
                &=\sqrt{\frac{\Delta k}{\sqrt{\pi}}}e^{ik_0x}e^{-\frac{x^2\Delta k^2}{2}}.
            \end{align}

            What is the probability density at $t=0$?
            We can calculate this directly by taking the magnitude square:
            \begin{equation}
                \abs{\psi(x,0)}^2=\frac{\Delta k}{\sqrt{\pi}}e^{-x^2\Delta k^2}=\frac{\Delta k}{\sqrt{\pi}}e^{-\frac{x^2}{\Delta x_0^2}},
            \end{equation}
            where we have defined $\Delta x_0=\frac{1}{\Delta k}$ as the width at $t=0$.
            So the probability density is also a Gaussian, this time centered at $x=0$, with width $\Delta x_0=\frac{1}{\Delta k}$.
            % TODO: check that it is normalised

            Notice that $\Delta x_0\Delta k=1$, which can be expressed in term of momentum via the de Broglie relation:
            \begin{equation}
                \Delta x_0\Delta p=\hbar.
            \end{equation}
            This is a manifestation of the uncertainty principle.
            The factor of $1$ arises from our definition of width of a Gaussian.
            Notice that the width of the position probability density is inversely proportional to the width of the wavenumber probability density.
            This reflects the fact that due to the uncertainty principle, if the uncertainty in momentum $\Delta k$ is small then the uncertainty in position $\Delta x_0$ must be wide and vice versa.

        \subsection{Time-evolution of the Gaussian Wave Packet}\label{subsec:time-evolution-of-the-gaussian-wave-packet}
            As we found in section~\ref{sec:free-particle:subsec:wave-packets}, the time evolution of our wave packet is given by
            \begin{equation}
                \psi(x,t)=\frac{1}{\sqrt{2\pi}}\int_{-\infty}^\infty A(k)e^{i(kx-\omega t)}\dd{k}.
            \end{equation}
            To evaluate this, we need to keep in mind that $\omega$ is a function of $k$ through the Einstein and de Broglie relations.
            Specifically, for a free particle only we have $E=\frac{p^2}{2m}$, which gives
            \begin{align}
                E&=\frac{p^2}{2m}\\
                \hbar\omega&=\frac{\hbar^2k^2}{2m}\\
                \implies\omega(k)&=\frac{\hbar k^2}{2m}.\label{eq:gaussian-dispersion}
            \end{align}
            This is our \textbf{dispersion relation} for matter waves.
            Note that the $\omega$ is not proportional to $k$, so matter waves are \textbf{dispersive}.
            This means that different wavelength will travel at different speeds.
            The phase velocity is given by
            \begin{equation}
                v_\text{ph}(k)=\frac{\omega}{k}=\frac{\hbar k}{2m},
            \end{equation}
            so shorter wavelengths (larger $k$) travel faster.
            This makes physical sense as large $k$ corresponds to large momentum.
            Meanwhile the group velocity for a wave packet centered at $k_0$ is
            \begin{equation}
                v_\text{gr}(k)=\left.\dv{\omega}{k}\right|_{k_0}=\frac{\hbar k_0}{2m}=2v_\text{ph}(k_0).
            \end{equation}
            This is the speed that the envelope of the wave packet will travel at.
            Combining these ideas together, we expect that the wave packet will be smudged out over time, as some wavelengths in it travel faster and some slower.

            Let's see if we can show this explicitly.
            Substituting the square root of equation~\ref{eq:gaussian-wavenumber-probability-density} for $A(k)$ and equation~\ref{eq:gaussian-dispersion} for $\omega(k)$ into the integral, we get
            \begin{equation}
                \psi(x,t)=\frac{1}{\sqrt{\Delta k\sqrt{\pi}}\sqrt{2\pi}}\int_{-\infty}^\infty e^{-\frac{(k-k_0)^2}{2\Delta k^2}}e^{i\left(kx-\frac{\hbar k^2}{2m}t\right)}\dd{k}.
            \end{equation}
            We would like to use the standard integral~\ref{eq:gaussian-standard-integral} for Gaussians again.
            Making the same substitution as in section~\ref{subsec:calculating-the-wavefunction} above ($q=k-k_0$) gives
            \begin{align}
                \psi(x,t)&=\frac{1}{\sqrt{\Delta k\sqrt{\pi}}\sqrt{2\pi}}\int_{-\infty}^\infty e^{-\frac{q^2}{2\Delta k^2}}e^{i(q+k_0)x}e^{-\frac{i\hbar(q+k_0)^2}{2m}t}\dd{q}\\
                &=\frac{1}{\sqrt{\Delta k\sqrt{\pi}}\sqrt{2\pi}}\int_{-\infty}^\infty e^{-\frac{q^2}{2\Delta k^2}}e^{iqx}e^{ik_0x}e^{-\frac{i\hbar q^2}{2m}t}e^{-\frac{i\hbar qk_0}{m}t}e^{-\frac{i\hbar k_0^2}{2m}t}\dd{q}.
            \end{align}
            We can simplify the notation a bit by defining $\omega_0=\frac{\hbar k_0^2}{2m}$, the frequency of the state with wavenumber $k_0$, and using the group velocity $v_\text{gr}(k_0)=\frac{\hbar k_0}{m}$.
            The fifth and sixth exponentials can be rewritten in terms of these quantities, and then the third and fifth exponentials can be taken out of the integral to get
            \begin{equation}
                \psi(x,t)=\frac{1}{\sqrt{\Delta k\sqrt{\pi}}\sqrt{2\pi}}e^{i(k_0x-\omega_0 t)}\int_{-\infty}^\infty e^{-\left(\frac{1}{2\Delta k^2}+\frac{i\hbar}{2m}t\right)q^2}e^{-i(v_\text{gr}(k_0)t-x)q}\dd{q}.
            \end{equation}
            Now we can use the standard integral with
            \begin{align}
                \alpha&=\frac{1}{2\Delta k^2}+\frac{i\hbar}{2m}t\\
                \beta&=i(v_\text{gr}(k_0)t-x),
            \end{align}
            to get
            \begin{align}
                \psi(x,t)&=\frac{1}{\sqrt{\Delta k\sqrt{\pi}}\sqrt{2\pi}}\sqrt{\frac{\pi}{\frac{1}{2\Delta k^2}+\frac{i\hbar}{2m}t}}e^{i(k_0x-\omega_0 t)}e^{-\frac{(x-v_\text{gr}(k_0)t)^2}{4\left(\frac{1}{2\Delta k^2}+\frac{i\hbar}{2m}t\right)}}\\
                &=\sqrt{\frac{\Delta k}{\sqrt{\pi}\left(1+\frac{i\hbar\Delta k^2t}{m}\right)}}e^{i(k_0x-\omega_0 t)}e^{-\frac{\Delta k^2(x-v_\text{gr}(k_0)t)^2}{2\left(1+\frac{i\hbar\Delta k^2t}{m}\right)}}.
            \end{align}

            Calculating the probability density, we get
            \begin{equation}
                \abs{\psi(x,t)}^2=\frac{\Delta k}{\sqrt{\pi}\sqrt{1+\frac{\hbar^2\Delta k^4t^2}{m^2}}}e^{-\frac{\Delta k^2(x-v_\text{gr}(k_0)t)^2}{1+\frac{\hbar^2\Delta k^4t^2}{m^2}}}.
            \end{equation}
            This is quite a complicated expression, but there are a few key things we can pick out that tell us in simple terms how the wave packet behaves.
            The structure of the probability density is basically
            \begin{equation}
                \abs{\psi(x,t)}^2=C(t)e^{-\frac{(x-x_0(t))^2}{\Delta x(t)^2}},
            \end{equation}
            so we see that the height, width, and centre of the wave packet all change with time while retaining a Gaussian form.
            
            Comparing the two expressions, the centre of the wave packet $x_0$ evolves as
            \begin{equation}
                x_0(t)=v_\text{gr}(k_0)t,
            \end{equation}
            so the position of the peak of the wavepacket moves to the right with a constant speed, the speed of the central wavenumber (or equivalently, the speed of the envelope).
            This makes sense, there are no forces acting on the particle so its speed stays constant with time.
            % TODO: calculate <x(t)> as an exercise to show that the peak position, which is <x>, follows the classical trajectory

            The width of the wave packet is
            \begin{equation}
                \Delta x(t)=\frac{1}{\Delta k}\sqrt{1+\frac{\hbar^2\Delta k^4t^2}{m^2}}=\Delta x_0\sqrt{1+\frac{\hbar^2\Delta k^4t^2}{m^2}},
            \end{equation}
            where we have reintroduced $\Delta x_0=\frac{1}{\Delta k}$, the width of the wave packet at $t=0$ from before.
            Thus the width $\Delta x$ increases over time just like we predicted.

            The height has the form
            \begin{equation}
                C(t)=\frac{1}{\sqrt{\pi}\Delta x(t)},
            \end{equation}
            so since the width increases with time, the height decreases.
            This makes sense, as to preserve the normalisation of the wave packet the amplitude must decrease as it spreads out.

            Notice that in the limit $t\to 0$, we get back our previous results for the initial wave packet $\psi(x,0)$.
            If $\Delta k$ is larger at $t=0$, i.e. if the uncertainty in momentum $\Delta p$ is larger, the the wavepacket spreads out more slowly.

            Does the uncertainty in momentum $\Delta p$ increase over just like $\Delta x$?
            From equation~\ref{eq:psi-fourier-momentum}, we have
            \begin{align}
                \psi(x,t)&=\frac{1}{\sqrt{2\pi\hbar}}\int_{-\infty}^\infty\phi(p,t)e^{\frac{i}{\hbar}px}\dd{p}\\
                &=\frac{1}{\sqrt{2\pi\hbar}}\int_{-\infty}^\infty\phi(p,0)e^{\frac{i}{\hbar}(px-Et)}\dd{p}.
            \end{align}
            For the free particle, we have $E=\frac{p^2}{2m}$ and so
            \begin{equation}
                \psi(x,t)=\frac{1}{\sqrt{2\pi\hbar}}\int_{-\infty}^\infty\phi(p,0)e^{-\frac{i}{\hbar}\frac{p^2}{2m}t}e^{\frac{i}{\hbar}px}\dd{p}.
            \end{equation}
            Hence we have that
            \begin{equation}
                \abs{\phi(p,t)}^2=\abs{\phi(p,0)e^{-\frac{i}{\hbar}\frac{p^2}{2m}t}}^2=\abs{\phi(p,0)}^2,
            \end{equation}
            so the momentum probability density remains the same over time and $\Delta p$ does not change.
            This is not a general fact, it only holds for the free particle.
            We will find that when there are forces present, $\Delta p$ does indeed increase over time.
            In that sense, this is sort of a quantum analogue of Newton's first law.

            % http://tinyurl.com/wave-packet
    
\end{document}
