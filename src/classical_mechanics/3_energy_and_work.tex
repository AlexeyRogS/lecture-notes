\documentclass[../classical_mechanics.tex]{subfiles}

\begin{document}

    \section{The Conservation of Energy}
        % TODO: introduce kinetic energy as integral of momemtum?

        \paragraph{}
        % TODO move some of the next section into here
        Consider the gravitational force $\vec{F}=-mg\khat$. Then
        \begin{align}
            \int_{t_1}^{t_2}\frac{\mathrm{d}}{\mathrm{d}t}\left(\frac{1}{2}mv^2\right)\mathrm{d}t&=\frac{1}{2}m(v(t_2)^2-v(t_1)^2)\\
            &=\int_{t_1}^{t_2}\vec{F}\cdot\vec{v}\mathrm{d}t\\
            &=-\int_{t_1}^{t_2}mgv_z\mathrm{d}t\\
            &=-mg(z(t_2)-z(t_1)),
        \end{align}
        and hence,
        \begin{equation}
            \frac{1}{2}mv_1^2+mgz_1=\frac{1}{2}mv_2^2+mgz_2.
        \end{equation}
        So this quantity is \textbf{constant} over the path of the object (since $t_1$ and $t_2$ were arbitrary).
        If we write
        \begin{equation}
            K=\frac{1}{2}mv^2, \quad U_g=mgz,
        \end{equation}
        then we have
        \begin{equation}
            E=K+U_g.
        \end{equation}
        \begin{example}
            Consider a pendulum on the end of a string. What is the maximum speed that the pendulum attains as it swings?
            % TODO: write this
        \end{example}

    \section{Work-Energy Theorem}
        \paragraph{}
        Lets calculate the rate of change of velocity squared.
        \begin{align}
            \frac{\mathrm{d}v^2}{\mathrm{d}t}&=\frac{\mathrm{d}\vec{v}}{\mathrm{d}t}\cdot\vec{v}+\vec{v}\cdot\frac{\mathrm{d}\vec{v}}{\mathrm{d}t}\\
            &=2\vec{v}\cdot\frac{\mathrm{d}\vec{v}}{\mathrm{d}t}\\
            &=2\vec{v}\cdot\frac{\vec{F}}{m},
        \end{align}
        where in the last line we have used Newton II. Thus,
        \begin{equation}
            \frac{\mathrm{d}}{\mathrm{d}t}\left(\frac{1}{2}mv^2\right)=\vec{F}\cdot{v}.
        \end{equation}
        We define the quantity in parentheses as the kinetic energy.
        \begin{definition}
            \textbf{Kinetic energy} is given by
            \begin{equation}
                K=\frac{1}{2}mv^2.
            \end{equation}
        \end{definition}
        Let's see how the kinetic energy changes for a given force. We can find the change in kinetic energy by integrating:
        \begin{align}
            \int_{t_1}^{t_2}\frac{\mathrm{d}}{\mathrm{d}t}\left(\frac{1}{2}mv^2\right)\mathrm{d}t&=\frac{1}{2}m(v(t_2)^2-v(t_1)^2)\\
            &=\Delta K=\int_{t_1}^{t_2}\vec{F}\cdot\vec{v}\mathrm{d}t.
        \end{align}
        For a constant force, we can take $\vec{F}$ out of the integral.
        \begin{align}
            \Delta K &= \vec{F}\cdot\int_{t_1}^{t_2}\vec{v}\mathrm{d}t\\
            &=\vec{F}\cdot\vec{d},
        \end{align}
        where $\vec{d}=\vec{r}_2-\vec{r}_1$.
        We call this quantity $\vec{F}\cdot\vec{d}$ the \textbf{work}, and give it the symbol $W$.

        \paragraph{}
        This result that $\Delta K=W$ is known as the work-energy theorem and can be extended to a general force which changes with time.
        In this case, we writ ethe work as the integral over the infinitesimal work done over an infinitesimal part of the path, $\mathrm{d}W=\vec{F}\cdot\mathrm{d}\vec{r}$.
        \begin{theorem}[Work-Energy Theorem]
            The net work on an object is equal to the change in its kinetic energy.
            \begin{align}
                W&=\int_\mathrm{path}\mathrm{d}W\\
                &=\int_{r_1}^{r_2}\vec{F}\cdot\mathrm{d}\vec{r}\\
                &=\int_{t_1}^{t_2}\vec{F}\cdot\frac{\mathrm{d}\vec{r}}{\mathrm{d}t}\mathrm{d}t\\
                % TODO: could also write F as d(mv)/dt (NII)
                &=\int_{t_1}^{t_2}\vec{F}\cdot\vec{v}\mathrm{d}t\\
                &=\int_{t_1}^{t_2}\frac{\mathrm{d}}{\mathrm{d}t}\left(\frac{1}{2}mv^2\right)\mathrm{d}t=\Delta K.
            \end{align}
        \end{theorem}
        The kinetic energy depends on the speed of the object, so if the net work is $>0$ then the object must have sped up.
        Likewise, if the net work is $<0$, the object has slowed down.
        If the net work is 0, the object must be at the same speed that it started at.
        % TODO: mention that if we have an object moving only under gravity or spring force then we can express work done as the negative of the change in potential energy
        % e.g. object falls, potential energy decreases, positive work done, object accelerates which increases kinetic energy
        \begin{example}
            Consider a chain of length $L$ and total mass $m$ hanging over the edge of a frictionless table.
            % TODO: write this example from notes
        \end{example}
        \paragraph{}
        Now let's look at a mass on a spring. The restoring force on the mass always acts opposite to displacement. Hooke's law says
        \begin{equation}
            F_s = -k(x-x_0),
        \end{equation}
        % TODO: add diagram of system, free-body diagram and force vs displacement
        where $k$ is the spring constant, $x_0$ is the equilibrium position of the spring.
        Then the work done by the spring force is
        \begin{align}
            W_s = \int_{x_0}^{x_f}F_s(x)\mathrm{d}x &= -\int_{x_0}^{x_f}k(x-x_0)\mathrm{d}x\\
            &=-\frac{1}{2}k(x_f-x_0)^2.
        \end{align}
        Note that the work done is always negative no matter if the displacement is positive or negative because the force points in the opposite direction.
        Now we define the spring potential energy as
        \begin{equation}
            U_s = -W_s = \frac{1}{2}k(x_f-x_0)^2,
        \end{equation}
        Then by the work-energy theorem we have
        \begin{equation}
            \Delta K = W_s = -\Delta U_s,
        \end{equation}
        so
        \begin{equation}
            K + U_s = E
        \end{equation}
        is constant.
        % TODO: mention the sign of work. Given by F.dr, but also note that positive work ON an object INCREASES its kinetic energy
        % TODO: mention the relation between NIII and sign of work

    \section{Friction}
        % move this to an earlier chapter? this section does not need energy/work to introduce its ideas
        \paragraph{}
        Friction is a very complicated process which occurs on a miroscopic scale, so in order to model in on a macroscopic scale we must use simplified empirical laws.
        In general, friction is a force which opposes change in motion. Hence if a force is applied parallel to a surface, then the frictional force will be antiparallel to this, perpendicular to the normal force.

        \paragraph{}
        Satic friction appears when two objects are motionless with respect to one another.
        If a force is applied between the two objects and they don't move, there must be a frictional force opposing the motion.
        % TODO: add a diagram
        \begin{equation}
            \vec{f}_s=-\vec{F}_{app}.
        \end{equation}
        As the applied force gets larger, the static friction must get larger to preserve equilibrium, until a maximum limit is reached and the object starts moving.
        \begin{definition}
            The maximum magnitude of \textbf{static friction} is given by
            \begin{equation}
                f_{s,max}=\mu_s\abs{\vec{N}}.
            \end{equation}
            Hence,
            \begin{equation}
                0\leq\abs{\vec{f}_s}\leq f_{s,max}.
            \end{equation}
        \end{definition}

        \paragraph{}
        Kinetic friction opposes the motion of two surfaces sliding against each other.
        \begin{definition}
            \textbf{Kinetic friction} is given by
            \begin{equation}
                \vec{f}_k=-\mu_k\abs{\vec{N}}\hat{v}.
            \end{equation}
        \end{definition}
        % TODO: include discussion of critical angle
        \begin{example}
            Find the stopping distance of a block sliding down a slope.
            % TODO: write this
            % TODO: solve this example using only energy conservation (or lack thereof in this case)
        \end{example}
        % TODO: show that total energy is not conserved when friction is considered

    \section{Conservative Forces}
        \paragraph{}
        We have seen that in some cases the total energy is conserved and in others it is not.
        Consider the definition of work:
        \begin{equation}
            W=\int_A^B \vec{F}\cdot\mathrm{d}\vec{r}.
        \end{equation}
        % TODO: include diagrams for this
        This is a line integral, so in general the value of the integral depends on the path chosen for integration, which in this case corresponds to the path of the object through space.
        However, we have seen for the case of gravity and the spring force that the work done depends \textit{only} on the initial and final positions; the integral is \textit{path independent}.
        We also saw that in this case we can define a potential energy function for which
        \begin{equation}
            W=\int_A^B \vec{F}\cdot\mathrm{d}\vec{r} = U(A) - U(B),
        \end{equation}
        where $F=-\frac{\mathrm{d}U}{\mathrm{d}\vec{r}}$.
        % TODO: include discussion of work around a closed path being 0
        This is actually consequence of a generalisation of the fundamental theorem of calculus.
        We call forces which have this property \textbf{conservative forces}.
        They are called this because by the work-energy theorem:
        \begin{equation}
            W = U(A) - U(B) = -\Delta U = \Delta K,
        \end{equation}
        and hence the total energy is conserved.
        \begin{example}
            Consider a force $\vec{F}$ then the work done by this force along an infinitesimal displacement $\mathrm{d}\vec{r}=\mathrm{d}x\ihat+\mathrm{d}y\jhat+\mathrm{d}z\khat$ is
            \begin{equation}
                \mathrm{d}W = \vec{F}\cdot\mathrm{d}\vec{r}=F_x\mathrm{d}x+F_y\mathrm{d}y+F_z\mathrm{d}z.
            \end{equation}
            If $\vec{F}$ is conservative, then $\mathrm{d}W=-\mathrm{d}U$. By expanding the full differential of $U$ as
            \begin{equation}
                \mathrm{d}U=\frac{\partial U}{\partial x}\mathrm{d}x+\frac{\partial U}{\partial y}\mathrm{d}y+\frac{\partial U}{\partial z}\mathrm{d}z,
            \end{equation}
            we see by comparing coefficients that
            \begin{equation}
                F_x=-\frac{\partial U}{\partial x}, \quad F_y=-\frac{\partial U}{\partial y}, \quad F_z=-\frac{\partial U}{\partial z},
            \end{equation}
            so
            \begin{equation}
                \vec{F}=-\nabla U.
            \end{equation}
        \end{example}

        \paragraph{}
        For a non-conservative force where the line integral depends on the path taken, for example friction, there is no potential energy function and total energy is not conserved.
        Another way of viewing this is that under the action of conservative forces, the work done along a path is equal to the negative of the work done by reversing along the path.
        So we get back the energy that we put in.
        But in the case of a non-conservative force like friction, we don't regain energy we put in by moving an object along a path.

        \paragraph{}
        In a general system, an object may be under the influence of multiple forces which can be conservative or non-conservative.
        If we split the resultant force on the system into a conservative part and a non-conservative part: $\vec{F}=\vec{F}_\text{conservative}+\vec{F}_\text{non-conservative}$, then using the work energy theorem again we can write
        \begin{align}
            \Delta K = W &= \int_A^B \vec{F}\cdot\mathrm{d}\vec{r}\\
            &= \int_A^B \vec{F}_\text{conservative}\cdot\mathrm{d}\vec{r} + \int_A^B \vec{F}_\text{non-conservative}\cdot\mathrm{d}\vec{r}\\
            &= - \Delta U + W_\text{non-conservative}.
        \end{align}
        If we call the sum $K+U$ --- the sum of kinetic energy and the potential energy from conservative forces --- the \textbf{mechanical energy} $E_\text{mech}$, then we get
        \begin{equation}
            \Delta E_\text{mech} = W_\text{non-conservative}.
        \end{equation}
        In the case where the non-conservative force is friction, this work done is converted to heat, or \textbf{thermal energy}.
        So $W_\text{non-conservative} = -\Delta E_\text{thermal}$.
        This implies that we can write energy conservation as
        \begin{equation}
            \Delta E_\text{mech} + \Delta E_\text{thermal} = 0.
        \end{equation}
        This implies that $E_\text{mech}\leq 0$, so the total mechanical energy in a closed system can only decrease.
        This is related to the second law of thermodynamics.
        If our system is not isolated and is acted on by an external force, we can say
        \begin{equation}
            \Delta E_\text{mech} + \Delta E_\text{thermal} = W_\text{ext}
        \end{equation}
        % TODO: give some examples of open systems
        \begin{example}
            A 2000kg elevator cable snaps at a height of 20m above a spring with $k=10,000$Nm$^{-1}$.
            Taking into consideration that the friction of the shaft walls exert a constant force of 15,000N to resist the fall of the elevator, what is the maximum compression of the spring?
            % TODO: write this example
        \end{example}

        \paragraph{}
        % TODO: make this section more rigorous
        Consider the work done by the the gravitational force between two bodies as we move them closer or further apart.
        The gravitational force acts in the opposite direction to the displacement, so the infinitesimal amount of work done for an infinitesimal displacement is
        \begin{align}
            \mathrm{d}W&=\vec{F}_G\cdot\mathrm{d}\vec{r}\\
            &=-F_G\mathrm{d}r\\
            &=-\frac{Gm_1m_2}{r^2}\mathrm{d}r.
        \end{align}
        Then the total work done on an object by the gravitational force is
        \begin{align}
            W=\int\mathrm{d}W&=-\int_{r_1}^{r_2}\frac{Gm_1m_2}{r^2}\mathrm{d}r\\
            &=\left.\frac{Gm_1m_2}{r}\right|_{r_1}^{r_2}=\frac{Gm_1m_2}{r_2}-\frac{Gm_1m_2}{r_1}.
        \end{align}
        Since this only depends on the initial and final positions, the law of universal gravitation is a conservative force and we can define the change in \textbf{gravitational potential energy} as
        \begin{equation}
            \Delta U_G=-W=\frac{Gm_1m_2}{r_1}-\frac{Gm_1m_2}{r_2}.
        \end{equation}
        If we choose $U_G$ to be 0 at $r=\infty$, then
        \begin{equation}
            U_G(r)=-\frac{Gm_1m_2}{r}.
        \end{equation}
        % TODO: include discussion of choosing zero-point
        If we differentiate this with respect to $r$, we get the law of universal gravitation as expected.
        % TODO: introduce power
        \begin{example}
            % TODO: provide justification for solution to this example
            Consider four possible paths of an object falling that start and end at the same height.
            Order the paths in terms of the final kinetic energy when there is no friction.
            What changes if there is friction?
            \begin{figure}[H]
                \centering
                \begin{tikzpicture}[scale=5]
                    \draw[<-] (0,1) node[left] {$h$} |- (2, 0);

                    \draw[shift={(0.2,0.9)}] (-0.06,-0.05) rectangle (0.06,0.05);
                    \draw[->,dashed] (0.2,0.8) -- (0.2,0.5);
                    \node[below] at (0.2,0) {A};
                    
                    \draw[shift={(0.56,0.9)},rotate=-68.2] (-0.06,-0.05) rectangle (0.06,0.05);
                    \draw (0.45,1) -- (0.85,0);
                    \node[below] at (0.65,0) {B};

                    \draw[shift={(1,0.9)},rotate=-73.3] (-0.06,-0.05) rectangle (0.06,0.05);
                    \draw (0.9,1) -- (1.2,0);
                    \node[below] at (1.05,0) {C};

                    \draw[shift={(1.33,0.9)},rotate=-83] (-0.06,-0.05) rectangle (0.06,0.05);
                    \draw (1.25,1) parabola bend (1.6,-0.2) (1.8,0);
                    \node[below] at (1.6,0) {D};
                \end{tikzpicture}
            \end{figure}
            With no friction, the final velocity is the same for all paths because the change in gravitional potential energy $U_g$ is the same.
            With friction, $v_A>v_B>v_C>v_D$.
        \end{example}

    \section{Collisions}
        \paragraph{}
        A collision is an interaction between two objects over a short time interval.
        To solve these problems, we can use the concept of momentum conservation and energy conservation that we have been studying in the last two chapters.
        Consider two blocks sliding along a frictionles surface towards each other (1-dimensional problem).
        % TODO: include an explanation of how momentum is transferred from one block to another (through spring potential energy, etc.)
        The blocks have masses $m_1$, $m_2$ and velocities $v_1$, $v_2$ respectively.
        % TODO: include diagram to illustrate this
        What we want to find is the velocities of the blocks after the collision.
        To do this, we write the total momentum and kinetic energy before and after as
        \begin{align}
            \text{Before:}\quad& P=m_1v_1+m_2v_2\\
            & K=\frac{1}{2}m_1v_1^2+\frac{1}{2}m_2v_2^2\\
            \text{After:}\quad& P^\prime=m_1v_1^\prime+m_2v_2^\prime\\
            & K^\prime=\frac{1}{2}m_1v_1^{\prime 2}+\frac{1}{2}m_2v_2^{\prime 2}.
        \end{align}
        Total momentum is always conserved in collisions.
        On the other hand, depending on the forces involved during the collision, total kinetic energy may or may not be conserved.
        We call the case where it is conserved ``\textbf{elastic}'' and the case where it is not ``\textbf{inelastic}''. 

        \paragraph{}
        In the case of elastic collisions, where total kinetic energy is conserved, we can write
        \begin{align}
            m_1v_1+m_2v_2&=m_1v_1^\prime+m_2v_2^\prime\\
            \frac{1}{2}m_1v_1^2+\frac{1}{2}m_2v_2^2&=\frac{1}{2}m_1v_1^{\prime 2}+\frac{1}{2}m_2v_2^{\prime 2}
        \end{align}
        This is a system of two equations for two unknowns, $v_1^\prime$ and $v_2^\prime$.
        We can solving this system using algebra, and the solution is
        \begin{align}
            v_1^\prime&=\frac{m_1-m_2}{m_1+m_2}v_1+\frac{2m_2}{m_1+m_2}v_2\\
            v_2^\prime&=\frac{2m_1}{m_1+m_2}v_1+\frac{m_2-m_1}{m_1+m_2}v_2.
        \end{align}
        Does this make sense?
        To examine whether this answer makes physical sense we can take some limits and see what happens to the solution.
        Set $v_2=0$ and then consider the limit where $m_1\gg m_2$.
        In this case, $v_1^\prime\to v_1$ and $v_2^\prime\to 2v_1$.
        % TODO: prove the limits properly
        This is like a bowling ball colliding with a ping-pong ball, the bowling ball keeps on going and the ping ball gets deflected in the same direction with twice the speed.
        % TODO: include diagrams for these
        If the two masses are equal, $v_1^\prime=0$ and $v_2^\prime=v_1$, which is like a perfect billiard ball collision.
        On the other hand, if $m_1\ll m_2$, $v_1^\prime\to -v_1$ and $v_2^\prime\to 0$.
        This corresponds to a ping-pong ball hitting a bowling ball at rest. It bounces off with the same speed in the opposite direction while the bowling ball stays still.

        \paragraph{}
        If we transform the velocities into the centre of mass frame we get
        \begin{align}
            v_{1,\text{COM}}&=\frac{m_2(v_1-v_2)}{m_1+m_2}\\
            v_{2,\text{COM}}&=\frac{m_1(v_2-v_1)}{m_1+m_2}=-\frac{m_1}{m_2}v_{1,\text{COM}}\\
            v_{1,\text{COM}}^\prime&=-v_{1,\text{COM}}\\
            v_{2,\text{COM}}^\prime&=-v_{2,\text{COM}}.
        \end{align}
        % TODO: prove the bottom two formulae above
        So in the centre of mass frame, the two objects approach each other from opposite directions with velocities antiproportional to thei masses.
        After the collision, the magnitude of the velocities remains the same but they switch sign.

        \paragraph{}
        In an inelastic collision, we only have conservation of momentum since some energy is lost to non-conservative forces in the collision.
        To solve the system, we need another constraint on the velocities after the collision.
        In the case where the \textit{maximum} kinetic energy is lost, which is when the objects stick together and move as a single body with velocity $v^\prime=v_1^\prime=v_2^\prime$.
        % TODO: prove that this is the case where maximum kinetic energy is lost
        This reduces the two equations for two unknowns that we had to solve before to one equation for one unknown.
        \begin{align}
            m_1v_1+m_2v_2&=m_1v^\prime+m_2v^\prime\\
            \implies v^\prime&=\frac{m_1v_1+m_2v_2}{m_1+m_2}.
        \end{align}
        Notice that $v^\prime$ is simply the centre of mass velocity.
        So, if we transform into the centre of mass frame the final velocity is 0
        \begin{equation}
            v_\text{COM}^\prime=v^\prime-v_\text{COM}=0.
        \end{equation}
        % TODO: need to distinguish between v_COM, the velocity in the COM frame, and v_COM, the COM velocity
        This means that in a \textbf{perfectly inelastic} collision seen from the centre of mass frame, the objects approach each other with the same velocities as in the elastic case, but then come together at rest at the origin.
        \begin{example}
            Golf ball on a basketball.
            % TODO: write this example
        \end{example}

\end{document}
