\documentclass[../classical_mechanics.tex]{subfiles}

\begin{document}

    \section{The Wave Equation}\label{sec:the-wave-equation}
        A wave is a periodic variation or disturbance which travels at a well defined speed through space.
        How do we describe waves mathematically?
        Suppose $f(\chi)$ is some periodic function which takes a phase $\chi$ measured in radians (fractions of $2\pi$).
        Then we can describe the variation in space at a specific point in time as a snapshot:
        % TODO: rewrite this using cos instead of general function?
        \begin{equation}
            y(x)=Af\left(\frac{2\pi}{\lambda}x+\delta\right),
        \end{equation}
        where $\lambda$ is the \textbf{wavelength} of the wave (the spatial period).
        % TODO: include figures
        We can also describe the variation in amplitude at a single point in space over time:
        \begin{equation}
            y(t)=Af\left(\frac{2\pi}{T}t+\theta\right),
        \end{equation}
        where $T$ is the temporal period.

        To put these pictures together, we can consider the snapshot picture with a shift $x-vt$ where $v$ is the speed of the wave.
        Then we have
        \begin{align}
            y(x,t)&=Af\left(\frac{2\pi}{\lambda}(x-vt)+\phi\right)\\
            &=Af\left(\frac{2\pi}{\lambda}x-\frac{2\pi v}{\lambda}t+\phi\right)\\
            &=Af\left(\frac{2\pi}{\lambda}x-\frac{2\pi}{T}t+\phi\right)\\
            &=Af(kx-\omega t+\phi).
        \end{align} 
        Where we have defined the \textbf{wavenumber} (spatial frequency measured in radians/m) $k=2\pi/\lambda$ and recalled $v=f\lambda$, $f=1/T$ and $\omega=2\pi f$.
        \begin{example}
            Suppose we have a sinusoidal wave given by $y(x,t)=A\cos(kx-\omega t+\phi)$.
            What is the particle velocity at fixed position $x$?

            To solve this, we take the \textit{partial derivative} with respect to time (to keep $x$ constant)
            \begin{equation}
                \pdv{y(x,t)}{t}=A\omega\sin(kx-\omega t+\phi).
            \end{equation}
            Note that this is different to the propagation speed of the wave itself, which is constant.
        \end{example}

        For simplicity, let's consider a general right-travelling wave $f(x-vt)$. We can make a substitution $u=x-vt$.
        Then we get
        \begin{align}
            \pdv{f}{x}&=\pdv{f}{u}\pdv{u}{x}\\
            &=\pdv{f}{u}\\
            \implies\pdv[2]{f}{x}&=\pdv[2]{f}{u},\\
            \pdv{f}{t}&=\pdv{f}{u}\pdv{u}{t}\\
            &=-v\pdv{f}{u}\\
            \implies\pdv[2]{f}{t}&=v^2\pdv[2]{f}{u}.
        \end{align}
        If we do this same calculation with a left-travelling wave $f(x+vt)$, we get the same relation.
        Thus, by construction, the general solution to the differential equation
        \begin{equation}
            \pdv[2]{f}{t}=v^2\pdv[2]{f}{x},
        \end{equation}
        is $f(x,t)=f_l(x+vt)+f_r(x-vt)$.
        This linear differential equation is known as the \textbf{wave equation}.
        All functions which satisfy our definition of a wave solve this equation.
        % TODO: there are also non-periodic solutions

    \section{Superposition of Waves}\label{sec:superposition-of-waves}
        Since the wave equation is linear, the solutions follow the principle of \textbf{linear superposition}.
        This means that when multiple waves come together, the amplitude at every point in space and time is determined by the sum of all the waves at that point.
        \begin{example}
            Consider two sinusoidal waves with the travelling with the same frequency and direction.
            Then the superposition is given by
            \begin{align}
                y(x,t)&=A\cos(kx-\omega t)+A\cos(kx-\omega t)\\
                &=2A\cos(kx-\omega t).
            \end{align}
            So, the resultant wave has the same frequency and direction but double the amplitude.
            % TODO: include diagrams for these examples

            Now consider what happens if one of the waves has a phase shift of $\pi$ radians.
            The resultant wave is
            \begin{align}
                y(x,t)&=A\cos(kx-\omega t)+A\cos(kx-\omega t+\pi)\\
                &=A\cos(kx-\omega t)-A\cos(kx-\omega t)\\
                &=0.
            \end{align}
            The two waves cancel each other out completely.

            In the general case with a phase shift $\Omega$, we get
            \begin{align}
                y(x,t)&=A\cos(kx-\omega t)+A\cos(kx-\omega t+\Omega)\\
                &=2A\cos\left(kx-\omega t+\frac{\Omega}{2}\right)\cos\left(-\frac{\Omega}{2}\right)\\
                &=\underbrace{2A\cos\left(\frac{\Omega}{2}\right)}_{\text{Amplitude }\leq 2A}\underbrace{\cos\left(kx-\omega t+\frac{\Omega}{2}\right)}_{\text{Time-dependent part}}.
            \end{align}
            Note that we have used the identity $\cos(\alpha)+\cos(\beta)=2\cos\left(\frac{\alpha+\beta}{2}\right)\cos\left(\frac{\alpha-\beta}{2}\right)$.
        \end{example}

        Most superpositions of two sinusoidal functions do not have a nice simplification which is easily interpreted like this, but from these basic examples we can build an intuition for what happens when two waves meet.

        Now consider what happens if the waves still have the same frequency but are moving in opposite directions.
        In this case the superposition is
        % TODO: put this in an example?
        % TOOD: include diagrams for standing waves
        \begin{align}
            y(x,t)&=A\cos(kx-\omega t)+A\cos(kx+\omega t)\\
            &=2A\cos\left(\frac{2kx}{2}\right)\cos\left(-\frac{2\omega t}{2}\right)\\
            &=\underbrace{2A\cos(kx)}_{A(x)}\cos(\omega t).
        \end{align}
        So we have a spatially varying amplitude $A(x)$ multiplied by a time-dependent variation.
        This is known as a \textbf{standing wave}.
        \begin{example}
            In the case where one of the waves has a phase shift $\Omega$. The relation above becomes
            \begin{align}
                y(x,t)&=A\cos(kx-\omega t)+A\cos(kx+\omega t+\Omega)\\
                &=2A\cos\left(\frac{2kx+\Omega}{2}\right)\cos\left(-\frac{w\omega t+\Omega}{2}\right)\\
                &=2A\cos\left(kx+\frac{\Omega}{2}\right)\cos\left(\omega t+\frac{\Omega}{2}\right).
            \end{align}
        \end{example}
        % TODO: include discussion of phasors for modelling superposition and interference

        The points on the standing wave which don't move at all are known as \textbf{nodes}, and points which move up the twice the ampltide of the waves are called \textbf{antinodes}.
        Nodes are separated by half a wavelength.
        In a standing wave, the whole string oscillates in simple harmonic motion.

        The standing waves allowed in a one-dimensional region of length $L$ are given by
        \begin{equation}
            \lambda=\frac{2L}{p},\quad f_p=p\frac{v}{2L}=pf_1,
        \end{equation}
        where $p\in\mathbb{Z}$.

        Note that sometimes it is more convenient to express waves in terms of complex exponential functions according to Euler's formula:
        \begin{equation}
            e^{i\theta}=\cos\theta+i\sin\theta,
        \end{equation}
        so a general sinusoidal wave like above would be written as
        \begin{equation}
            y(x,t)=Ae^{i(kx-\omega t+\phi)}.
        \end{equation}
        We recover the trigonometric form (cosine in this case) by taking the real part of this function.
        For the standing wave example above, we have
        \begin{align}
            y(x,t)&=Ae^{i(kx-\omega t)}+Ae^{i(kx+\omega t)}\\
            &=Ae^{ikx}\left[e^{-i\omega t}+e^{i\omega t}\right]\\
            &=2Ae^{ikx}\cos(\omega t),
        \end{align}
        where in the last line we have used Euler's identity to get $e^{-i\theta}+e^{i\theta}=2\cos\theta$.
        Taking the real part of this, we recover $y(x,t)=2A\cos(kx)\cos(\omega t)$ which we found above.
        \begin{example}
            Given a periodic wave, what is the phase difference between two points on the wave separated by a distance $\Delta x$?
            \begin{equation}
                \Delta\phi=2\pi\frac{\Delta x}{\lambda}=k\Delta k.
            \end{equation}
            What is the phase difference between a single point over a interval of time $\Delta t$?
            \begin{equation}
                \Delta\phi=2\pi\frac{\Delta t}{T}=\omega\Delta t.
            \end{equation}
        \end{example}

    \section{Phase Velocity \& Group Velocity}\label{sec:phase-velocity-and-group-velocity}
        As we have seen, the speed you need to keep up with a point of constant phase along the wave is given by
        \begin{equation}
            v_\phi=f\lambda=\frac{\omega}{k}.
        \end{equation}
        This is known as the \textbf{phase velocity}.
        The dependence of $\omega$ on $k$ (or vice-versa) is called the \textbf{dispersion relation}.
        If the relationship is linear, i.e. if $v_\phi$ is constant, the wave is said to be dispersionless.
        Otherwise, the wave will undergo dispersion as different frequencies will travel at difference speeds.

        The \textbf{group velocity} is defined as
        \begin{equation}
            v_g=\dv{\omega}{k}.
        \end{equation}
        So if a wave is dispersionless, the phase velocity and group velocity will be the same.
        In the case where $v_g\neq v_\phi$, the group velocity is the speed that the wave envelope propagates.

    \section{Transverse Waves on a String}\label{sec:transverse-waves-on-a-string}
        Consider an infinite string under constant tension $T$.
        We will now show that the equation of motion of the string is the wave equation and derive the wave speed.
        Consider a short section of the string of length $\Delta x$.
        % TODO: include diagram of this
        We are assuming that the string has linear density $\mu$, zero stiffness, and we are ignoring the effects of gravity.
        Then assuming that there are only small displacements on the string, then $\pdv{y}{x}$ is small, so the angles $\theta_1$ and $\theta_2$ are also small.
        Hence we use the small angle approximation and say that $\cos\theta_1\approx\cos\theta_2\approx 1$.
        \begin{align}
            \sum F_x&=-\abs{\vec{T}_1}\cos\theta_1+\abs{\vec{T}_2}\cos\theta_2=0\\
            \implies\abs{T_{1,x}}&\approx\abs{T_{2,x}}\approx T.
        \end{align}
        From these we get that $T_{1,x}\approx -T$ and $T_{2,x}\approx T$.
        
        Now, using some trigonometry, notice that
        \begin{align}
            \left.\pdv{y}{x}\right|_x&=\frac{T_{1,y}}{T_{1,x}}\approx -\frac{T_{1,y}}{T}\\
            \left.\pdv{y}{x}\right|_{x+\Delta x}&=\frac{T_{2,y}}{T_{2,y}}\approx\frac{T_{2,y}}{T}.
        \end{align}
        Thus the net force in the $y$-direction is given by
        \begin{align}
            F_y&=T_{1,y}+T_{2,y}\\
            &=T\left(-\left.\pdv{y}{x}\right|_x+\left.\pdv{y}{x}\right|_{x+\Delta x}\right).
        \end{align}
        % TODO: replace this part with the better (in my opinion) derivation from Griffiths
        Using Newton's second law, we get
        \begin{align}
            F_y&=ma_y\\
            &=\mu\Delta xa_y\\
            &=\mu\Delta x\left.\pdv[2]{y}{t}\right|_{x+\frac{\Delta x}{2}}=\left(\left.\pdv{y}{x}\right|_{x+\Delta x}-\left.\pdv{y}{x}\right|_x\right)T.
        \end{align}
        Finally we divide by $\Delta x$ on both sides and take the limit as $\Delta x\to0$ to get
        \begin{align}
            \lim_{\Delta x\to0}\left(\mu\left.\pdv[2]{y}{t}\right|_{x+\frac{\Delta x}{2}}\right)&=\lim_{\Delta x\to0}\left[\frac{\left(\left.\pdv{y}{x}\right|_{x+\Delta x}-\left.\pdv{y}{x}\right|_x\right)T}{\Delta x}\right]\\
            \mu\pdv[2]{y}{t}&=T\pdv[2]{y}{x}\\
            \implies\pdv[2]{y}{t}&=\frac{T}{\mu}\pdv[2]{y}{x}.
        \end{align}
        This is the wave equations and we can see that for waves on a string, $v=\sqrt{\frac{T}{\mu}}$.

        What is the mechanical energy stored in a wave on a string?
        It will have two contributions, potential energy which depends on the displacement of every point and kinetic energy which depends on the velocity of every point.
        Consider a segment of the string of length $\mathrm{d}x$, mass $\mathrm{d}m=\mu\mathrm{d}x$.
        The infinitesimal contribution to the kinetic energy of the wave is given by
        \begin{equation}
            \mathrm{d}K=\frac{1}{2}\mathrm{d}mv_y^2=\frac{1}{2}\mu\mathrm{d}x\left(\pdv{y}{t}\right)^2.
        \end{equation}
        To get a value for this, we integrate it over some length $L$.
        \begin{equation}
            K=\frac{1}{2}\mu\int_{0}^{L}\left(\pdv{y}{t}\right)^2\mathrm{d}x.
        \end{equation}
        The potential energy is due to the stretching of the string.
        A segment of length $\mathrm{d}x$ stretches to a length $\mathrm{d}s$, and we can calculate the relationship between the two as follows:
        \begin{align}
            \mathrm{d}s&=\sqrt{\mathrm{d}x^2+\mathrm{d}y^2}\\
            &=\sqrt{\mathrm{d}x^2+\mathrm{d}x^2\left(\pdv{y}{x}\right)^2}\\
            &=\mathrm{d}x\sqrt{1+\left(\pdv{y}{x}\right)^2}\\
            &\approx\mathrm{d}x\left(1+\frac{1}{2}\left(\pdv{y}{x}\right)^2\right),
        \end{align}
        where in the last line we have used the Taylor expansion $\sqrt{1+u^2}\approx 1+\frac{1}{2}u^2$ when $u$ is small.
        This means we can calculate the potential energy as
        \begin{align}
            \mathrm{d}U&=T(\mathrm{d}s-\mathrm{d}x)\\
            &\approx\frac{1}{2}T\left(\pdv{y}{x}\right)^2\mathrm{d}x\\
            \implies U&=\frac{1}{2}T\int_{0}^{L}\left(\pdv{y}{x}\right)^2\mathrm{d}x.
        \end{align}
        \begin{example}
            Consider a sinusoidal wave $y=A\cos(kx-\omega t)$.
            What is the energy per unit wavelength?
            The partial derivatives are given by
            \begin{align}
                \pdv{y}{t}&=A\omega\sin(kx-\omega t)\\
                \pdv{y}{x}&=-Ak\sin(kx-\omega t),
            \end{align}
            so the infinitesimal contribution to the total energy is
            \begin{align}
                \mathrm{d}E&=\mathrm{d}K+\mathrm{d}U\\
                &=\frac{1}{2}\left[\mu\left(\pdv{y}{t}\right)^2+T\left(\pdv{y}{x}\right)^2\right]\mathrm{d}x\\
                &=\frac{1}{2}A^2\sin^2(kx-\omega t)(\mu\omega^2+Tk^2)\mathrm{d}x.
            \end{align}
            Note that $v=\frac{\omega}{k}=\sqrt{T}{\mu}$, so $Tk^2=\mu\omega^2$.
            Hence for a sinudoidal wave, the kinetic and potential energies are the same.
            The energy per unit wavelength is then
            \begin{align}
                E_\lambda&=\mu A^2\omega^2\int_{0}^{\lambda}\sin^2(kx)\mathrm{d}x\\
                &=\frac{1}{2}\lambda\mu A^2\omega^2.
            \end{align}
            Note that we choose to write the energy in terms of $\mu$ rather than $T$ because linear density is an easily measurable property whereas the tension is not.
            One important thing to mention is that the dependence on $A^2$ is actually general to all forms of waves, not just sinusoidal.
            We can calculate the power transmitted through a single point by the wave as
            \begin{align}
                P=E_\lambda f&=\frac{1}{2}\lambda f\mu A^2\omega^2\\
                &=\frac{1}{2}v\mu A^2\omega^2\\
                &=\frac{1}{2}\sqrt{\mu T}A^2\omega^2.
            \end{align}
        \end{example}

        We can calculate the power transmitted through the wave from first principles as well.
        Each string segment exerts a force and does work on the adjoining segments.
        For a point $x_0$ on the string, work is done on the string by to the right of $x_0$ by a tension force $T_y$ applied by the string to the left of $x_0$.
        % TODO: include a diagram for this
        Assuming motion in the $x$-direction is negligible, we have that the work done by the tension is $\dd{W}=T_y\dd{y}$.
        Since we are assuming $\pdv{y}{x}$ is small, $T_y=-T\pdv{y}{x}$.
        Then we have that the instantaneous power is
        \begin{align}
            P_\text{inst}(x,t)=\dv{W}{t}&=T_y\pdv{y}{t}\\
            &=-T\pdv{y}{x}\pdv{y}{t}\\
            &=\frac{T}{V}\left(\pdv{y}{t}\right)^2=\sqrt{\mu T}\left(\pdv{y}{t}\right)^2,
        \end{align}
        where in the last line we have used the wave equation.
        For a sinusoidal pattern of motion, we have
        \begin{equation}
            P_\text{inst}(x,t)=\sqrt{\mu T}A^2\omega^2\sin^2(kx-\omega t).
        \end{equation}
        The average value of $\sin^2$ is $\frac{1}{2}$, so the average power transmitted is
        \begin{equation}
            P_\text{avg}=\sqrt{\mu T}A^2\omega^2,
        \end{equation}
        which is what we derived before.

    \section{Boundaries, Transmission, and Reflection}\label{sec:boundaries-transmision-and-reflection}
        A boundary for a wave is a change in medium.
        When a wave encounters a boundary, some energy is transmitted across the boundary and some is reflected.
        Consider a pulse wave travelling along a string:
        \begin{itemize}
            \item At a \textit{fixed} end, the string exerts an upward force on the fixed end which pulls back down on the string according to Newton III.
            This generates an upside-down wave pulse travelling in the opposite direction (what we actually see while this is happening is a superposition of both pulses).
            \item At a \textit{free} end, the same thing happens except that the reflected pulse is the same way up.
            Because of the superposition, the free end reaches twice the peak amplitude of the pulse.
            \item When we have a mix between these two cases, for example a change in linear density of the string, we will see partial transmission and reflection.
        \end{itemize}

        % TODO: include a diagram for this
        To figure out how much gets reflected and transmitted, we must make some assumptions about what happens to the wave at the boundary.
        At the boundary $x_b$ we have:
        \begin{itemize}
            \item $y(x_b,t)$ must be continuous (no gaps in the string).
            \item $\pdv{y(x_b,t)}{x}$ must be continuous (no sharp kinks).
        \end{itemize}
        Therefore, we must have the following two conditions at the boundary (they do not have to hold anywhere else!):
        \begin{gather}
            y_i(x_b,t)+y_r(x_b,t)=y_t(x_b,t)\\
            \pdv{y_i(x_b,t)}{t}+\pdv{y_r(x_b,t)}{t}=\pdv{y_t(x_b,t)}{t},
        \end{gather}
        where subscript $i$, $r$, and $t$ represent the incident, reflected, and transmitted waves respectively.
        Solving for $y_r$ and $y_t$ using the wave equation, we get
        % TODO: show this
        \begin{align}
            y_r(x_b,t)&=\frac{v_2-v_1}{v_2+v_1}y_i(x_b,t)=ry_i(x_b,t)\\
            y_t(x_b,t)&=\frac{2v_2}{v_1+v_2}y_i(x_b,t)=\tau y_i(x_b,t),
        \end{align}
        where we have defined the \textbf{reflection coefficient} $r$ and the \textbf{transmission coefficient} $\tau$ as
        \begin{equation}
            r=\frac{v_2-v_1}{v_1+v_2},\quad\tau=\frac{2v_2}{v_1+v_2},
        \end{equation}
        where $v_1$ is the wave velocity in the left medium and $v_2$ is the velocity on the right.
        Note that $-1\leq r\leq 1$, and $0\leq\tau\leq 2$.
        These coefficients can also be defined in terms of the wave impedance $Z=\sqrt{\mu T}$:
        \begin{equation}
            r=\frac{Z_1-Z_2}{Z_1+Z_2},\quad\tau=\frac{2Z_1}{Z_1+Z_2}.
        \end{equation}
        Because $r$ and $\tau$ are defined as the ratios of the amplitudes of the reflected and transmitted waves to the incident wave, we have a constraint $\abs{r}+\abs{\tau}=1$.
        Note that this has nothing to do with energy conservation, it holds even when energy is not conserved!

        Consider a boundary where the density increases.
        The wave speed is slower on the other side of the boundary, $v_2<v_1$, so $r<0$ and $\tau<1$.
        The pulse also becomes narrower because the wave speed is slower.
        % TODO: include diagrams of these scenarios
        What about where the density decreases?
        The new wave speed is faster, $v_2>v_1$, so $r>0$ and $\tau>1$.
        The reflected pulse has the same width but with smaller amplitude and the transmitted pulse is broader because its speed is higher.
        The maths for these scenarios is general to \textit{all} shapes of waves.
        For periodic waves, the left-hand side of the boundary consists of the superposition of the incident and reflected waves.
        For the transmitted wave, since the frequency is fixed by the source the wavelength \textit{must} be the quantity to change.
        If energy is conserved, it is conserved at the boundary, so we also have the constraint
        \begin{equation}
            \abs{P_{\text{inst}_i}(x_b,t)}=\abs{P_{\text{inst}_r}(x_b,t)}+\abs{P_{\text{inst}_t}(x_b,t)}.
        \end{equation}
        Note that this is different to the equations above where incident and reflected were on the same side, here reflected and transmitted are on the same side.

    \section{Normal Modes and Fourier Series}\label{sec:normal-modes-and-fourier-series}
        Consider a standing wave $y(x,t)=Ae^{i(kx-\omega t)}+Be^{i(kx+\omega t)}$ on a string clamped at both ends.
        % TODO: include diagram of this
        Suppose the string runs from $x=0$ to $x=L$, then at the boundaries we have $y(0,t)=y(L,t)=0$.
        Applying the left boundary condition to the standing wave, we get
        \begin{gather}
            y(0,t)=\Re(Ae^{-i\omega t}+Be^{i\omega t})=0\\
            A\cos(\omega t)+B\cos(\omega t)=0\\
            \implies A=-B.
        \end{gather}
        This makes physical sense because as we have seen above, wave pulses invert at fixed boundaries.
        For the right boundary condition, we get
        \begin{align}
            y(L,t)&=\Re(A[e^{i(kL-\omega t)-e^{i(kL+\omega t)}}])\\
            &=\Re(Ae^{ikL}[\underbrace{e^{-i\omega t}-e^{i\omega t}}_{-2i\sin(\omega t)}])\\
            &=2A\sin(kL)\sin(\omega t)=0\\
            &\implies\sin(kL)=0\\
        \end{align}
        $\sin(kL)=0$ implies that $kL=n\pi$, so we can write the standing wave in the form
        \begin{equation}
            y(x,t)=A\sin(k_n x)\cos(\omega_n t+\phi_0),
        \end{equation}
        where
        \begin{equation}
            k_n=\frac{n\pi}{L},\quad\omega_n=\frac{n\pi v}{L}.
        \end{equation}
        Note that wavelength is related to angular wavenumber by $\lambda=\frac{2\pi}{k}$, so $\lambda_n=\frac{2\pi}{k_n}=\frac{2L}{n}$.
        These allowed wavenumbers/frequencies are the \textbf{normal modes} of vibration for the string clamped at both ends.

        Recall in chapter~\ref{chap:oscillations} we discussed exciting all the frequencies of a system at once using the impulse method.
        If we pluck the string, multiple normal modes are excited.
        The resulting motion of the string is a superposition of the normal modes, given by
        \begin{equation}
            y(x,t)=\sum_n A_n\sin(k_n x)\cos(\omega_n t).
        \end{equation}
        We have chosen $\cos$ instead of $\sin$ here for the temporal part because the displacement is maximal at $t=0$.
        It makes no difference to the physics which one we choose.
        At $t=0$, we have
        \begin{equation}\label{eq:waves:plucked-string-initial-conditions}
            y(x,0)=\sum_n A_n\sin\left(\frac{n\pi}{L}x\right).
        \end{equation}
        If we can calculate all the $A_n$'s, we can determine the subsequent motion of the string.
        But there may be infinitely many $A_n$'s!
        Luckily, we can use a mathematical tool called \textbf{Fourier series} which makes the calculation of all of them straightforward.

        If we have a periodic function (for simplicity we can assume the period is $2\pi$, since we can stretch or shrink it otherwise), it is (almost) always possible to represent it as an infinite series of sines and cosines.
        % TODO: clarify the almost
        \begin{equation}\label{eq:waves:fourier-series}
            f(x)=\frac{1}{2}a_0+\sum_{j=1}^\infty a_j\cos(jx)+\sum_{j=1}^\infty b_j\sin(jx).
        \end{equation}
        Using some convenient properties of integrals of sines and cosines, there is a simple method to calculate what the coefficients $a_j$ and $b_j$ should be for any function $f(x)$.
        We can see how this will help us solve our problem of representing a wave pulse as a sum of normal modes.

        Specifically, we will make use of the following three integrals:
        \begin{align}
            &\int_{-\pi}^\pi\sin(mx)\sin(nx)\dd{x}=\begin{cases}
                0\text{ if }m\neq n\\
                \pi\text{ if }m=n
            \end{cases}\\
            &\int_{-\pi}^\pi\cos(mx)\cos(nx)\dd{x}=\begin{cases}
                0\text{ if }m\neq n\\
                \pi\text{ if }m=n
            \end{cases}\\
            &\int_{-\pi}^\pi\sin(mx)\cos(nx)\dd{x}=0\quad\forall m,n.
        \end{align}
        % TODO: talk about orthogonality
        Now consider multiplying equation~\ref{eq:waves:fourier-series} by $\cos(nx)$ and integrating from $-\pi$ to $\pi$.
        We get
        \begin{align}
            \begin{split}
                \int_{-\pi}^\pi\cos(nx)f(x)\dd{x}&=\frac{1}{2}a_0\int_{-\pi}^\pi\cos(nx)\dd{x}+\sum_{j=1}^\infty a_j\int_{-\pi}^\pi\cos(nx)\cos(jx)\dd{x}\\
                &+\underbrace{\sum_{j=1}^\infty b_j\int_{-\pi}^\pi\sin(nx)\cos(jx)\dd{x}}_{=0}
            \end{split}\\
            &=\frac{1}{2}a_0\int_{-\pi}^\pi\cos(nx)\dd{x}+\sum_{j=1}^\infty a_j\int_{-\pi}^\pi\cos(nx)\cos(jx)\dd{x}.
        \end{align}
        If $n=0$ we get
        \begin{equation}
            \int_{-\pi}^\pi\cos(0)f(x)\dd{x}=\frac{1}{2}a_0\int_{-\pi}^\pi\cos(0)\dd{x}=\pi a_0.
        \end{equation}
        Whereas if $n>0$ we find
        \begin{equation}
            \int_{-pi}^\pi\cos(nx)f(x)\dd{x}=\sum_{j=1}^\infty a_j\int_{-\pi}^\pi\cos(nx)\cos(jx)\dd{x}.
        \end{equation}
        All the terms on the right-hand side are zero except the one where $j=n$, where the result will be $\pi a_n$.
        So we have found formula for $a_0$ and $a_n$, they are
        \begin{align}
            a_0&=\frac{1}{\pi}\int_{-\pi}^\pi f(x)\dd{x}\\
            a_n&=\frac{1}{\pi}\int_{-\pi}^\pi\cos(nx)f(x)\dd{x}.
        \end{align}
        Similarly, if we multiply by $\sin(nx)$ and integrate from $-\pi$ to $\pi$ we find
        \begin{equation}
            b_n=\frac{1}{\pi}\int_{-\pi}^\pi\sin(nx)f(x)\dd{x}.
        \end{equation}

        More generally, if a function has a period $P$, then we can adjust the periods of sine and cosine and the formulae for the coefficients as follows:
        \begin{align}
            f(x)&=\frac{1}{2}a_0+\sum_{j=1}^\infty a_j\cos\left(\frac{2\pi n}{P}x\right)+\sum_{j=1}^\infty b_j\sin\left(\frac{2\pi n}{P}x\right)\\
            a_0&=\frac{2}{P}\int_{-\frac{P}{2}}^{\frac{P}{2}}f(x)\dd{x}\\
            a_n&=\frac{2}{P}\int_{-\frac{P}{2}}^{\frac{P}{2}}\cos\left(\frac{2\pi n}{P}x\right)f(x)\dd{x}\\
            b_n&=\frac{2}{P}\int_{-\frac{P}{2}}^{\frac{P}{2}}\sin\left(\frac{2\pi n}{P}x\right)f(x)\dd{x}.
        \end{align}
        In fact, the integrals don't necessarily have to be from $-\frac{P}{2}$ to $\frac{P}{2}$, as long as they go over one full period.
        We can choose the most convenient range to integrate over.

        Note that if a function is odd ($f(-x)=-f(x)$), the Fourier series will \textit{only} contain sine terms.
        Likewise, if a function is even ($f(-x)=f(x)$) its Fourier series will \textit{only} contain cosine terms.
        This is because sine is an odd function and cosine is even.
        \begin{example}
            Find the Fourier series of a square wave of amplitude $d$ and period $P$, which has the form
            \begin{equation}
                f(x)=\begin{cases}
                    -d,\quad-\frac{P}{2}<x<0\\
                    d,\quad 0<x<\frac{P}{2}
                \end{cases}
            \end{equation}
            on the domain $\left[-\frac{P}{2},\frac{P}{2}\right]$.
            Outside of this domain it repeats periodically.
            % TODO: include a diagram of square wave

            We have defined the square wave above as an odd function, so there should be no cosine terms.
            We can show this explicitly by calculating the $a_n$ coefficients:
            \begin{align}
                a_n&=\frac{2}{P}\int_{-\frac{P}{2}}^{\frac{P}{2}}\cos\left(\frac{2\pi n}{P}x\right)f(x)\dd{x}\\
                &=\frac{2}{P}\left[\int_{-\frac{P}{2}}^0 -d\cos\left(\frac{2\pi n}{P}x\right)\dd{x}+\int_0^{\frac{P}{2}}d\cos\left(\frac{2\pi n}{P}x\right)\dd{x}\right]\\
                &=\frac{2d}{P}\underbrace{\left[\int_0^{\frac{P}{2}}\cos\left(\frac{2\pi n}{P}x\right)\dd{x}-\int_{-\frac{P}{2}}^0\cos\left(\frac{2\pi n}{P}x\right)\dd{x}\right]}_{=0\text{ since }\cos(-x)=\cos(x)}.
            \end{align}
            Now calculating the $b_n$ coefficients, we find
            \begin{align}
                b_n&=\frac{2}{P}\int_{-\frac{P}{2}}^{\frac{P}{2}}\sin\left(\frac{2\pi n}{P}x\right)f(x)\dd{x}\\
                &=\frac{2}{P}\left[\int_{-\frac{P}{2}}^0 -d\sin\left(\frac{2\pi n}{P}x\right)\dd{x}+\int_0^{\frac{P}{2}}d\sin\left(\frac{2\pi n}{P}x\right)\dd{x}\right]\\
                &=\frac{2d}{P}\left[\int_0^{\frac{P}{2}}\sin\left(\frac{2\pi n}{P}x\right)\dd{x}-\int_{-\frac{P}{2}}^0\sin\left(\frac{2\pi n}{P}x\right)\dd{x}\right]\\
                &=\frac{2d}{P}\left(\left[\frac{P}{2\pi n}\cos\left(\frac{2\pi n}{P}x\right)\right]_{-\frac{P}{2}}^0-\left[\frac{P}{2\pi n}\cos\left(\frac{2\pi n}{P}x\right)\right]_0^{\frac{P}{2}}\right)\\
                &=\frac{d}{\pi n}([1-\cos(-\pi n)]-[\cos(\pi n)-1])\\
                &=\frac{2d}{\pi n}[1-\cos(\pi n)].
            \end{align}
            If $n$ is even then $\cos(\pi n)=1$ and if $n$ is odd then $\cos(\pi n)=-1$, so we get
            \begin{equation}
                b_n=\begin{cases}
                    0\text{ if $n$ is even}\\
                    \frac{4d}{\pi n}\text{ if $n$ is odd.}
                \end{cases}
            \end{equation}
            Therefore the Fourier series for the square wave consists only of sine terms with odd $n$.
            The first few terms are
            \begin{equation}
                f(x)=\frac{4d}{\pi}\sin\left(\frac{2\pi}{P}x\right)+\frac{4d}{3\pi}\sin\left(\frac{6\pi}{P}x\right)+\frac{4d}{5\pi}\sin\left(\frac{10\pi}{P}x\right)+\dots.
            \end{equation}
            % TODO: show plot of Fourier series
        \end{example}

        Let's now apply this amazing mathematical technique to the motion of a plucked string clamped at both ends.
        We know that the initial displacement of the string is given by equation~\ref{eq:waves:plucked-string-initial-conditions}, and we want to find the $A_n$'s.
        However, the displacement of the string is not periodic in $x$ because the string is finite in length!
        How we can find a Fourier series?
        We can just find a Fourier series for a periodic function that matches our string in the range $0$ to $L$.
        % TODO: include diagram for this periodic extension
        Really, what we are asking is ``what normal modes are excited when we pluck the string'', so we actually want our Fourier series to contain only the normal modes.
        Since $\lambda_1=2L$, our periodic representation of the string \textit{must} have period $2L$ otherwise the Fourier series won't contain the fundamental mode.
        Note that it must also be an odd function, because the initial conditions for the string only contains sines.
        % TODO: include diagram of what fundamental mode looks like on domain
        With these constraints, we end up with the following function:
        \begin{equation}
            y(x,0)=\begin{cases}
                4d\frac{x}{L},\quad 0<x<\frac{L}{4}\\
                \frac{4d}{3}\left(1-\frac{x}{L}\right),\quad\frac{L}{4}<x<L,
            \end{cases}
        \end{equation}
        where $d$ is the maximum initial displacement of the string.
        Using the formula for the Fourier coefficients with $y^\prime(x)$ as the periodic extension of $y(x,0)$, we get
        \begin{align}
            A_n&=\frac{1}{L}\int_{-L}^L\underbrace{\sin\left(\frac{2\pi}{\lambda_n}x\right)}_\text{odd}\underbrace{y^\prime(x)}_\text{odd}\dd{x}\\
            &=\frac{2}{L}\int_0^L\sin\left(\frac{2\pi}{\lambda_n}x\right)y^\prime(x)\dd{x}\\
            &=\frac{2}{L}\int_0^L\sin\left(\frac{2\pi}{\lambda_n}x\right)y(x,0)\dd{x}.
        \end{align}
        The second line follows because two odd functions multiplied together make an even function, so we can cut the range of the integral in half and double the result.
        The last line follows because $y^\prime(x)=y(x,0)$ on the domain $[0,L]$, so actually only the initial displacement between $0$ and $L$ matters!
        Now we can find the $A_n$'s:
        \begin{equation}
            A_n=\frac{8d}{L^2}\int_0^{\frac{L}{4}}x\sin\left(\frac{n\pi}{L}x\right)\dd{x}+\frac{8d}{3L}\int_{\frac{L}{4}}^L\sin\left(\frac{n\pi}{L}x\right)\dd{x}-\frac{8d}{3L^2}\int_{\frac{L}{4}}^L x\sin\left(\frac{n\pi}{L}x\right)\dd{x}.
        \end{equation}
        We can solve these integrals using the following results
        \begin{align}
            \int\sin\left(\frac{n\pi}{L}x\right)\dd{x}&=-\frac{L}{n\pi}\cos\left(\frac{n\pi}{L}x\right)+c\\
            \int x\sin\left(\frac{n\pi}{L}x\right)\dd{x}&=\frac{L^2}{n^2\pi^2}\sin\left(\frac{n\pi}{L}x\right)-\frac{Lx}{n\pi}\cos\left(\frac{n\pi}{L}x\right).
        \end{align}
        So we find
        \begin{equation}
            \begin{split}
                A_n&=8d\left[\frac{1}{n^2\pi^2}\sin\left(\frac{n\pi}{4}\right)-\frac{1}{4\pi n}\cos\left(\frac{n\pi}{4}\right)\right]\\
                &+\frac{8d}{3}\left[\frac{1}{n\pi}\cos\left(\frac{n\pi}{4}\right)-\frac{1}{n\pi}\cos(n\pi)\right]\\
                &+\frac{8d}{3}\left[\frac{1}{n\pi}\cos(n\pi)+\frac{1}{n^2\pi^2}\sin\left(\frac{n\pi}{4}\right)-\frac{1}{4\pi n}\cos\left(\frac{n\pi}{4}\right)\right].
            \end{split}
        \end{equation}
        The cosine terms all cancel out, and we are left with
        \begin{equation}
            A_n=\frac{32d}{3n^2\pi^2}\sin\left(\frac{n\pi}{4}\right).
        \end{equation}
        $\sin\left(\frac{n\pi}{4}\right)$ follows the repeating sequence $\frac{1}{\sqrt{2}}$, $1$, $\frac{1}{\sqrt{2}}$, $0$, $-\frac{1}{\sqrt{2}}$, $-1$, $-\frac{1}{\sqrt{2}}$, $0$..., so the first few terms of the Fourier series are
        \begin{equation}
            \begin{split}
                y(x,0)=\frac{32d}{3\pi^2}\biggl[&\frac{1}{\sqrt{2}}\sin\left(\frac{\pi}{L}x\right)+\frac{1}{4}\sin\left(\frac{2\pi}{L}x\right)+\frac{1}{9\sqrt{2}}+\sin\left(\frac{3\pi}{L}x\right)\\
                &-\frac{1}{25\sqrt{2}}\sin\left(\frac{5\pi}{L}x\right)-\frac{1}{36}\sin\left(\frac{6\pi}{L}x\right)-\frac{1}{49\sqrt{2}}\sin\left(\frac{7\pi}{L}x\right)+\dots\biggr]
            \end{split}
        \end{equation}
        We can plot the \textbf{frequency spectrum} of $y(x,0)$, which is a plot showing $A_n$ against $n$.
        We see that the lowest $n$ are the largest components, and they can smaller as $n$ increases.
        The gaps in the frequency spectrum are where the modes have a node at the point where the string was plucked.
        This means that because of where we chose to pluck the string, the resulting vibration will not contain any of those frequencies at all.

        With damping effects, we find that all of the normal modes decay away, with the higher frequency modes decaying faster.

    \section{Sound Waves}\label{sec:sound-waves}
        So far we have been looking at transverse waves, where the displacement is perpendicular to the direction of wave travel.
        Examples of transverse waves in real life are waves on a string and light waves.
        Waves can also be \textbf{longitudinal}, where the displacement is along the direction of travel.
        Sound waves are longitudinal, with regions of high density along the direction of travel (compression) and low density (rarefaction).
        
        Let's look at some snapshots of longitudinal waves.
        We denote the displacement of a particle from its equilibrium position by $s(x,t)$.
        Then the particles oscillate back and forth along the direction of motion with SHM.
        % TODO: include a diagram of displacement with particle positions overlayed
        Where $s$ is positive, the particles are displaced to the right, and where $s$ is negative they are displaced to the left.
        The points of zero displacement represent pressure maxima and minima.

        It is a general rule for all mechanical waves that the wave speed is related to the ratio of restoring force to inertia.
        For liquids and gases, this is
        \begin{equation}
            v=\sqrt{B}{\rho},
        \end{equation}
        where $\rho$ is the density and $B$ is the \textbf{bulk modulus}, which is a measure of how easy it is to compress the medium, given by
        \begin{equation}
            B=-V\dv{P}{V}.
        \end{equation}
        In everday situations, we can approximate $\dv{P}{V}$ as $\frac{\Delta P}{\Delta V}$ i.e. the pressure change that accompanies a small volume change.
        We find that for air $B_\text{air}=\qty{1.42e5}{\newton\per\meter\squared}$, and for water $B_\text{water}=\qty{2.2e9}{\newton\per\meter\squared}$.
        The density of air will change in everyday situations depending on the temperature.
        At room temperature (\qty{293}{\kelvin}), we have $v=\qty{343}{\meter\per\second}$.
        More generally, we have $v\propto T$.

        The equations of pressure waves have the same form as transverse waves:
        \begin{equation}
            s(x,t)=A\cos(kx-\omega t+\phi_0).
        \end{equation}
        % TODO: justify this
        Can we get a formula for the variation of pressure across a sound wave?
        Along the direction of propogation, volume elements oscillate in SHM and pressure variations cause the volumes to change slightly.
        This is because the left and right-hand sides of undergo slightly different displacements.
        Thus by the definition of the bulk modulus, we get
        % TODO: include a diagram of this
        \begin{align}
            P(x,t)&=-B\pdv{s(x,t)}{x}\\
            &=BAk\sin(kx-\omega t+\phi_0)\\
            &=P_\text{max}\sin(kx-\omega t+\phi_0).
        \end{align}
        Note that this pressure is the excess pressure deviation from equilibrium, \textit{not} the absolute pressure in the fluid.
        From this, we can see that on the plot of $s(x,t)$, points of zero displacement with \textit{positive} derivative are the pressure \textit{maxima}, and those with \textit{negative} derivative are the pressure \textit{minima}.

        When we calculated the power transmitted by a transverse wave on a string, we could neglect the other directions and consider the problem in 1D.
        For a sound wave this is not possible, so we have to consider plane waves (waves that have constant value on a plane perpendicular to the direction of motion).
        Instead of power, we look at \textit{intensity} which is defined as power per unit area, and we measure it across surfaces perpendicular to the propagation of the wave.
        We have a force $F$ which does work along a distance $\dd{s}$, so $\dd{W}=F\dd{s}$, then the intensity is the power $\dv{W}{t}$ divided by the area which the force acts on $S$:
        \begin{align}
            I_\text{inst}&=\dv{}{t}\left(\frac{F}{S}\dd{s}\right)\\
            &=\frac{F}{S}\pdv{s}{t}\\
            &=P(x,t)\pdv{s}{t}\\
            &=-B\pdv{s}{x}\pdv{s}{t}\\
            &=BA^2k\omega\sin^2(kx-\omega t+\phi_0).
        \end{align}
        Note how similar the second last line looks to the equation for instantaneous power of a transverse wave on a string.
        Taking the time average, we get
        \begin{align}
            I_\text{avg}&=\frac{1}{2}BA^2k\omega\\
            &=\frac{1}{2}\frac{P_\text{max}^2}{\sqrt{\rho B}},
        \end{align}
        where in the second line we have substituted more physical quantities.
        The denominator $\sqrt{\rho B}$ is known as the \textbf{specific acoustic impedance} $Z$, and this can be used to define transmission and reflection coefficients as we did before.

    \section{Longitudinal Standing Waves}\label{sec:longitudinal-standing-waves}
        We now have two ways of describing longitudinal waves, by the displacement of the particles and by the pressure along the direction of travel.
        \begin{align}
            s(x,t)&=A\cos(kx-\omega t+\phi_0)\\
            P(x,t)&=P_\text{max}\sin(kx-\omega t+\phi_0).
        \end{align}
        We can see that if we have a longitudinal standing wave, displacement nodes will be pressure antinodes, and pressure nodes will be displacement antinodes.

        Let's look at the allowed wavenumbers for longitudinal waves in a pipe with closed and open ends.
        In the case with both ends closed, we will have displacement nodes at each end of the pipe (the medium cannot move through the end).
        Thus the normal modes will have the form
        \begin{equation}
            s_n(x,t)=A\sin(k_n x)\cos(\omega_n t+\phi_0),
        \end{equation}
        where $k_n=\frac{n\pi}{L}$, $\lambda_n=\frac{2L}{n}$, and $f_n=\frac{nv}{2L}=nf_1$.
        In the case where both ends are open, we have displacement antinodes at each end.
        This leads to normal modes of the form
        \begin{equation}
            s_n(x,t)=A\cos(k_n x)\cos(\omega_n t+\phi_0),
        \end{equation}
        with the same allowed wavenumbers.
        In the case of pipe with one and closed and one end open, there will be a displacement node at the closed end and a displacement antinode at the open end.
        Depending on which end of the pipe is open, we would use sine or cosine for the spatial part of the normal modes.
        The allowed wavenumbers are then $k_n=\frac{(2n-1)\pi}{2L}$, $\lambda_n=\frac{4L}{2n-1}$, $f_n=\frac{(2n-1)v}{4L}=(2n-1)f_1$.
        Note that in this case the fundamental mode has a wavelength of $4L$.

    \section{The Doppler Effect}\label{sec:the-doppler-effect}
        Consider a source at rest emitting spherical waves with frequency $f_s$ and speed $v$.
        What does a detector towards the source with speed $v_d$ see?
        % TODO: include a diagram for this
        In this situation, the wavefronts approach the detector with speed $v+v_d$, so the frequency as measured by the detector is
        \begin{equation}
            f_d=\frac{v+v_d}{\lambda}=f_s\frac{v+v_d}{v}.
        \end{equation}
        If the detector is moving away from the source, then the observed frequency will be
        \begin{equation}
            f_d=f_s\frac{v-v_d}{v}.
        \end{equation}
        This change in frequency due to a difference in velocity is known as the \textbf{Doppler effect}.

        What about a moving source and a stationary detector?
        % TODO: include diagram for this
        During one period $T_s$ of the source emitting, the waves will propagate a distance $vT_s$.
        The source itself moves a distance $v_sT_s$.
        If the source is moving towards the detector, then the observed wavelength is $\lambda_d=vT_s-v_sT_s=T_s(v-v_s)$, which gives an observed frequency of
        \begin{equation}
            f_d=\frac{v}{T_s(v-v_s)}=f_s\frac{v}{v-v_s}.
        \end{equation}
        If the source is moving away from the detector, then we get
        \begin{equation}
            f_d=f_s\frac{v}{v+v_s}.
        \end{equation}

        In the general case of a moving source and detector, we have
        \begin{equation}
            f_d=f_s\frac{v\pm v_d}{v\mp v_s},
        \end{equation}
        where we have the $\frac{+}{-}$ case when the source and detector are moving closer and the $\frac{-}{+}$ case when they are moving apart.

    \section{Wave Interference}\label{sec:wave-interference}
        When two waves come together the result is the sum of both waves at every point.
        A special case of superposition is created when two waves of the same frequency (coherent waves) meet, called \textbf{interference}.
        When the crests of two coherent waves meet, they add together and we get a crest which is larger than either of the initial ones, which is called \textbf{constructive interference}.
        When a crest of one wave meets a trough of another, they cancel out and we get a much smaller amplitude, or even nothing if the initial amplitudes match.
        This is \textbf{destructive interference}.
        Let's look at this mathematically.
        Consider two waves of the same amplitude and frequency travelling in the same direction in 1D.
        Lets look at the amplitude at a point $x_i$ which is a distance $x_1$ from the source of the first wave and a distance $x_2$ from the source of the second.
        The resulting amplitude is
        \begin{equation}
            y(x_i,t)=A\cos(kx_1-\omega t+\phi_{0,1})+A\cos(kx_2-\omega t+\phi_{0,2}).
        \end{equation}
        Now, we will make use of the addition formula
        \begin{equation}
            A\cos(\theta_1)+A\cos(\theta_2)=2A\cos\left(\frac{\theta_1+\theta_2}{2}\right)\cos\left(\frac{\theta_1-\theta_2}{2}\right).
        \end{equation}
        Then the amplitude at $x_i$ becomes
        \begin{align}
            y(x_i,t)&=2A\cos\left(\frac{k(x_1+x_2)-2\omega t+\phi_{0,1}+\phi_{0,2}}{2}\right)\cos\left(\frac{k(x_1-x_2)+\phi_{0,1}-\phi_{0,2}}{2}\right)\\
            &=2A\cos\left(\frac{\Delta\phi}{2}\right)\cos(k\bar{x}-\omega t+\bar{\phi}_0),
        \end{align}
        where we have defined the \textit{total} phase difference $\Delta\phi=k(x_2-x_1)+\phi_{0,2}-\phi_{0,1}$, the average distance from $x_i$ to each source $\bar{x}=\frac{x_1+x_2}{2}$, and the average initial phase $\bar{\phi}_0=\frac{\phi_{0,1}+\phi_{0,2}}{2}$.
        The maximum amplitude of the superposition is $2A\cos\left(\frac{\Delta\phi}{2}\right)$, so we see that this is maximised --- i.e. we get destructive interference --- for $\Delta\phi=2m\pi$ for integer $m$.
        Likewise the maximum amplitude goes to zero and we get destructive interference when $\Delta\phi=(2m+1)\pi$.

        To extend this to more than one dimension, we must keep in mind that wavefronts are circular in 2D and spherical in 3D.
        If we are very far away from the source, the curvature of the waves becomes negligible and we get \textbf{plane waves}.
        This is known as the \textbf{far-field limit}.

        Consider an colinear array of $n$ wave sources all separated by a distance $d$.
        % TODO: include a diagram of this
        Let's put ourselves in the far-field limit, so that all the incoming wavefronts from the array will be parallel.
        The \textbf{path difference} (the extra distance one wave has travelled relative to another) is given by $h=d\sin\theta$.
        Then the \textbf{phase difference} is just the path difference multiplied by the wavelength $\Delta\phi=kh=\frac{2\pi}{\lambda}d\sin\theta$.
        If this phase difference is a multiple of $2\pi$, we get constructive interference.
        This is given by the condition
        \begin{equation}
            d\sin\theta_m=m\lambda,
        \end{equation}
        where $\theta_m$ labels the angles where we get maxima in amplitude.
        The value of $m=0,1,2,...$ is called the \textbf{order} of the maxima.
        Inbetween the maxima, we get secondary maxima with amplitude given by
        \begin{equation}
            s(\theta)=\varepsilon\frac{\sin\left(n\frac{\Delta\phi}{2}\right)}{\sin\left(\frac{\Delta\phi}{2}\right)}=\varepsilon\frac{\sin\left(n\frac{kd\sin\theta}{2}\right)}{\sin\left(\frac{kd\sin\theta}{2}\right)},
        \end{equation}
        where $n$ is the number of sources and $\varepsilon$ is the amplitude of a single source.
        % TODO: prove this formula
        The intensity will be proportional to the square of the amplitude.
        \begin{equation}
            I(\theta)=I_1\frac{\sin^2\left(n\frac{kd\sin\theta}{2}\right)}{\sin^2\left(\frac{kd\sin\theta}{2}\right)},
        \end{equation}
        where $I_1$ is the intensity from a single source.
        % TODO: show plots of I against theta for 2, 3, 10 sources

        Constructive interference is quite often used to examine structures which are too small to observe with visible light, as was historically the case with \textbf{Bragg scattering}.
        Lawrence and William Henry Bragg used constructive interference of X-rays to measure the distance between atoms in crystals in the early 1910s.
        Consider a crystal structure with layers of atoms separated by a distance $d$.
        % TODO: include a diagram for this
        When X-rays enter the crystal, they scatter off the layers.
        Consider two parallel incident rays which reflect off the top two layers.
        The path difference between the two rays will be $2d\sin\theta$, so therefore we get constructive interference when the path difference is an integer multiple of the wavelength:
        \begin{equation}
            2d\sin\theta=n\lambda.
        \end{equation}

        Interference also gives rise to the phenomenon known as \textbf{diffraction}, which is where waves are observed to curve around apertures and obstructions in their path.
        Each part of the wavefront in the gap or around the barrier becomes a secondary source of spherical waves, and these waves interfere to give a curved interference pattern.
        This method of analysis is known as the \textbf{Huygens-Fresnel principle}.
        % TODO: include a diagram of this

        Consider a plane wave moving through a gap of height $d$.
        Let's look at a thin slice of the gap of length $\dd{y}$ at a distance $y$ from the middle as a point source.
        % TODO: include a diagram of this
        The amplitude due to this small slice at a far away point $P$ is $\dd{s}=\varepsilon_R\dd{y}\cos(kr-\omega t)$, where $r$ is the distance from the thin slice to $P$ and $R$ is the distance from the middle of the gap to $P$.
        % TODO: why? show this in more detail
        If $R\gg d$, then $r\approx R$ so $\varepsilon_R$ only depends on $R$.
        It can be shown that $r\approx R-y\sin\theta$, which we will need to use because small differences in the distance \textit{will} make a big difference for the phase of the wave at $P$.
        Integrating over the whole gap, we get
        \begin{align}
            s(\theta)=\int\dd{s}&=\varepsilon_R\int_{-\frac{d}{2}}^{\frac{d}{2}}\cos(k(R-y\sin\theta)-\omega t)\dd{y}\\
            &=2\varepsilon_R d\frac{sin\left(\frac{kd\sin\theta}{2}\right)}{kd\sin\theta}\cos(kR-\omega t).
        \end{align}
        The intensity is proportional to the time average of the square of the amplitude.
        \begin{equation}
            I(\theta)=4I_0\frac{\sin^2\left(\frac{kd\sin\theta}{2}\right)}{k^2d^2\sin^2\theta},
        \end{equation}
        where $I_0$ is the intensity at $\theta=0$.
        Note that in this situation minima occur at integer multiples of the wavelength, not maxima.

        If we superpose two sounds with a very similar frequency, we will hear a periodic variation in sound intensity.
        Consider two longitudinal sound waves:
        \begin{align}
            s_1(x,t)&=A\cos(k_1x-\omega_1t)\\
            s_2(x,t)&=A\cos(k_2x-\omega_2t),
        \end{align}
        where $k_1\approx k_2$ and $\omega_1\approx\omega_2$.
        Then using the addition formula, the superposition is
        \begin{align}
            s_1+s_2&=2A\cos\left(\frac{k_1+k_2}{2}x-\frac{\omega_1+\omega_2}{2}t\right)\cos\left(\frac{k_1-k_2}{2}x-\frac{\omega_1-\omega_2}{2}t\right)\\
            &=2A\cos(\bar{k}x-\bar{\omega}t)\cos(k_\text{mod}x-\omega_\text{mod}t),
        \end{align}
        where we have defined the average wavenumber and frequency $\bar{k}$ and $\bar{\omega}$, and a \textbf{modulation wavenumber} and \textbf{modulation frequency} $k_\text{mod}$ and $\omega_\text{mod}$.
        This superposition is a product of a wave with the average frequency of the two original ones and a low frequency modulation (since the two original frequencies are close, the modulation frequency is very low).
        % TODO: include diagram of this
        The intensity rises and falls twice per period, so the \textbf{beat frequency} is twice the modulation frequency, $f_\text{beat}=2f_\text{mod}=f_1-f_2$.

        Now consider the case where we allow the phase velocities of the waves to be different ($v_1=\frac{\omega_1}{k_1}\neq v_2=\frac{\omega_2}{k_2}$).
        Then we can write
        \begin{align}
            s_1(x,t)&=A\cos((k_0+\Delta k)x-(\omega_0+\Delta\omega)t)\\
            s_2(x,t)&=A\cos((k_0-\Delta k)x-(\omega_0-\Delta\omega)t),
        \end{align}
        and we get
        \begin{align}
            s_1+s_2&=2A\cos\left(\frac{2k_0}{2}x-\frac{2\omega_0}{2}t\right)\cos\left(\frac{2\Delta k}{2}x-\frac{2\Delta\omega}{2}t\right)\\
            &=2A\cos(k_0x-\omega_0t)\cos(\Delta kx-\Delta\omega t).
        \end{align}
        This is the product of a higher frequency wave with a lower frequency envelope.
        The higher frequency wave has phase speed $v_\text{crest}=\frac{\omega_0}{k_0}$ and the envelope has speed $v_\text{env}=\frac{\Delta\omega}{\Delta k}$.
        If $\omega\propto k$, then these speeds are always the same.
        However, if this is not the case then the wave crests travel at a different speed to the envelope.
        If the range of wavenumbers in a superposition is small, then the group velocity is the speed of the envelope (the largest amplitude) and we have $v_\text{gr}=\left.\dv{\omega}{k}\right|_{k_0}$.
        % TODO: write more detail in this section
        % TODO: talk about wavepackets and bandwidth theorem?
        
        % TODO: lots of sections in this chapter need rearranging, especially parts of the last section

\end{document}
