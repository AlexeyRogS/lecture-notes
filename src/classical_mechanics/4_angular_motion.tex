\documentclass[../classical_mechanics.tex]{subfiles}

\begin{document}

    \section{Circular Motion}
        \paragraph{}
        We have studied in great detail the mechanics of objects travelling in straight lines.
        Now we want to extend this to more general situations where objects can move along curved paths.
        The simplest case of a curved path is circular motion.
        Suppose we have an object moving in a circular path.
        Instead of describing its trajectory as a 2D vector in cartesian coordinates, it is much simpler to describe its trajectory in \textbf{polar coordinates}.
        % TODO: include diagram for this
        The position of the object in 2D space is described by the distance from the origin and the angle which the position vector makes with the $x$ axis.
        In circular motion, the distance from the centre is constant so the 2D motion is reduced to a 1D problem.
        We can then define an \textit{angular displacement} which is given in terms of the angle.
        \begin{definition}
            The \textbf{angular displacement} of an object is the difference in angle to the $x$ axis between two times $t_1$ and $t_2>t_1$.
            \begin{equation}
                \Delta\theta=\theta(t_2)-\theta(t_1)=\theta_2-\theta_1.
            \end{equation}
        \end{definition}
        Now consider the velocity of the object, analogously to the linear case, we can define the angular velocity as the rate of change of angular displacement.
        \begin{definition}
            The \textbf{instananeous angular velocity} of an object is defined as the time derivative of angular displacement.
            \begin{equation}
                \omega(t)=\lim_{\Delta t\to0}\frac{\theta(t+\Delta t)-\theta(t)}{\Delta t}=\dv{x(t)}{t}.
            \end{equation}
        \end{definition}
        Likewise, the angular acceleration is given by the rate of change of angular velocity.
        \begin{definition}
            The \textbf{instaneous angular acceleration} of an object is defined as the time derivative of angular velocity.
            \begin{equation}
                \alpha(t)=\lim_{\Delta t\to0}\frac{\omega(t+\Delta t)-\omega(t)}{\Delta t}=\dv{\omega(t)}{t}=\dv[2]{\theta(t)}{t}.
            \end{equation}
        \end{definition}
        Again analogously to the linear case, we define the average angular velocity and angular acceleration as an integral.
        \begin{align}
            \bar{\omega}(t)&=\frac{\Delta\theta}{\Delta t}=\frac{1}{\Delta t}\int_{t_1}^{t_2}\omega(t)\mathrm{d}t\\
            \bar{\alpha}(t)&=\frac{\Delta\omega}{\Delta t}=\frac{1}{\Delta t}\int_{t_1}^{t_2}\alpha(t)\mathrm{d}t.
        \end{align}

    \section{Constant Angular Acceleration}
        \paragraph{}
        % TODO: make references to equations in same section in chapter 1
        In the case where we have a constant angular acceleration, we can derive a set of equation analogous to the SUVAT equations for linear motion.
        From the equation for average acceleration above, we get
        \begin{equation}
            \Delta\omega=\omega(t)-\omega_0=\alpha t.
        \end{equation}
        Then from the definition of angular displacement,
        \begin{align}
            \Delta\theta&=\int_{t_1}^{t_2}(\omega_0+\alpha t)\mathrm{d}t\\
            &=\omega_0t+\frac{1}{2}\alpha t^2\\
            \implies\theta(t)&=\theta_0+\omega_0t+\frac{1}{2}\alpha t^2.
        \end{align}
        Squaring the first equation and substituting it into the second gives the last equation:
        \begin{equation}
            \omega^2(t)=\omega_0^2+2\alpha\theta(t).
        \end{equation}

    \section{Relating Linear and Angular Quantities}
        \paragraph{}
        The arc length along a circle is given by $s=r\theta$, where $r$ is the radius of the circle and $\theta$ is the angle.
        For an object travelling in a circular path, the arc length is the displacement $s(t)$.
        Thus, the velocity along the path is given as
        \begin{equation}
            v=\dv{s}{t}=r\dv{\theta}{t}=r\omega.
        \end{equation}
        Similarly, the acceleration is given by
        \begin{equation}
            a=\dv{v}{t}=r\dv{\omega}{t}=r\alpha.
        \end{equation}
        % TODO: derive vector versions of these quantites
        % TODO: rename this next section "Uniform Circular Motion"?
        Let's look at things from a vector perspective in the case where the object is moving with a constant speed, i.e. $\alpha=0$.
        In this case, $\omega$ is constant and so we have $\theta=\omega t$.
        Using some trigonometry, we can see that the position vector, velocity and acceleration in cartesian coordinates is given by
        \begin{align}
            \vec{r}=r\cos(\omega t)\ihat+r\sin(\omega t)\jhat\\
            \vec{v}=\dv{\vec{r}}{t}=-r\omega\sin(\omega t)\ihat+r\omega\cos(\omega t)\jhat\\
            \vec{a}=\dv{\vec{v}}{t}=-r\omega^2\cos(\omega t)\ihat-r\omega^2\sin(\omega t)\jhat.
        \end{align}
        As a sanity check, we can see that
        \begin{align}
            \abs{\vec{v}}=\sqrt{v_x^2+v_y^2}&=\sqrt{(-r\omega\sin(\omega t))^2+(r\omega\cos(\omega t))^2}\\
            &=\sqrt{r^2\omega^2(\sin^2(\omega t)+\cos^2(\omega t))}=r\omega,
        \end{align}
        which is what we found before.
        Let's look at the dot product of the vectors.
        \begin{align}
            \vec{v}\cdot\vec{r}&=(-r\omega\sin(\omega t)\ihat+r\omega\cos(\omega t)\jhat)\cdot(r\cos(\omega t)\ihat+r\sin(\omega t)\jhat)\\
            &=-r^2\omega\sin(\omega t)\cos(\omega t)+r^2\omega\sin(\omega t)\cos(\omega t)\\
            &=0.
        \end{align}
        Hence the velocity and position vectors are perpendicular.
        This makes sense because as we know from before, the velocity is always tangent to the trajectory, which corresponds to being perpendicular to the position vector in the case of a circle.
        Finally, notice that
        \begin{equation}
            \vec{a}=-\omega^2\vec{r}.
        \end{equation}
        The acceleration is antiparallel to the position vector.
        The magnitude of the acceleration is
        \begin{align}
            \abs{\vec{a}}=\sqrt{a_x^2+a_y^2}&=\sqrt{(-r\omega^2\cos(\omega t))^2+(-r\omega^2\sin(\omega t))^2}\\
            &=\sqrt{r^2\omega^4(\sin^2(\omega t)+\cos^2(\omega t))}\\
            &=r\omega^2=\frac{v^2}{r}.
        \end{align}
        Thus the magnitude of the acceleration does not change over time, but the direction constantly changes as the object moves on its circular path.
        By Newton's second law, a nonzero acceleration implies an unbalanced force.
        This force which has constant magnitude and constantly changing direction is what keeps the object moving on its circular path and is known as the \textbf{centripetal force}.
        It is given by
        \begin{equation}
            \vec{F}_\text{centripetal}=m\vec{a}=-\frac{mv^2}{r}\hat{r}.
        \end{equation}
        \begin{example}
            Consider a child on a merry-go-round.
            If the platform is rotating at 60rpm and the child is holding on, what is the force on the child's arm?
            % TODO: complete this example
        \end{example}
        \begin{example}
            Consider a conical pendulum. 
            If the bob of mass of 200g on a string of length 50cm is swinging around at a frequency of 1 rotation per second, what is the angle that the pendulum makes with the vertical?
            % TODO: complete this example
        \end{example}
        \begin{example}
            Consider a car going round a circular bend.
            Find the maximum velocity that the car can take the bend at without skidding.
            % TODO: complete this example
        \end{example}
        \begin{example}
            Consider a ball rolling on a circular banked curve.
            What is the speed required to maintain a constant height on the curve as a function on the banking angle?
            % TODO: complete this example 
        \end{example}
        This next example uses the concept of energy conservation in combination with the traditional method of comparing forces to solve the problem.
        \begin{example}
            Consider a mass on a string.
            The mass starts hanging vertically downwards, then it gets projected sideways at a speed $v_0$.
            When the angle between the string and the vertical is $120^\circ$, the string becomes slack and the mass falls.
            Find the initial speed $v_0$ in terms of the length of the string.
            % TODO: complete this example
            First use energy conservation to relate the change in kinetic energy to the change in gravitational potential energy.
            % TODO: explain why this is valid since gravitational force is conservative and centripetal force doesn't change magnitude of velocity
            Then evaluate the forces at the top of the path to get a formula for the final velocity.
            Finally use the conservation of energy to solve for $v_0$.

            \paragraph{}
            How fast would the mass haves to be projected to get to the top of the loop?
        \end{example}

    \section{Non-uniform Angular Acceleration}
        \paragraph{}
        Suppose now that the angular acceleration $\alpha$ changes with time.
        % TODO: include derivation of general motion in 2D formulae

    % TODO: mechanics of a spool

\end{document}
