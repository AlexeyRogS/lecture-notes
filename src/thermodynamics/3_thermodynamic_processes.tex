\documentclass[../thermodynamics.tex]{subfiles}

\begin{document}

    \section{Reversible \& Irreversible Processes}
        \paragraph{}
        A reversible process is one in which after it is carried out, the system can be restored to its starting conditions without any change in the rest of the universe.
        For irreversible processes, this is not possible.
        For example, free expansion of an ideal gas is irreversible (a gas won't contract by itself).
        The gas does no work as it expands, its internal energy does not change and neither does its temperature.
        We call processes where no energy is transferred by heating \textbf{adiabatic}.
        \begin{equation}
            Q=0,\quad\Delta U=W.
        \end{equation}
        Free expansion is adiabatic, but not all adiabatic processes are irreversible.

        \paragraph{}
        Imagine we have an ideal gas contained in a box with a piston on the top face.
        If we decrease the force on the piston, the gas will be allowed to expand slowly.
        The work done by the gas on the piston is equal to $W$, which is equivalent to $-W$ being done on the gas.
        \begin{equation}
            \dbar{W}=-F\mathrm{d}z.
        \end{equation}
        The force on the piston $F$ is equal to the gas pressure times area $PA$, and noting that $A\mathrm{d}z=\mathrm{d}V$, we get
        \begin{equation}
            \dbar{W}=-P\mathrm{d}V.
        \end{equation}
        Using the first law (since $Q=0$), we get that
        \begin{equation}
            \mathrm{d}U=-P\mathrm{d}V,
        \end{equation}
        and hence
        \begin{equation}
            W=-\int_{V_i}^{V_f}P(V)\mathrm{d}V.
        \end{equation}
        This is valid for an adiabatic process of slow expansion/compression, which is also reversible.
        This integral can be visualised as the area under a plot of pressure $P$ as a function of volume $V$ (a $P$-$V$ diagram).

        \paragraph{}
        On a $P$-$V$ diagram, two different paths can have the same set of initial and final conditions and yet the work can be different because work is not a function of state, it is \textbf{path dependent}.
        For some processes, like free expansion, the process is not even continuous so it is impossible to draw the path of the system on a $P$-$V$ diagram.
        In the last chapter, we saw that $\mathrm{d}U=\frac{f}{2}Nk_B\mathrm{d}T$.
        Using this and the ideal gas law, we get
        \begin{align}
            -\frac{\mathrm{d}V}{V}&=\frac{f}{2}\frac{\mathrm{d}T}{T}\\
            -\int_{V_i}^{V_f}\frac{\mathrm{d}V}{V}&=\frac{f}{2}\int_{T_i}^{T_f}\\
            \implies T_iV_i^{\frac{2}{f}}&=T_fV_f^{\frac{2}{f}}\\
            P_iV_i^{1+\frac{2}{f}}&=P_fV_f^{1+\frac{2}{f}}.
        \end{align}
        If we define $\gamma=1+\frac{2}{f}$, we get
        \begin{equation}
            P_iV_i^\gamma=P_fV_f^\gamma.
        \end{equation}
        Se reversible adiabatic processes for an ideal gas follow lines where this quantity $PV^\gamma$ is constant.
        These lines are called \textbf{adiabats}.
        % TODO: include diagram for this
        One the other hand, \textbf{isothermal processes}, ones that occur at constant temperature follow \textbf{isotherms}, which are given by $P\propto V^{-1}$.
        Thus adiabats always drop off faster with volume than isotherms since $\gamma>1$.

    \section{Heat Capacity}
        \paragraph{}
        Heat capacity tells us the amount of heat we need to add to raise a system's temperatre by one unit.
        \begin{equation}
            C=\frac{\dbar{Q}}{\mathrm{d}T}.
        \end{equation}
        Note that this depends on how much substance our system contains.
        Specific heat capacity is defined for a unit mass:
        \begin{equation}
            c=\frac{C}{m}=\frac{1}{m}\frac{\dbar{Q}}{\mathrm{d}T}.
        \end{equation}
        From the first law of thermodynamics, we know that
        \begin{equation}
            \dbar{Q}=\mathrm{d}U-\dbar{W},
        \end{equation}
        and hence,
        \begin{equation}
            C=\frac{\mathrm{d}U-\dbar{W}}{\mathrm{d}T}.
        \end{equation}
        This means that heat capacity can be different depending on the work done.
        We need to specify some more conditions so that heat capacity is useful.
        Using $\dbar{W}=-P\mathrm{d}V$, we get
        \begin{equation}
            C=\dv{U}{T}+P\dv{V}{T}.
        \end{equation}
        Now we can define two different heat capacities, one at constant volume where $\mathrm{d}V=0$ and one at constant pressure.
        \begin{align}
            C_v&=\left(\pdv{U}{T}\right)_V\\
            C_P&=\left(\pdv{U}{T}\right)_P+P\left(\pdv{V}{P}\right)_P.
        \end{align}
        Now using the equipartion theorem $U=\frac{f}{2}Nk_BT$, we get
        \begin{align}
            C_V&=\frac{f}{2}Nk_B\\
            C_p&=\frac{f}{2}Nk_B+P\frac{\partial}{\partial T}\left(\frac{Nk_BT}{P}\right)\\
            &=\left(1+\frac{f}{2}\right)Nk_B.
        \end{align}
        So, we see that the ratio of the two specific heats is
        \begin{equation}
            \frac{C_P}{C_V}=1+\frac{2}{f}=\gamma.
        \end{equation}

        \paragraph{}
        Sometimes we can add or remove energy from a system without changing the temperature e.g. during a first-order phase transition.
        These transitions involve \textbf{latent heat}, which is the amount of heat needed to make a unit mass of substance undergo the phase change.
        \begin{equation}
            L=\frac{Q}{m}.
        \end{equation}
        However, this is still ambiguous as it doesn't specify the conditions that the system is in during the phase change.
        We usually say that the substance is under constant pressure and the only work done in the process is expansion against the atmosphere.

\end{document}
